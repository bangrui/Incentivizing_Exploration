\section{Introduction} \label{sec:introduction}

Many websites and apps facilitate joint discovery,
sharing, and recommendation of content.
This includes news-, photo-, and video-sharing sites;
sites that host user-written reviews of products, restaurants, hotels, and travel experiences;
and citizen science projects
such as eBird \citep{sullivan2009ebird,xue-ebird} and Galaxy Zoo \citep{lintott-galaxy-zoo}.
Users of these sites improve their own experiences 
by learning from the experiences of others 
\citep{schmit2017human}.

Viewed more abstractly, users jointly explore a space of
options (products, news stories, photos, birdwatching sites,
%patches of sky to train their telescopes on, 
$\ldots$),
with the implicit goal of identifying the ``best'' ones.
Such scenarios are often modeled in a bandit
learning framework.
However, unlike standard bandit settings, the utilities of the
decision makers (the users) are not aligned with overall utility.
Societally (i.e., in a suitable aggregate over users),
substantial exploration of options could provide higher future rewards.
However, individual users interact with the site a limited number
of times, and therefore have little incentive to explore.
Recent work has focused on the setting in which users interact only once, 
and therefore have no intrinsic incentive to explore.

Two recent lines of work have shown that
effecting a societally near-optimal outcome in this setting requires
explicitly inducing exploration:
\citet{kremer2014implementing}
and
\citet{mansour2015bayesian,mansour2016bayesian,mansour2018competition}
(see \cite{slivkins:asymmetry} for an overview)
assume that the site (also called the \emph{principal}) has an
informational advantage in being the only one to observe the results
of past arm pulls
(as in driving route recommendations).
The principal can use her%
\footnote{We use male pronouns to refer to users and female pronouns
  to refer to the principal.}
advantage to induce exploration by recommending apparently sub-optimal arms,
as long as agents cannot do better on their own.
\citet{frazier2014incentivizing} and 
\citet{han2015incentivizing} instead assume that the results of all
past arm pulls are publicly observable
(as on a review-sharing site).
They suppose that the principal can incentivize exploration by offering arm-specific reward payments.
We follow this second model.

Past work has assumed that users have homogeneous preferences over
arms, i.e., the expected reward derived from an arm is the same for all users.
(\citet{han2015incentivizing} model users that are
heterogeneous in their tradeoff between utility derived from arm pulls
and utility derived from the principal's payment,
but assume that preferences across arms are homogenous.)
In reality, users have different preferences, e.g., gastronomic,
political, aesthetic, practical.
Indeed, the websites and mobile apps most widely used for joint discovery,
sharing, and recommendation of content tend to concern products and items 
with heterogeneity in preferences (movies, restaurants,
videos, travel experiences), and not items with a universally
agreed-on best order.
This is perhaps because regimes with heterogeneous preferences are the
ones where people have the most difficulty discovering the best items, 
and thus where online platforms provide the most value.  
Thus, we see an appropriate
accounting for heterogeneity as critical to incentivizing
exploration in online communities.

In this work, we present the first algorithm and analysis (of which we
are aware) for incentivizing exploration when users have heterogeneous 
preferences over arms.
Our analysis provides insight into the
impact of user heterogeneity on the principal's ability to achieve
high social utility with low incentive payments,
and on the best approaches for doing so.
Heterogeneity presents both a challenge and an opportunity.
On the one hand, unobserved heterogeneity hides critical
information about an agent's preferences from the principal,
making her unaware of agents' preferred arms.
Thus, even with an unlimited incentive budget, incentivizing 
pulls according to a standard multi-armed bandit algorithm does not provide
low regret.  Instead, she must use incentives sparingly and 
allow agents to reveal their preferences through action.
On the other hand, heterogeneity also offers
the possibility of ``free exploration.''
Even unincentivized, agents will
play a variety of arms, revealing information about their attributes.
This stands in sharp contrast to the case of homogeneous preferences,
where unincentivized agents will herd onto a single apparently best arm,
and where effecting essentially any exploration at all requires incentives.


% With these challenges and opportunities in mind, 
% our goal is to understand the
% impact of user heterogeneity on the principal's ability to achieve
% high social utility with low incentive payments,
% and on the best approaches for doing so.
% We wish to understand whether incentivizing exploration with
% heterogeneous preferences  is ``harder'' or ``easier'' than with
% homogeneous ones,
% and to understand how exploration strategies that do well in the
% heterogeneous preference setting differ from those in the homogeneous one.
% Toward that end, we model our setting as follows.
We describe our model at a high level here,
with formal definitions given in Section~\ref{sec:prob}.
The \ARMNUM arms and users (or \emph{agents})
are characterized by payoff-relevant
\emph{attribute} (or \emph{feature}) vectors.
Arms' attributes are a priori unknown,
and agents' attributes are drawn from a known distribution.
An agent's reward from pulling an arm is the inner product of his
vector with the arm's vector (plus noise).
When an arm is pulled, a noisy version of its attribute vector is
observed by everyone.
This models full-text product reviews on websites like Amazon,
or ratings of ``service'', ``value'',
and other restaurant attributes on websites like Tripadvisor.  
Agents are myopic and will pull the arm whose expected attribute
vector (based on past noisy observations) maximizes their reward.
This assumption requires agents to have access to aggregate
information about the feature vector estimates of all (relevant) arms. 
Such access is facilitated by the platforms' ability to aggregate and
effectively display past feedback, e.g., by displaying simple averages
of ratings and using automatic summarization of full-text reviews 
\citep{wang2010product,liu2012movie,abulaish2009feature}.
While platforms could in principle manipulate the display of such
data, most have an incentive to be seen as truthful.
%\pfedit{Agents' ability to effectively base decisions on all past observations models platforms' abilities to aggregate and effectively display past feedback, through simple averages of ratings and automatic summarization of full-text reviews \citep{wang2010product,liu2012movie,abulaish2009feature}.  While platforms can manipulate the display of user data, most have an incentive to be seen as truthful.}
The principal can incentivize agents to pull particular arms by
offering arm-specific payments.
The principal's goal is to keep the cumulative regret across all
agents small, while incurring only small total payments.

Our main theorem can be stated informally as follows:

%One example where agents give noisy observations of items’ feature vectors is restaurant reviews on TripAdvisor.  Users provide not just an overall rating, but also separate ratings for Service, Value, Food and Atmosphere.  We may treat these as our feature vector.  Platforms collect similar explicit feature vector observations whenever their users rate not just satisfaction with an item but the more granular components that contribute to it.

%As another example, consider websites like Amazon and Yelp whose users provide full-text reviews of items.  Reviews on Yelp describe a restaurants’ attributes (service, value, etc.), and reviews on Amazon similarly describe products’ attributes.  Thus, we can think of a user reading past reviews as aggregating past users’ observations of items’ feature vectors.  In this way, platforms with full-text reviews implicitly provide feature vector observations.

%As a third example, a platform with full-text reviews could run an NLP algorithm to extract features and sentiments from reviews, and then surface them to users as rated attributes.


\begin{theorem} \label{thm:main-intro}
Assume that for each arm, at least a constant fraction \MinProb
of the population likes this arm best,
and let \TieDensity be a measure of the ``density of near-ties''
between agents' arm preferences
(in a sense made precise in Section~\ref{sec:prob}).
There is a policy that achieves expected 
cumulative regret $O (\ARMNUM \e^{2/\MinProb} + \TieDensity \ARMNUM \log^3(T))$,
using expected cumulative payments of $O(\ARMNUM^2 \e^{2/\MinProb})$.
In particular, when agents who are close to tied between two arms have measure $0$,
both the expected regret and expected payment are bounded by constants
(with respect to $T$). 
\end{theorem}


The policy achieving the result of Theorem~\ref{thm:main-intro} is
simple: it mostly lets agents exploit arms, but incentivizes
them to explore when arms appear unlikely to be pulled without incentives.
It is presented in detail in Section~\ref{sec:ub}.
