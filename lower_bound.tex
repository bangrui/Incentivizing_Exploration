\section{A Lower Bound of $\Omega(\log(T))$} \label{sec:lb}

We saw in Corollary~\ref{cor:constant-regret} that
when agent preferences for their best arm are sufficiently clear,
in the sense that $\TieDensity = 0$, the regret of
Algorithm~\ref{alg:basic-incentivizing} is bounded by a constant.
One may conjecture that this should hold more generally,
in that the regret of the (fewer and fewer) agents on the boundary
between close arms should go to 0, while their fraction also goes to 0.
In this section, we establish a lower bound,
showing that even in very simple settings, a (logarithmic) dependence
on $T$ is typicaly unavoidable for \emph{any} incentivization strategy.

We consider an instance with two arms,
whose attribute vectors are $(0,0)$ and $(0,1)$, respectively.
Agents types are distributed uniformly on the (edge of the)
two-dimensional unit square%
\footnote{While our model technically assumed that all agent coordinates are
non-negative, we could simply shift the unit square.
The present choice is solely for ease of notation.}  
$\Set{\AgV}{\max(\Ag{1}, \Ag{2}) = 1}$,
with density $\frac{1}{8}$.
Because $\AgV \cdot (0,0) = 0$ and $\AgV \cdot (0,1) = \Ag{2}$,
the best choice for agent \AgV is arm $(0,1)$ iff $\Ag{2} > 0$,
i.e., the top half of the unit circle square the arm $(0,1)$,
and the bottom half prefers the arm $(0,0)$.

Since we are proving a lower bound, we give the algorithm the
following extra two advantages:
(1) there is no noise in the observations of the arm $(0,0)$,
and all agents know its location deterministically.
(2) in each time step, regardless of which arm is pulled, the algorithm
and all agents observe a pull from arm $(0,1)$.
For simplicity of notation, we set the standard deviation of the arm
$(0,1)$ to $\sigma = 1$;
different values only lead to a scaling of the results.

Under these advantages, myopic play is clearly optimal, so it suffices
to bound the regret of the myopic algorithm which never incentivizes
agents. \dkcomment{This feels ``clear'', but is it?}

Because a pull of arm $(0,1)$ is observed in each time step,
after $t$ rounds, the estimate \ArmEV{t}{(0,1)} is of the form
$(0,1) + (\Noise[t]{1}, \Noise[t]{2})$,
where $\Noise[t]{1} \sim \Normal{0}{\sqrt{1/t}}$
and $\Noise[t]{2} \sim \Normal{0}{\sqrt{1/t}})$.
\dkcomment{I think we define this to be the estimates at the beginning
  of round $t$, so we may be off by 1 here and in other places.}
We lower-bound the regret in step $t$ by focusing on the event that
both normal noise coordinates are non-negative,
which by symmetry has probability \quarter:

\begin{align*}
\Expect{\Regret{t}} 
  & \geq \ExpectC{\Regret{t}}{\Noise[t]{1} > 0, \Noise[t]{2} > 0}
         \cdot \Prob{\Noise[t]{1} > 0, \Noise[t]{2} > 0}
  \; = \; \quarter \ExpectC{\Regret{t}}{\Noise[t]{1} > 0, \Noise[t]{2} > 0}.
\end{align*}

For the moment, focus on some time step $t$,
and write $\NoiseV = \NoiseV[t]$.
Then, there are two types of agents who pull the wrong arm and incur
regret:
\begin{enumerate}
\item If $\Ag{2} > 0$ and $\Ag{1} \Noise{1} + \Ag{2} (1+\Noise{2}) < 0$
then \AgV should pull $(0,1)$,
but will wrongly pull $(0,0)$ and incur regret \Ag{2}.
The range of \AgV making the wrong choice is thus
$0 < \Ag{2} < \frac{- \Noise{1}}{1+\Noise{2}} \cdot \Ag{1}$.
Since we are proving a lower bound, we only focus on the
case $\Ag{1} = -1$, and ignore the case $\Ag{2} = 1$
(which is rare, since it requires \Noise{1} to be large).
Thus, the set of agents incurring regret contains the set
$\Set{\AgV = (-1, \Ag{2})}{0 < \Ag{2} < \frac{\Noise{1}}{1+\Noise{2}}}$.

% For any particular value of \Ag{2},
% the length of the interval of corresponding \Ag{1} is
% \begin{align*}
%   \sqrt{1-\Ag{2}^2} - \frac{1+\Noise{2}}{\Noise{1}} \cdot \Ag{2}
%   & \geq 1 - \Ag{2} - \frac{1+\Noise{2}}{\Noise{1}} \cdot \Ag{2}
%   \; = \; 1 - \frac{1+\Noise{1}+\Noise{2}}{\Noise{1}} \cdot \Ag{2}.
% \end{align*}

\item If $\Ag{2} < 0$ and $\Ag{1} \Noise{1} + \Ag{2} (1+\Noise{2}) > 0$,
then \AgV should pull $(0,0)$,
but will wrongly pull $(0,1)$ and incur regret $-\Ag{2}$.
This region and its regret are rotationally symmetric to the previous
case.
\end{enumerate}

Hence, the expected regret for given \Noise{1}, \Noise{2} is at least
%\begin{align*}
$2 \int_0^{\frac{\Noise{1}}{1+\Noise{2}}}
  \Ag{2} \cdot \frac{1}{8} \dd \Ag{2}
=
\frac{1}{8} \cdot \left(\frac{\Noise{1}}{1+\Noise{2}}\right)^2$.
%\end{align*}
The expected regret, conditioned on
$\Noise{1} > 0$ and $\Noise{2} > 0$, is therefore

\dkcomment{Where did the 4 come from? Where did it go? Also, why did
  you have $\pi^3$ (instead of $\pi^2$) before? (I changed it.)}
\begin{align*}
\ExpectC{\Regret{t}}{\Noise{1} > 0, \Noise{2} > 0}
 & \geq \dkedit{4??} \cdot \frac{1}{8}
    \int_{0}^{\infty} \int_{0}^{\infty}
    \left( \frac{\Noise{1}}{1+\Noise{2}} \right)^2 \cdot
    \frac{\e^{-\frac{t \Noise{1}^2}{2}} \sqrt{t}}{\sqrt{2\pi}} \dd\Noise{1}
    \frac{\e^{-\frac{t \Noise{2}^2}{2}} \sqrt{t}}{\sqrt{2\pi}} \dd\Noise{2} \\
 & = \frac{1}{16\pi} \int_{0}^{\infty} \int_{0}^{\infty}
   \left( \frac{\Noise{1}}{\sqrt{t}+\Noise{2}} \right)^2
   \e^{-\Noise{1}^2/2} \e^{-\Noise{2}^2/2} \dd \Noise{1} \dd \Noise{2}.
\end{align*}

We want to show that the expected regret per time step decreases only
at a rate of $\Omega(1/t)$, and thereto consider the limit of the
ratio of the two quantities:

\begin{align*}
\lim_{t \to \infty} \frac{\ExpectC{\Regret{t}}{\Noise{1} > 0, \Noise{2} > 0}}{\frac{1}{t}}
  & \geq
    \lim_{t \to \infty} \left( \frac{1}{16\pi} \cdot
    \int_{0}^{\infty} \int_{0}^{\infty}
    t \cdot \left( \frac{\Noise{1}}{\sqrt{t}+\Noise{2}} \right)^2
    \e^{-\Noise{1}^2/2} \e^{-\Noise{2}^2/2} \dd \Noise{1} \dd \Noise{2} \right) \\
  & = \frac{1}{16\pi} \cdot \int_{0}^{\infty} \int_{0}^{\infty}
    \lim_{t \to \infty}
    \left( t \cdot \left( \frac{\Noise{1}}{\sqrt{t}+\Noise{2}} \right)^2 \right)
    \e^{-\Noise{1}^2/2} \e^{-\Noise{2}^2/2} \dd \Noise{1} \dd \Noise{2} \\
  & = \frac{1}{16\pi} \cdot \int_{0}^{\infty} \int_{0}^{\infty}
    \e^{-\Noise{1}^2/2} \e^{-\Noise{2}^2/2} \dd \Noise{1} \dd \Noise{2}
  \; = \; \frac{1}{16 \pi}; 
\end{align*}

the second line is due to the monotone convergence theorem,
which can be applied because
$t \left(\frac{\Noise{1}}{\sqrt{t}+\Noise{2}}\right)^2$
is strictly increasing in $t$.
Because the expected regret in each time step is at least
$\Omega(1/t)$, the total expected regret is at least
$\Omega(\sum_{t=1}^{T}\frac{1}{t} = \Omega(\log(T))$.
