\section{Proof of Theorem~\ref{rst:budget}}
\label{sec:payment-proof}

We restate the theorem here for convenience:

\begin{rtheorem}{Theorem}{\ref{rst:budget}}
The payment budget for Algorithm~\ref{alg:basic-incentivizing} is bounded above by 
\begin{align}
O \left(\ARMNUM^2 \MAXR \Diam^2 \TieDensity d^3 \sigma\cdot \exp(\frac{2}{\MinProb}) \right).  \nonumber 
\end{align}
\end{rtheorem}

The high-level idea of the proof is motivated by Lemma~\ref{lem:numP},
which shows that the expected \emph{number} of payments is constant.
Unfortunately, in contrast to the regret, there is no hard upper bound
on the payment in any one round.
If a draw of a particular arm comes out wildly inaccurate --- which is
an event of low but positive probability ---
then agents may require very large incentives to pull this arm again
in the future (and correct the inaccurate estimate).
The high payments are offset by the exceedingly low probability of
having to incur them, but a rigorous analysis requires some care:
if a high payment is required in one phase, this indicates a very
inaccurate estimate, which may require multiple phases to correct.
Hence, we need to handle dependency of payments across time steps and
phases.

To reason about such estimation errors formally, we define
\emph{envelopes} of sample paths.
A sample path \SP captures all the random events affecting the
algorithm, i.e., the random draws \AgentV{t} of agents and the
noise \NoiseV[t] in the draws the pulled arms,
with an infinite time horizon.

With foresight, we define $g(s, \ell) := \frac{12 \sigma \ell}{s^{2/5}}$.
Let $\ErrV{t}{i} = \ArmEV{t}{i} - \ArmV{i}$ be the estimation
error for the attribute vector \ArmV{i} at time $t$,
with components \Err{t}{i}{j}.
For any sample path \SP, let $s (t,\SP)$ be the phase number that the
algorithm is in at time $t$ with the sample path \SP.
Define the sets

\begin{align*}
\hat{\mathcal{L}}_\ell
  & = \Set{\SP}{|\Err{t}{i}{j}(\SP)| \leq g(s(t,\SP),\ell)
    \mbox{ for all } i,j,t},\\
\Env{1} & = \hat{\mathcal{L}}_1,\\ 
\Env{\ell} & = \hat{\mathcal{L}}_{\ell} \setminus \hat{\mathcal{L}}_{\ell-1}
  \quad \mbox{ for } \ell \geq 2.
\end{align*}

% \begin{align*}
% L'[\ell](t)
%   & = \Set{\SP}{|\Err{t}{i}{j}(\SP)|\leq g(s(t,\SP),\ell), \forall i,j\}
% \end{align*}
We call \Env{\ell} the \Kth{\ell} envelope.
In words, \Env{\ell} consists of all sample paths such that at all times
$t$, all coordinates of all arm estimation errors are bounded by
$g(s, \ell)$,
but for at least one time $t$, at least one coordinate of one arm
estimation error is \emph{not} bounded by $g(s, \ell-1)$.
When \SP is clear, we omit it in the notation for
\Err{t}{i}{j}, payments, etc.
The importance of envelopes is that for small $\ell$, the payments are
tightly bounded, while for large $\ell$, the cumulative probability of
the sample paths in \Env{\ell} is small.
This is captured by the following two lemmas.

\begin{lemma} \label{lem:sample-path-payment}
If $\SP \in \Env{\ell}$ and $s(t,\SP) = s$, then
the payment in step $t$ is upper-bounded by
$\bar{c} (s,\ell) = \MAXR + 2 \Diam d \cdot g(s,\ell)$.
\end{lemma}

\begin{proof}
The maximum payment is upper-bounded by the maximum perceived
difference in value for any agent type and any two arms:
\begin{align*}
\bar{c} (s, \ell) & \leq 
\max_{\AgV} \left(  \max_{i} \AgV \cdot \ArmEV{t}{i}
                 - \min_{i'} \AgV \cdot \ArmEV{t}{i'} \right) \\
& = \max_{\AgV} \left( \max_{i}\AgV \cdot (\ErrV{t}{i}+\ArmV{i})
                    - \min_{i'}\AgV \cdot (\ErrV{t}{i'}+\ArmV{i'}) \right) \\
& \leq \max_{\AgV} \left(  \max_{i} \AgV \cdot \ArmV{i}
                        - \min_{i'} \AgV \cdot \ArmV{i'}
                        + \max_{i} \AgV \cdot \ErrV{t}{i}
                        - \min_{i'} \AgV \cdot \ErrV{t}{i'} \right) \\
& \leq \MAXR + 2 \Diam d \cdot g(s,\ell). 
\end{align*}
The final inequality used the definition of the envelope.
\end{proof}

\begin{lemma} \label{lem:envelope-probability}
For every $\ell \geq 2$, we have that
$\Prob{\SP \in \Env{\ell}} \leq 24 \ARMNUM d\exp(-1.8(\ell-1)^2)$. 
\end{lemma}

\begin{proof}
For \SP to be in \Env{\ell}, by definition,
for at least one time $t$,
at least one coordinate of at least one arm's estimation error must
exceed $g(s(t,\SP),\ell-1)$.
For now, fix an arm $i$ and coordinate $j$.

Define $x_s := \frac{4 \sigma (\ell-1)}{s^{1/2}}$ and
$x'_s := g(s, \ell-1) = \frac{12 \sigma (\ell-1)}{s^{2/5}}$.

Recall that $\NumPull{t}{i}(\SP) \geq s(t,\SP)$ is the number of times
that arm $i$ has been pulled by time $t$ under \SP.
Let the random stopping time $\tau_{i,j}$ be the first value of
\NumPull{t}{i} (i.e., the first pull of arm $i$) such that 
the estimation error of attribute $j$ of arm $i$ exceeds
$g(\NumPull{t}{i}, \ell-1)$.
$\tau_{i,j} = \infty$ means that the estimation error never exceeds 
$g(\NumPull{t}{i}, \ell-1)$.
We next verify that $x_s, x'_s$ as we defined them satisfy the 
assumption 
\[
  \sqrt{0.6 n \cdot \log (\log_{1.1}(n) + 1) + \frac{n x_n^2}{16 \sigma^2}}
  \; \leq \; \frac{n x'_n}{2 \sigma}
\]
of Lemma~\ref{lem:round-prob}, for all $n \geq 1$.
First, we show that
$\sqrt{0.6n \log(\log_{1.1}(n)+1)} \leq 5 n^{3/5}$
for all $n \geq 1$.
By squaring the inequality and canceling out a factor $n$,
the statement is equivalent to showing that
$25 n^{1/5} \geq \log(\log_{1.1}(n)+1)$.
Indeed, a stronger statement holds, namely, that
$25 n^{1/5} \geq \log_{1.1}(n)+1$.  
To see this, notice that the derivative of the left-hand side is
always strictly larger than the derivative of the right-hand side,
so the difference between the sides is minimized at $n=1$,
where it is positive. 
Using this inequality and subadditivity of $\sqrt{\cdot}$, we can bound

\[
  \sqrt{0.6 n \cdot \log (\log_{1.1}(n) + 1) + \frac{n x_n^2}{16 \sigma^2}}
\; \leq \;
  5 n^{3/5} + \frac{\sqrt{n} x_n}{4 \sigma}
\; = \; 
  5 n^{3/5} + (\ell-1)
  \; \leq \; \frac{n x'_n}{2 \sigma}.
\]

\dkcomment{Unless I made a calculation mistake, we could replace a 12
  with a 10.}

Applying Lemma~\ref{lem:round-prob} to $\tau_{i,j}$,
we conclude that the probability that the error in coordinate $j$ of
arm $i$ exceeds $x'_s = g(s,\ell-1)$ at time $\tau_{i,j}$
(and hence at any time, by definition of $\tau_{i,j}$)
is at most
$24 \exp\left(-\frac{1.8 s x_s^2}{16 \sigma^2} \right)
\; = \; 24 \exp(-1.8 (\ell-1)^2)$.
Now, taking a union bound over all arms $i$ and
coordinates $j$ completes the proof.
\end{proof}

Next, we show that for any sample path \SP in the envelope \Env{\ell},
we can bound the total number of payments made in terms of $\ell$.
Define $h(\ell) := \max \left( \exp(\frac{2}{\MinProb}),
\left( \frac{48 \sigma \ell \TieDensity \Diam d}{\MinProb} \right)^{5/2}
\right)$.

\begin{lemma} \label{lem:envelope-payments}
Let \SP be a sample path in \Env{\ell}.
Then, under \SP, the algorithm makes payments at most for the first 
$h(\ell)$ phases. 
\end{lemma}

\begin{proof}
The proof is similar to that of Lemma~\ref{lem:numP}.
Fix a sample path $\SP \in \Env{\ell}$.
Consider a phase $s > h(\ell)$ and a time $t$ in phase $s$.
Because $\SP \in \Env{\ell}$, in particular, all estimation errors are
bounded at time $t$, in the sense that the event \AccEU{t}{g(s,\ell)}
happens.
Because $s \geq h(\ell)$, we get that
$g(s,\ell) \leq \frac{\MinProb}{4 \Diam d \TieDensity}$
as well as $s \geq \exp(2/\MinProb)$.
Therefore, we can apply Lemma~\ref{lem:no-incentives} and conclude
that the pull at time $t$ was not incentivized.
\end{proof}

\begin{extraproof}{Theorem~\ref{rst:budget}}
We can write the total expected payment as
\begin{align*}
  \Expect{\sum_{t=1}^{\infty} \PayA{t}}
& = \sum_{\ell = 1}^{\infty} \sum_{\SP \in \Env{\ell}}
  \Prob{\SP} \cdot \sum_{t=1}^{\infty} \PayA{t}(\SP)
\; = \; \sum_{\ell = 1}^{\infty} \sum_{\SP \in \Env{\ell}}
  \Prob{\SP} \cdot \sum_{s=1}^{\infty} \sum_{t: s(t,\SP) = s} \PayA{t}(\SP).
\end{align*}

By Lemma~\ref{lem:envelope-payments},
the payments are 0 for $s > h(\ell)$,
and by Lemma~\ref{lem:sample-path-payment},
when $\SP \in \Env{\ell}$ and $s(t,\SP) = s$, we can bound
$\PayA{t}(\SP) \leq \MAXR + 2 \Diam d \cdot g(s,\ell)$.
Furthermore, in any one phase, because each arm is incentivized at
most once, there are at most \ARMNUM payments total.
Substituting these bounds, we obtain that

\begin{align*}
  \Expect{\sum_{t=1}^{\infty} \PayA{t}}
& \leq \sum_{\ell = 1}^{\infty} \sum_{\SP \in \Env{\ell}} \Prob{\SP} \cdot
  \sum_{s=1}^{h(\ell)} \ARMNUM \cdot (\MAXR + 2 \Diam d \cdot g(s,\ell))
\\ & = \ARMNUM \cdot
  \sum_{\ell = 1}^{\infty} \Prob{\SP \in \Env{\ell}} \cdot
  \sum_{s=1}^{h(\ell)} \left( \MAXR + \frac{24 \Diam d \sigma \ell}{s^{2/5}} \right).
\end{align*}

We now lower-bound $s^{2/5} \geq 1$, split off the term for $\ell=1$
and bound $\Prob{\SP \in \Env{1}} \leq 1$, and apply
Lemma~\ref{lem:envelope-probability} to the remaining
$\Prob{\SP \in \Env{\ell}}$ terms, to bound

\begin{align*}
\Expect{\sum_{t=1}^{\infty} \PayA{t}}
& \leq
\ARMNUM \cdot \sum_{s=1}^{h(1)} (\MAXR + 24 \Diam d \sigma)
+ \ARMNUM \cdot
  \sum_{\ell = 2}^{\infty} 24 \ARMNUM d\exp(-1.8(\ell-1)^2) \cdot
  \sum_{s=1}^{h(\ell)} (\MAXR + 24 \Diam d \sigma \ell)
\\ & \leq
\ARMNUM \cdot h(1) \cdot (\MAXR + 24 \Diam d \sigma)
+ 24 \ARMNUM^2 d \cdot \sum_{\ell = 2}^{\infty}
     \exp(-1.8(\ell-1)^2) \cdot h(\ell) \cdot (\MAXR + 24 \Diam d \sigma \ell).
\end{align*}
Because $h(\ell)$ and $(\MAXR + 24 \Diam d \sigma \ell)$ grow
polynomially in $\ell$, whereas $\exp(-1.8(\ell-1)^2)$ decreases
exponentially in $\ell$, the sum is dominated by its first term, and
the overall expected payment is bounded by

\begin{align*}
O \left(\ARMNUM^2 d \cdot (\MAXR + \Diam d \sigma) \cdot h(1) \right)
& = 
O \left(\ARMNUM^2 \MAXR \Diam^2 \TieDensity d^3 \sigma
\cdot \exp(\frac{2}{\MinProb}) \right).
\end{align*}
\end{extraproof}
