\section{Proof of Theorem~\ref{rst:budget}}
\label{sec:payment-proof}

We restate the theorem here for convenience:

\begin{rtheorem}{Theorem}{\ref{rst:budget}}
\dkcomment{Copy here the final form of the theorem once it is finalized.}
\end{rtheorem}

The high-level idea of the proof is motivated by Lemma~\ref{lem:numP},
which shows that the expected \emph{number} of payments is constant.
Unfortunately, in contrast to the regret, there is no hard upper bound
on the payment in any one round.
If a draw of a particular arm comes out wildly inaccurate --- which is
an event of low but positive probability ---
then agents may require very large incentives to pull this arm again
in the future (and correct the inaccurate estimate).
The high payments are offset by the exceedingly low probability of
having to incur them, but a rigorous analysis requires some care:
if a high payment is required in one phase, this indicates a very
inaccurate estimate, which may require multiple phases to correct.
Hence, we need to handle dependency of payments across time steps and
phases.

To reason about such estimation errors formally, we define
\emph{envelopes} of sample paths.
A sample path \SP captures all the random events affecting the
algorithm, i.e., the random draws \AgentV{t} of agents and the
noise \Noise[t] in the draws the pulled arms,
with an infite \dkcomment{Is it infinite?} time horizon.
\dkcomment{We should be consistent about using filtration vs.~sample
  path.}
\pfcomment{I removed discussion of filtrations and replaced them by discussion of history.}

With foresight, we define $g(s, \ell) := \frac{12 \sigma \ell}{s^{2/5}}$.
Let $\ErrV{t}{i} = \ArmEV{t}{i} - \ArmV{i}$ be the estimation
error for the attribute vector \ArmV{i} at time $t$,
with components \Err{t}{i}{j}.
For any sample path \SP, let $s (t,\SP)$ be the phase number that the
algorithm is in at time $t$ with the sample path \SP.
Define the sets

\begin{align*}
\hat{L}_\ell(t)
  & = \Set{\SP}{|\Err{t}{i}{j}(\SP)| \leq g(s(t,\SP),\ell)
    \mbox{ for all } i,j},\\
\Env[t]{1} & = \hat{L}_1(t)\\ 
\Env[t]{\ell} & = \hat{L}_{\ell} (t) - \hat{L}_{\ell-1} (t)
  \qquad \mbox{ for } \ell \geq 2.
\end{align*}

% \begin{align*}
% L'[\ell](t)
%   & = \Set{\SP}{|\Err{t}{i}{j}(\SP)|\leq g(s(t,\SP),\ell), \forall i,j\}
% \end{align*}
We call \Env[t]{\ell} the \Kth{\ell} envelope at time $t$.
\pfcomment{I believe we need the following definition for envelope $\ell$: 
$\hat{L}_\ell
  = \Set{\SP}{|\Err{t}{i}{j}(\SP)| \leq g(s(t,\SP),\ell)
    \mbox{ for all } i,j,t}$
This is what we use below in the proof of Theorem 3, plus a corresponding $L_\ell$.
It's ok to have $\hat{L}(t)$ defined as well, but if we are going to define the version without $t$ then maybe we don't need the one wiht $t$.
}


In words, \Env[t]{\ell} consists of all sample paths such that at time
$t$, all coordinates of all arm estimation errors are bounded by
$g(s, \ell)$, but at least one coordinate of one arm estimation error
is \emph{not} bounded by $g(s, \ell-1)$.
When $t$ is clear from the context, we omit it from the notation;
similarly, when \SP is clear, we omit it in the notation for
\Err{t}{i}{j}, payments, etc.
The importance of envelopes is that for small $\ell$, the payments are
tightly bounded, while for large $\ell$, the cumulative probability of
the sample paths in \Env[t]{\ell} is small.
This is captured by the following two lemmas.

\begin{lemma} \label{lem:sample-path-payment}
If $\SP \in \Env[t]{\ell}$ and $s(t,\SP) = s$, then
the payment in step $t$ is upper-bounded by
$\bar{c} (s,\ell) = \MAXR + 2 \Diam d \cdot g(s,\ell)$.
\dkcomment{Verify that the way I talk about $t$ vs.~$s$ is right.}
\end{lemma}

\begin{proof}
The maximum payment is upper-bounded by the maximum perceived
difference in value for any agent type and any two arms:
\begin{align*}
\bar{c} (s, \ell) & \leq 
\max_{\AgV} \left(  \max_{i} \AgV \cdot \ArmEV{t}{i}
                 - \min_{i'} \AgV \cdot \ArmEV{t}{i'} \right) \\
& = \max_{\AgV} \left( \max_{i}\AgV \cdot (\ErrV{t}{i}+\ArmV{i})
                    - \min_{i'}\AgV \cdot (\ErrV{t}{i'}+\ArmV{i'}) \right) \\
& \leq \max_{\AgV} \left(  \max_{i} \AgV \cdot \ArmV{i}
                        - \min_{i'} \AgV \cdot \ArmV{i'}
                        + \max_{i} \AgV \cdot \ErrV{t}{i}
                        - \min_{i'} \AgV \cdot \ErrV{t}{i'} \right) \\
& \leq \MAXR + 2 \Diam d \cdot g(s,\ell). 
\end{align*}
The final inequality used the definition of the envelope.
\end{proof}

\begin{lemma} \label{lem:envelope-probability}
For every $\ell \geq 2$ and $t$, we have that
$\Prob{\SP \in \Env[t]{\ell}} \leq 24 \ARMNUM d\exp(-1.8(\ell-1)^2)$. 
\end{lemma}

\begin{proof}
The proof idea is quite similar to the proof of
Lemma~\ref{lem:round-prob};
however, the specific form of the envelopes necessitates some subtle
changes in the specific forms of the bounds used.
For \SP to be in \Env[t]{\ell}, by definition, at least one coordinate
of at least one arm's estimation error must exceed $g(s(t,\SP),\ell-1)$.
For now, fix an arm $i$ and coordinate $j$.
Recall that $\NumPull{t}{i}(\SP) \geq s(t,\SP)$ is the number of times
that arm $i$ has been pulled by time $t$ under \SP.
We can express a bound on large deviations in terms of
the scaled estimation error
$\frac{\NumPull{t}{i} \cdot \Err{t}{i}{j}}{2\sigma}$:
\begin{align*}
\Prob{|\Err{t}{i}{j}| \geq g(s(t,\SP), \ell-1)}
& = \Prob{\frac{\NumPull{t}{i} \cdot \Err{t}{i}{j}}{2\sigma}
          \geq \frac{6\NumPull{t}{i}(\ell-1)}{s(t,\SP)^{2/5}}}\\
& \leq \; \Prob{\frac{\NumPull{t}{i} \cdot \Err{t}{i}{j}}{2\sigma}
          \geq 6\NumPull{t}{i}^{3/5}(\ell-1)}. 
\end{align*}

We will show below that
$5 n^{3/5} \geq \sqrt{0.6n \log(\log_{1.1}(n)+1)}$
for all $n \geq 1$.
Applying this inequality and sublinearity of $\sqrt{\cdot}$,
we obtain that

\begin{align*}
6 \NumPull{t}{i}^{3/5}(\ell-1)
& \geq 5 \NumPull{t}{i}^{3/5} + \NumPull{t}{i}^{3/5}(\ell-1)  \\
& \geq \sqrt{0.6 \NumPull{t}{i} \log(\log_{1.1}(\NumPull{t}{i})+1)}
     + \sqrt{(\ell-1)^2 \NumPull{t}{i}}  \\
& \geq \sqrt{0.6\NumPull{t}{i}\log(\log_{1.1}(\NumPull{t}{i})+1)+(\ell-1)^2 \NumPull{t}{i}}. 
\end{align*}

Because the scaled estimation error
$\frac{\NumPull{t}{i} \cdot \Err{t}{i}{j}}{2\sigma}$
is a summation of $1/2$-Gaussian random variables, 
we can now apply Lemma~\ref{lem:ACI-inequality}
with $b(n) = (\ell-1)^2$ \dkcomment{omit dependence on $n$ here?},
to obtain that
\begin{align*}
\Prob{|\Err{t}{i}{j}| \geq g(s(t,\SP), \ell-1)}
& \leq 24 \e^{-1.8(\ell-1)^2}.
\end{align*}
Now, taking a union bound over all arms $i$ and
coordinates $j$ completes the proof.

It remains to show that
$5 n^{3/5} \geq \sqrt{0.6n \log(\log_{1.1}(n)+1)}$
for all $n \geq 1$.
By squaring the inequality and canceling out a factor $n$,
the statement is equivalent to showing that
$25 n^{1/5} \geq \log(\log_{1.1}(n)+1)$.
We will show the stronger statement that
$25 n^{1/5} \geq \log_{1.1}(n)+1$.  
To see this, notice that the derivative of the left-hand side is
always strictly larger than the derivative of the right-hand side,
so the difference between the sides is minimized at $n=1$,
where it is positive. 
\end{proof}

\begin{extraproof}{Theorem~\ref{rst:budget}}


\noindent\textbf{Step 3: Rewrite the total payment expression}.

\pfcomment{I deleted Step 2 and replaced it by this simpler argument}
Let $C(s)$ be the cost incurred in phase $s$.
In each phase, we can make at most $N$ payments, because each arm is incentivized at most once.
Noting that each sample path $\omega$ is in exactly one envelope and and using Lemma~\ref{lem:sample-path-payment}, we get
this bound on the cumulative payment:
\begin{align*}
\sum_{t=1}^{\infty}c(t) =\sum_{s=1}^{\infty}C(s)
\leq  \sum_{s=1}^{\infty}\sum_{\ell=1}^{\infty}\mathbbm{1}\{\SP\in \Env{\ell}\} N \bar{c}(s,\ell). 
\end{align*}

Set $g(s,\ell)$ to be $\frac{12\sigma \ell}{s^{2/5}}$.
Since if $|\Arm{i}{j}-\ArmE{t}{i}{j}|\leq \lambda$ is true $\forall i$, $\forall j$, then we know for those $\AgV\in \{\AgV:\AgV\cdot \ArmV{\Best{\AgV}}-\AgV \cdot \ArmV{\Second{\AgV}}> 2\Diam d\lambda\}$, they will correctly identify their best arm. Thus, if $|\Arm{i}{j}-\ArmE{t}{i}{j}|\leq \frac{12\sigma l}{s^{2/5}} \leq \frac{p^{-1}(\frac{\MinProb}{2})}{2\Diam d}$ $\forall i$ and $\forall j$, then the probability that an unincentivized agent would pull arm $i$ is at least $\frac{p}{2}$. Further, if time $t$ is in a phase $s$ that satisfies $1/log(s)\leq \MinProb/2$, then our algorithm will not incentivize pulling any arms. Denote $a_0=\frac{24\Diam d\sigma}{p^{-1}(\frac{\MinProb}{2})}$. In order to have $\frac{12\sigma l}{s^{2/5}}\leq \frac{p^{-1}(\frac{\MinProb}{2})}{2\Diam d}$, it is sufficient to have $s\geq \lceil (a_{0} l)^\frac{5}{2} \rceil$. In order to have $1/\log(s)\leq \frac{\MinProb}{2}$, we need $s\geq \exp(\frac{2}{\MinProb})$. Denote $s_2=\exp(\frac{2}{\MinProb})$. Thus, we know we can only incur regret \pfcomment{this should be payment cost, right?} for sample paths \SP in the $\ell^{th}$ envelope in the first $\max\{s_2,\lceil (a_0 \ell)^\frac{5}{2} \rceil\}$ phases.

Thus,
\begin{align*}
\sum_{t=1}^{\infty}c(t)\leq\sum_{\ell=1}^{\infty}\sum_{s=1}^{\max\{s_2,\lceil (a_0 \ell)^\frac{5}{2} \rceil\}}\bar{c}(s,\ell)\mathbbm{1}\{\SP\in \Env{\ell}\}N. 
\end{align*}

Therefore,

\begin{align*}
&E\left[\sum_{t=1}^{\infty}c(t)\right] \\
\leq &\sum_{\ell=1}^{\infty}\sum_{s=1}^{\max\{s_2,\lceil (a_0 \ell)^\frac{5}{2} \rceil\}}\bar{c}(s,\SP)\Prob{\SP \in \Env{\ell}}N  \\
=&\sum_{\ell=1}^{\infty}\sum_{s=1}^{\max\{s_2,\lceil (a_0 \ell)^\frac{5}{2} \rceil\}}\left[\MAXR+2\Diam d\frac{12\sigma \ell}{s^{2/5}}\right]\Prob{\SP\in \Env{\ell}}N  \\
\leq &\sum_{\ell=1}^{\infty}\sum_{s=1}^{\max\{s_2,\lceil (a_0 \ell)^\frac{5}{2} \rceil\}}\left[\MAXR+24\Diam d\sigma l\right]\Prob{\SP\in \Env{\ell}}N. 
\end{align*}

\noindent\textbf{Step 5: Final Step}

Thus,
\begin{align*}
&E\left[\sum_{t=1}^{\infty}c(t)\right]  \\
\leq&\sum_{\ell=1}^{\infty}\sum_{s=1}^{\max\{s_2,\lceil (a_0 \ell)^{5/2} \rceil\}}\left[\MAXR+24\Diam d\sigma \ell\right]\Prob{\SP\in \Env{\ell}}N  \\
\leq& \sum_{s=1}^{\max\{s_2,\lceil (a_0)^{5/2} \rceil\}}\left[\MAXR+24\Diam d\sigma \right]\Prob{\SP\in L[1]}N +\sum_{l=2}^{\infty}\sum_{s=1}^{\max\{s_2,\lceil (a_0 \ell)^{5/2} \rceil\}}\left[\MAXR+24\Diam d\sigma \ell\right]\Prob{\SP\in \Env{\ell}}N  \\
\leq& \sum_{s=1}^{\max\{s_2,\lceil (a_0)^{5/2} \rceil\}}\left[R+24\Diam d\sigma \right]N +\sum_{\ell=2}^{\infty}\sum_{s=1}^{\max\{s_2,\lceil (a_0 \ell)^{5/2} \rceil\}}\left[\MAXR+24\Diam d\sigma \ell\right]24\ARMNUM d e^{-1.8(\ell-1)^2}N \\
\leq & \left[\MAXR+24\Diam d\sigma \right]\ARMNUM \max\{s_2,\lceil (a_0)^{5/2} \rceil\}+\sum_{\ell=2}^{\infty}24\ARMNUM^2 d[\MAXR+24\Diam d\ell\sigma]\max\{s_2,\lceil (a_0 \ell)^{5/2} \rceil\} e^{-1.8(\ell-1)^2}  \\
=& O\left(\max\left\{\frac{\ARMNUM^2}{\MinProb^{5/2}},\ARMNUM^2\exp\left(\frac{2}{\MinProb}\right)\right\}\right). 
\end{align*}
\pfcomment{the above bound got tighter.  I didn't carry this forward anywhere where we report on / use the bound.}

\end{extraproof}
