\subsection{Bounding the Total Payment}

As a first step towards bound the total payment (and also regret),
we show that for sufficiently late phases,
under the event \AccEU{t}{x} for suitably small $x$,
the algorithm does not offer any payments.

\begin{lemma} \label{lem:no-incentives}
Fix an arm $i$.
Let $s \geq \exp(2/\MinProb)$, and let $\tau_s$ be the (random)
time when arm $i$ is pulled for the \Kth{s} time.
Under \AccEU{\tau_s}{\frac{p}{4\Diam d \TieDensity}},
this pull of arm $i$ is not incentivized.
\end{lemma}

Towards bounding the algorithm's total payment, we now bound the
\emph{number} of rounds in which the algorithm makes a payment by a
constant.
This bound also turns out to be useful for bounding the total regret.

\begin{lemma} \label{lem:numP}
The expected number of time steps in which
Algorithm~\ref{alg:basic-incentivizing}
makes any payment is at most $O\left( N\exp\left(\frac{2}{p}\right) \right)$.
\end{lemma}

\begin{proof}
We partition phases into early and late phases.
For each of the early phases,
we crudely bound the number of payments by \ARMNUM,
using that each arm is incentivized at most once per phase.
For later phases,
we use Lemma~\ref{lem:no-incentives},
which rules out any incentives unless large misestimates of the arm
locations occur, which is exponentially unlikely by 
Lemma~\ref{lem:round-prob}.

To make this intuition precise, we set
$\delta = \frac{\MinProb}{2 \TieDensity}$,
and $x = x' = \frac{\delta}{2 \Diam d}$
(which is independent of the phase number $s$).
The cutoff point between early and late phases is now set to
$\EvenLaterPhase = \max(2, \frac{30 \sigma^3}{x^3}, \exp(\frac{2}{\MinProb}))$.
We will verify below that
$\sqrt{0.6 n \cdot \log (\log_{1.1}(n) + 1) + \frac{n x^2}{16 \sigma^2}}
\leq \frac{n x}{2 \sigma}$
for all $n \geq \max(2, \frac{30 \sigma^3}{x^3})$.

Fix an arm $i$, and consider the \Kth{s} pull of an arm $i$,
for $s \geq \EvenLaterPhase$.
If this pull occurs before phase $s$, it was definitely not incentivized.
So we condition on the pull occurring in phase $s$,
and let $\tau_s$ be the time step when this pull occurred;
notice that this means that this must be the first pull of arm $i$ in
phase $s$.

By Lemma~\ref{lem:round-prob} (with our choice of $x = x'$),
and a union bound over all $i,j$, we bound the probability
$\Prob{\AccEU{\tau_s}{x}}
\geq 1 - 24 \ARMNUM d \cdot \exp\left(-\frac{1.8 s x^2}{16 \sigma^2} \right)$.
And by Lemma~\ref{lem:no-incentives},
under the event ${\mathcal E}_{\tau_s}(x)$,
arm $i$ is not payment-eligible.

Thus, for any $s \geq \EvenLaterPhase$,
the \Kth{s} pull of arm $i$ is incentivized  
with probability at most
$24 \ARMNUM d \cdot \exp \left(
  -\frac{1.8 s \cdot \MinProb^2}{256 \TieDensity^2 \Diam^2 d^2 \sigma^2}
\right)$.
Summing over all arms and phases $s$,
adding the at most $\ARMNUM \EvenLaterPhase$ incentivizations in
the first \EvenLaterPhase phases, 
and using that $\exp(-x) \leq 1-x/2$ for $x \leq 1$,
the expected total number of arm incentivizations is at most
\begin{align*}
\lefteqn{\ARMNUM \EvenLaterPhase
  + 24 \ARMNUM^2 d \cdot \frac{1}{1 - \exp \left(
    \frac{-1.8 \MinProb^2}{256 \TieDensity^2 \Diam^2 d^2 \sigma^2}
  \right)}}  \\
& = O\left( \max\left(\frac{\ARMNUM \TieDensity^3 \Diam^3 d^3 \sigma^3}{\MinProb^3}, N\exp(\frac{2}{p})\right)
  + \frac{\ARMNUM^2 d^3 \TieDensity^2 \sigma^2 \Diam^2}{\MinProb^2} \right)
\; = \; O\left( N\exp\left(\frac{2}{p}\right) \right).
\end{align*}

It remains to show that
$\sqrt{0.6 n \cdot \log (\log_{1.1}(n) + 1) + \frac{n x^2}{16 \sigma^2}}
\leq \frac{n x}{2 \sigma}$
for all $n \geq \max(2, \frac{30 \sigma^3}{x^3})$.
We first use that $\sqrt{\cdot}$ is sublinear to bound
\[
  \sqrt{0.6 n \cdot \log (\log_{1.1}(n) + 1) + \frac{n x^2}{16 \sigma^2}}
\; \leq \; \sqrt{0.6 n \cdot \log (\log_{1.1}(n) + 1)} + \frac{n x}{4 \sigma}
\]
(here we use the fact $\sqrt{n} \leq n$).
Next, we show that
$\sqrt{0.6 n \log (\log_{1.1}(n) + 1)} \leq \frac{n x}{4 \sigma}$;
then, adding the two terms gives the desired bound.

By squaring the claimed statement and rearranging,
it is equivalent to show that
$\frac{n}{\log(\log_{1.1}(n)+1)} \geq \frac{9.6\sigma^2}{x^2}$.
Because $n \geq 2$,
a numerical calculation and derivative test shows that
$\frac{n}{\log(\log_{1.1}(n)+1)} \geq n^{2/3}$,
and because $n \geq \frac{30 \sigma^3}{x^3}$,
we get that
$n^{2/3} \geq \frac{9.6 \sigma^2}{x^2}$, completing the proof.

\end{proof}

It would be desirable to simply identify a constant upper bound on the
payment made each time.
Unfortunately, while the agent types are drawn from a bounded support,
the noise in arm locations is not;
hence, with small probability, arm locations may be grossly
misestimated, resulting in high incentive payments.
As a result, the actual analysis of the total payment is significantly more
intricate;
the proof of Theorem~\ref{rst:budget} is given in
Appendix~\ref{sec:payment-proof}.
