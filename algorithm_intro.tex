\section{Main Algorithm and Analysis}
\label{sec:ub}
The algorithm achieving the bounds of
Theorems~\ref{rst:budget} and~\ref{rst:regret} is simple.
It mostly allows agents to exploit, but when an arm is sufficiently
unlikely to be pulled,
it incentivizes this arm with a payment high enough
to guarantee that the next agent pulls it.
This way, the algorithm ensures that each arm is pulled often enough.

More precisely, the algorithm divides time into \emph{phases}
$s = 1, 2, 3, \ldots$.
Phase $s$ starts when each arm has been pulled at least $s$ times.
We indicate the start time of phase $s$ by $t_s$. An arm $i$ is \emph{payment-eligible} at time $t$ (in phase $s$)
if both of the following hold:

\begin{itemize}
\item $i$ has been pulled at most%
\footnote{in fact: exactly, since the algorithm entered phase $s$}
$s$ times up to time $t$, i.e., $\NumPull{t}{i} \leq s$.
\item 
The conditional probability of pulling arm $i$ is less than
$1/\log(s)$ given the current estimates \ArmEV{t}{i'} of the arms'
attribute vectors.  In other words, we require
$\PullProb{t}{i} < 1/\log(s)$ where 
$\PullProb{t}{i} = \Prob[\AgV \sim \AgentDist]{\AgV \cdot \ArmEV{t}{i} > \AgV
\cdot \ArmEV{t}{i'} \mbox{ for all } i' \neq i \mid \ArmEV{t}{i'}\ \forall i'}$
is the probability that arm $i$ will be pulled
by the next (random) agent based on the current estimates. 
Our assumption of a continuous noise distribution and one free pull
ensure that ties in $\AgV \cdot \ArmEV{t}{i}$ between arms occur with
probability $0$.
\dkcomment{Should we say Gaussian here? Should we write everything in
  the general model?}
\end{itemize}


When multiple arms are payment-eligible, the algorithm picks one arbitarily to incentivize.
When the algorithm decides to incentivize an arm $i$,
it offers ``whatever it takes,'' i.e., offers a payment of
$\Pay{t}{i} = \max_{\AgV,i'} \AgV \cdot (\ArmEV{t}{i'} - \ArmEV{t}{i})$.
The maximum for \AgV is taken over the support of \AgentDist;
recall that we assumed this support to be compact.
The payment \Pay{t}{i} may appear unnecessarily high.
Indeed, it suffices to
incentivize only a $1/\log(s)$ fraction of the agents,
and our bounds also hold for an alternate version of our algorithm that 
offers payment
$\Pay{t}{i} = \sup \Set{c \geq 0}{%
\Prob[\AgV \sim \AgentDist]{c + \AgV \cdot \ArmEV{t}{i} \geq \max_{i'\ne i} \AgV \cdot \ArmEV{t}{i'}} \leq 1/\log(s)}$.
(This definition ensures that \Pay{t}{i} is well-defined
and incentivizes at least a $1/\log(s)$ measure of agents,
even if $f$ has discrete points.)
However, we focus on the higher payments for simplicity of presentation.

Notice that \PullProb{t}{i} depends on the estimates for \emph{all}
arms; thus, by pulling another arm $i'$, an arm $i$ may become
payment-eligible, or cease to be so.
Algorithm~\ref{alg:basic-incentivizing} gives the full details.


\begin{algorithm}
\caption{Algorithm: Incentivizing Exploration \label{alg:basic-incentivizing}}
\begin{algorithmic}
\STATE Set the current phase number $s = 1$.
\COMMENT{Each arm is pulled once initially ``for free.''}
\FOR{time steps $t = 1, 2, 3, \ldots$}
\IF{$\NumPull{t}{i} \geq s+1$ for all arms $i$}
\STATE Increment the phase $s = s + 1$.
\ENDIF
\IF{there is a payment-eligible arm $i$}
\STATE Let $i$ be an arbitary payment-eligible arm.
\STATE Offer payment
$\Pay{t}{i} = \max_{\AgV,i'} \AgV \cdot (\ArmEV{t}{i'} - \ArmEV{t}{i})$
for pulling arm $i$
(and payment 0 for all other arms).
\ELSE
\STATE Let agent $t$ play myopically, i.e., offer payments 0 for all arms.
\ENDIF
\ENDFOR
\end{algorithmic}
\end{algorithm}

The high-level idea in the proofs of our main results, Theorems~\ref{rst:budget} and~\ref{rst:regret},
is the following.
Because the algorithm ensures that each arm is pulled
``frequently enough,''
the estimates \ArmEV{t}{i} become gradually more accurate in the
phase number $s$.
Thus, the fraction of agents who misidentify their best arm decreases.
Because each arm has enough agents that would prefer it based
  on its true attribute vector, 
once the arms' attribute vectors are learned well enough,
the algorithm will not need to incentivize any more,
resulting in a payment bound independent of $T$.
Similarly, instantaneous regret will decrease, and mostly accrue
due to ``problematic'' agents who are nearly tied in their preferences
between their top two arms.
The detailed analysis following this outline is complicated by
dependence between the agents' arm pulls and the
estimates which in turn are based on past arm pulls.
We begin with several technical lemmas that are used for both the
payment and regret bounds.

To formally reason about the event that the estimates of arms'
attributes vectors are accurate enough --- or fail to be so ---
we define the events
$\AccE{t}{i}{j}{x} := [|\ArmE{t}{i}{j} - \Arm{i}{j}| \leq x]$
that attribute $j$ of arm $i$ at time $t$ is estimated to
within accuracy $x$ or better.
Then, 
$\AccEU{t}{x} = \bigcap_{i,j} \AccE{t}{i}{j}{x}$
is the event that at time $t$, all arm attribute
estimates are accurate to within $x$ simultaneously.
We will show that for suitable choices of $t, x$,
the events \AccE{t}{i}{j}{x}
(and hence, by a union bound, \AccEU{t}{x})
have high probability,
and that when they do, myopic agents do not make large mistakes.

%We now prove our main results, Theorems~\ref{rst:budget} and~\ref{rst:regret}.


