\section{Proof Sketch of Theorem~\ref{rst:discrete}}
\label{sec:discussion-proof1}

\begin{rtheorem}{Theorem}{\ref{rst:discrete}}
Under Assumption~\ref{a:discrete}, the Discrete-Preference Algorithm has expected payment budget bounded above by 
\begin{align*}
O \left(\frac{\ARMNUM^2 d \cdot (\MAXR + \Diam d \sigma)\sigma^{5/2}L^{5/2}\Diam^{5/2}d^{5/2}}{\MinProb^{5/2}} \right),
\end{align*}
and expected regret bounded above by 
\begin{align}
O \left(\MAXR\cdot \frac{\ARMNUM \TieDensity^4 \Diam^4 d^4 \sigma^4}{\MinProb^4}
  + \frac{\MAXR \ARMNUM^2 d^3 \TieDensity^2 \Diam^2 \sigma^2}{\MinProb^2}
  \right).  \nonumber
\end{align}
\end{rtheorem}

\begin{proof}
The proof of the expected payment bound is the same as the proof of Theorem~\ref{rst:budget}, except that we now define $h(\ell) := \max \left( \frac{2}{\MinProb},
\left( \frac{48 \sigma \ell \TieDensity \Diam d}{\MinProb} \right)^{5/2}
\right)$.

To prove the bound on the expected regret, one can first prove tightened versions of Lemmas~\ref{lem:no-incentives} and \ref{lem:numP},
which replace the $\exp(2/\MinProb)$ term with simply $2/\MinProb$.
In return, the length of phase $s$ is now bounded only be $\ARMNUM s$ instead of $\ARMNUM \log(s)$.
Substituting these changes into the proof of Theorem~\ref{rst:regret}, we obtain the claimed bounds.
\end{proof}

\section{Proof Sketch of Theorem~\ref{rst:known-p}}
\label{sec:discussion-proof2}


\begin{rtheorem}{Theorem}{\ref{rst:known-p}}
Under Assumption~\ref{a:known-p}, the Known-\MinProb Algorithm has an expected payment budget bounded above by 
\begin{align*}
O \left(\frac{\ARMNUM^2 d \cdot (\MAXR + \Diam d \sigma)\sigma^{5/2}L^{5/2}\Diam^{5/2}d^{5/2}}{\MinProb^{5/2}} \right),
\end{align*}
and an expected regret bounded above by
\bcedit{
\begin{align*}
O \left(\frac{\MAXR \ARMNUM \TieDensity^3 \Diam^3 d^3 \sigma^3}{\MinProb^3}
  + \frac{\MAXR \ARMNUM^2 d^3 \TieDensity^2 \Diam^2 \sigma^2}{\MinProb^2}
  + \frac{\ARMNUM\Diam^2 d^2\sigma^2 \TieDensity(\log(T))^3}{\MinProb}\right).
\end{align*}
}
\end{rtheorem}

\begin{proof}
  The proof of the expected payment bound follows Lemma~\ref{rst:budget} except for defining
  $h(\ell) := \left( \frac{48 \sigma \ell \TieDensity \Diam d}{\MinProb} \right)^{5/2}$
(without including the $\exp(2/\MinProb)$ term).

The proof of the expected regret bound first establishes tightened versions of Lemmas~\ref{lem:no-incentives} and \ref{lem:numP},
proving the following upper bound on the number of time steps in which a payment is made:
\begin{align}
O\left(\frac{\ARMNUM \TieDensity^3 \Diam^3 d^3 \sigma^3}{\MinProb^3}
  + \frac{\ARMNUM^2 d}{1 - \exp \left(
    \frac{-1.8 \cdot \MinProb^2}{256 \TieDensity^2 \Diam^2 d^2 \sigma^2}
  \right)} \right). \nonumber 
\end{align}
The proof of this result follows that of Lemma~\ref{lem:numP},
but the less aggressive incentivization allows us to define $\EvenLaterPhase = \max(2, \frac{30 \sigma^3}{x^3})$ since $\frac{1}{\log(s+s_0)} \leq \frac{\MinProb}{2}$ is true for all $s$.

Using this tighter bound on the number of incentiziations, and the fact that phases now last at most $\ARMNUM \log(s+s_0)$ steps in expectation, we can bound the regret in Equation~\ref{equ:small_regret_bound} by
%$71.11 \ARMNUM\Diam^2 d^2\sigma^2 \TieDensity\log(T)(\log(T)+1)\log(T+s_0)$
$O\left(\frac{\ARMNUM\Diam^2 d^2\sigma^2 \TieDensity(\log(T))^3}{\MinProb}\right)$.
\end{proof}
