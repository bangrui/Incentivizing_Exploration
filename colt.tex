\documentclass[final,12pt]{colt2018}
% \usepackage[anon]{jmlr2e}


% Any additional packages needed should be included after jmlr2e.
% Note that jmlr2e.sty includes epsfig, amssymb, natbib and graphicx,
% and defines many common macros, such as 'proof' and 'example'.
%
% It also sets the bibliographystyle to plainnat; for more information on
% natbib citation styles, see the natbib documentation, a copy of which
% is archived at http://www.jmlr.org/format/natbib.pdf


\usepackage[]{algorithm}
\usepackage[noend]{algorithmic}

%\usepackage[utf8]{inputenc} % allow utf-8 input
%\usepackage[T1]{fontenc}    % use 8-bit T1 fonts
%\usepackage{microtype}      % microtypography
%\usepackage{xr} 

\usepackage{url}            % simple URL typesetting
\usepackage{amsfonts}       % blackboard math symbols
\usepackage{amsmath}
\usepackage{nicefrac}       % compact symbols for 1/2, etc.
%\usepackage{amsthm}
%\usepackage{mathtools}

%\usepackage{comment}
%\usepackage{subcaption}
%\usepackage{xcolor}
%\usepackage{float}
\usepackage{bm}
\usepackage{bbm}
\usepackage{ifthen}
\usepackage{xspace}

%\usepackage{color-edits}
% home-made package by David that provides the following macros:
% 1. \pfedit{}: prints argument in orange
% 2. \pfcomment{}: prints argument in orange in [Peter: #1]
% 3. \pfmargincomment{}: prints argument in orange in the margin as
% [Peter: #1]
% 4. \pfdelete{}: instead of text, marks a note in the margin in orange that says
% "Peter deleted here"
% Package has option [suppress], which gets rid of all color and all comments
% and [showdeletions], which shows deleted text in color/strikeout.
%\addauthor[Peter]{pf}{orange}
%\addauthor[Bangrui]{bc}{green}
%\addauthor[David]{dk}{red}

%\newcommand{\bccomment}[1]{{\color{blue}BC: #1}}
%\newcommand{\pfcomment}[1]{{\color{blue}PF: #1}}

\providecommand{\SET}[1]{\ensuremath{\{ #1 \}}\xspace}
\providecommand{\Abs}[1]{\ensuremath{| #1 |}\xspace}
\providecommand{\Set}[2]{\ensuremath{\SET{#1 \mid #2}}\xspace}

\providecommand{\PROB}{\ensuremath{{\rm Prob}}\xspace}
\providecommand{\Prob}[2][]{\ensuremath{%
\ifthenelse{\equal{#1}{}}{\PROB\left[#2\right]}{\PROB_{#1}\left[#2\right]}}\xspace}
\providecommand{\ProbC}[3][]{\Prob[#1]{#2\;|\;#3}}
\providecommand{\Expect}[2][]{\ensuremath{%
\ifthenelse{\equal{#1}{}}{\mathbb{E}}{\mathbb{E}_{#1}}%
\left[#2\right]}\xspace}
\providecommand{\ExpectC}[3][]{\Expect[#1]{#2\;|\;#3}}

\newcommand{\e}{\mathrm{e}}
\newcommand{\dd}{\mathrm{d}}
\newcommand{\quarter}{\ensuremath{\frac{1}{4}}\xspace}

\providecommand{\Kth}[1]{\ensuremath{{#1}^{\rm th}}}

\def\QED{{\phantom{x}} \hfill \ensuremath{\rule{1.3ex}{1.3ex}}}

\newcommand{\extraproof}[1]{\rm \trivlist \item[\hskip \labelsep{\bf Proof of #1. }]}
\def\endextraproof{\QED \endtrivlist}

\def\emptyproof{\rm \trivlist \item[\hskip \labelsep{\bf Proof. }]}
\def\endemptyproof{\endtrivlist}

\newcommand{\emptyextraproof}[1]{\rm \trivlist \item[\hskip \labelsep{\bf Proof of #1. }]}
\def\endemptyextraproof{\endtrivlist}

\newenvironment{rtheorem}[3][]{

\bigskip

\noindent \ifthenelse{\equal{#1}{}}{\bf #2 #3}{\bf #2 #3 (#1)}
\begin{it}
}{\end{it}}

\newenvironment{rlemma}[3][]{

\bigskip

\noindent \ifthenelse{\equal{#1}{}}{\bf #2 #3}{\bf #2 #3 (#1)}
\begin{it}
}{\end{it}}

\newcommand{\argmax}{\mathop{\mathrm{argmax}}}
\newcommand\floor[1]{\lfloor#1\rfloor}
\newcommand\ceil[1]{\lceil#1\rceil}
\newcommand{\fracpartial}[2]{\frac{\partial #1}{\partial  #2}}
\newcommand{\R}{\mathbb{R}} % real numbers

\newtheorem{assumption}{Assumption}

% Definitions of handy macros can go here

\newcommand{\vc}[1]{\bm{#1}} % vectors

\newcommand{\Normal}[2]{\ensuremath{{\mathcal N}(#1,#2)}\xspace}
% normal distribution

\newcommand{\dataset}{{\cal D}} % used anywhere?

%%%%%%%%%%%%%%%%%%%%%%%%%%%%%%%%%%%%
% Agreed upon symbols without macros
%%%%%%%%%%%%%%%%%%%%%%%%%%%%%%%%%%%%

% dimension for attribute vector: d
% stopping time: \tau
% phase of the algorithm: s


% To define and add in appropriate spots

% \Delta for utility difference
% \bar{\zeta} for error in utility difference
% q_{\delta} for mass of users


%%%%%%%%%%%%%%%%%%%%%%%%%%%%%%%%%%%%%
% Stuff related to Arms and Arm Pulls
%%%%%%%%%%%%%%%%%%%%%%%%%%%%%%%%%%%%%

\newcommand{\ARMNUM}{\ensuremath{N}\xspace} % number of arms

\newcommand{\ArmV}[1]{\ensuremath{\vc{\mu}_{#1}}\xspace}
% vector for location of an arm

\newcommand{\Arm}[2]{\ensuremath{\mu_{#1}^{(#2)}}\xspace}
% individual entries of the ith arm's location vector
% Argument 1: arm index
% Argument 2: coordinate

\newcommand{\NoiseV}[1][]{\ensuremath{\vc{\zeta}_{#1}}\xspace}
% vector of noise added to arm

\newcommand{\Noise}[2][]{\ensuremath{%
\ifthenelse{\equal{#1}{}}{\zeta_{#2}}{\zeta_{#1,#2}}\xspace}}
% individual entries of noise vector

\newcommand{\ObsV}[1]{\ensuremath{\vc{y}_{#1}}\xspace}
% observation of an arm: location plus noise

\newcommand{\ArmEV}[2]{\ensuremath{\vc{\hat{\mu}}_{#1,#2}}\xspace} 
% empirical estimator of the arm location based on samples
% Argument 1: time step
% Argument 2: arm

\newcommand{\ArmE}[3]{\ensuremath{\hat{\mu}_{#1,#2}^{(#3)}}\xspace}
% coordinates of empirical estimater of the arm location
% Argument 1: time step
% Argument 2: arm
% Argument 3: coordinate/attribute

\newcommand{\ErrV}[2]{\ensuremath{\vc{\epsilon}_{#1,#2}}\xspace} 
% error in the empirical estimator of the arm location
% Argument 1: time step
% Argument 2: arm

\newcommand{\Err}[3]{\ensuremath{\epsilon_{#1,#2}^{(#3)}}\xspace}
% coordinates of error in the empirical estimater of the arm location
% Argument 1: time step
% Argument 2: arm
% Argument 3: coordinate/attribute

\newcommand{\PullProb}[2]{\ensuremath{\phi_{#1,#2}}\xspace}
% probability that a random myopic agent will pull the arm
% Argument 1: time step
% Argument 2: arm
% \phi is a placeholder for now.


%%%%%%%%%%%%%%%%%%%%%%%%%
% Stuff related to agents
%%%%%%%%%%%%%%%%%%%%%%%%%

\newcommand{\AgV}{\ensuremath{\vc{\theta}}\xspace}
% vector for location of an agent, without subscript

\newcommand{\Ag}[1]{\ensuremath{\theta_{#1}}\xspace}
% individual entries of an agent location vector without time step

\newcommand{\AgentV}[1]{\ensuremath{\AgV_{#1}}\xspace}
% vector for location of an agent, subscripted by time of arrival

\newcommand{\Agent}[2]{\ensuremath{\theta_{#1,#2}}\xspace}
% individual entries of the agent location vector

\newcommand{\Best}[1]{\ensuremath{B_{#1}}\xspace}
% Best arm for a given agent

\newcommand{\Second}[1]{\ensuremath{B'_{#1}}\xspace}
% Second-best arm for a given agent

\newcommand{\FirstTwo}[2]{\ensuremath{\Omega_{#1,#2}}\xspace}
% Set of agents who have i as their first choice and i' as their second



%%%%%%%%%%%%%%%%%%%%%%%%%%%%%%%%%%%%%
% Stuff related to agent distribution
%%%%%%%%%%%%%%%%%%%%%%%%%%%%%%%%%%%%%

\newcommand{\AgentDist}{\ensuremath{f}\xspace}
% distribution of locations for agents

\newcommand{\Diam}{\ensuremath{D}\xspace}
% "diameter" of support of agent distribution, [0,D]^d

\newcommand{\MinProb}{\ensuremath{p}\xspace}
% minimum probability of support for any arm

\newcommand{\TieDensity}{\ensuremath{L}\xspace}
% upper bound on the derivative of the density of near-ties

\newcommand{\AlmostTied}[1]{\ensuremath{q(#1)}\xspace}
% probability mass of almost tied agents across all arms

%\newcommand{\ClearPref}[1]{\ensuremath{p(#1)}\xspace}
% probability mass of agents with a clear preference for their first choice,
% minimized over all arms.
% not used anywhere.


%%%%%%%%%%%%%%%%%%%%%%%%%%%
% Stuff related to policies
%%%%%%%%%%%%%%%%%%%%%%%%%%%

\newcommand{\POLICY}{\ensuremath{\mathcal A}\xspace} % algorithm/policy
\newcommand{\Pay}[2]{\ensuremath{c_{#1,#2}}\xspace}
% payment offered for arms:
% Argument 1: time step
% Argument 2: arm

\newcommand{\PayA}[1]{\ensuremath{c_{#1}}\xspace}
% payment actually made in step t, i.e., c_{t,\Pull{t}}

\newcommand{\TotalPay}[1]{\ensuremath{C_{#1}}\xspace}
% cumulative payment made up to and including step t.

\newcommand{\Pull}[1]{\ensuremath{i_{#1}}\xspace}
% arm that was actually pulled at time step t

\newcommand{\NumPull}[2]{\ensuremath{m_{#1,#2}}\xspace}
% number of pulls of arm i up to time t:
% Argument 1: time step
% Argument 2: arm

\newcommand{\Regret}[1]{\ensuremath{r_{#1}}\xspace}
% regret incurred in time step t

\newcommand{\TotalRegret}[1]{\ensuremath{R_{#1}}\xspace}
% total regret incurred up to and including time step t

\newcommand{\MAXR}{\ensuremath{R}\xspace}
% upper bound on the maximum regret in any one round

\newcommand{\History}[1]{H_{#1}}
% History of pulls, payments, and observations

%%%%%%%%%%%%%%%%%%%%%%%%%%%%%%%%%%
% Stuff related to specific bounds
%%%%%%%%%%%%%%%%%%%%%%%%%%%%%%%%%%

\newcommand{\LatePhase}{\ensuremath{s_0}\xspace}
% lower bound after which concentration lemma holds

\newcommand{\EvenLaterPhase}{\ensuremath{s_1}\xspace}
% lower bound after which payments become unlikely



%%%%%%%%%%%%%%%%%%%%%%%%%%%%%%%%
% Stuff related to payment proof
%%%%%%%%%%%%%%%%%%%%%%%%%%%%%%%%

\newcommand{\SP}{\ensuremath{\omega}\xspace}
% sample path

\newcommand{\Env}[1]{\ensuremath{{\mathcal L}_{#1}}\xspace}
% envelope
% Argument: level of envelope


\title{Incentivizing Exploration by Heterogeneous Users}
\usepackage{times}

\coltauthor{\Name{Bangrui Chen} \Email{bc496@cornell.edu} \\
\Name{Peter I.\ Frazier} \Email{pf98@cornell.edu} \\
\addr Operations Research and Information Engineering\\
Cornell University\\
New York, NY 14850, USA
\AND
\Name{David Kempe} \Email{david.m.kempe@gmail.com}  \\
\addr Department of Computer Science\\
University of Southern California\\
Los Angeles, CA 90089, USA}

\begin{document}

\maketitle

\begin{abstract}

We consider the problem of incentivizing exploration with heterogeneous agents.
In this problem, bandit arms provide vector-valued outcomes equal to an unknown
arm-specific attribute vector,
perturbed by independent noise.
Agents arrive sequentially and choose arms to pull based on their own
private and heterogeneous linear utility functions over attributes
and the estimates of the arms' attribute vectors derived from
observations of other agents' past pulls.
Agents are myopic and selfish and thus would choose the arm with
maximum estimated utility.
A principal, knowing only the distribution from which agents'
preferences are drawn, but not the specific draws,
can offer incentive payments for pulling specific arms
in order to encourage agents to explore underplayed arms.
The principal seeks to minimize the total expected cumulative regret
incurred by agents relative to their best arms,
while also making a small expected cumulative payment.

We propose an algorithm that incentivizes arms played infrequently in the
past whose probability of being played in the next round would be small
without incentives.
Under the assumption that each arm is preferred by at
least a fraction $\MinProb > 0$ of agents,
we show that this algorithm achieves expected
cumulative regret of $O (\ARMNUM \e^{2/\MinProb} + \ARMNUM \log^3(T))$,
using expected cumulative payments of $O(\ARMNUM^2 \e^{2/\MinProb})$, where $\ARMNUM$ is the total number of arms.
If \MinProb is known or the distribution over agent
preferences is discrete,
the exponential term $\e^{2/\MinProb}$ can be replaced with suitable
polynomials in \ARMNUM and $1/\MinProb$.
For discrete preferences, the regret dependence on $T$ can be
eliminated entirely, giving constant (depending only polynomially on
\ARMNUM and $1/\MinProb$) expected regret and payments.
This constant regret stands in contrast to the $\Theta(\log(T))$ dependence of
regret in standard multi-armed bandit problems.
It arises because even unobserved heterogeneity in agent preferences
allows exploitation of arms to also explore arms fully;
succinctly, heterogeneity provides free exploration.

\end{abstract}

\begin{keywords}
Incentivizing Exploration, Multi-Armed Bandits, Social Learning
\end{keywords}

     

\section{Introduction}

Many websites and apps are designed to facilitate joint discovery,
sharing, and recommendations of content.
Such sites include news, photo, and video sharing sites,
sites to review restaurants, hotels, or travel experiences,
online stores at which users write reviews (such as Amazon),
and citizen science projects
(such as eBird \citep{sullivan2009ebird,xue-ebird} or Galaxy Zoo \citep{lintott-galaxy-zoo}).
By learning from the experiences of other users, individuals can
improve their own experience \citep{schmit2017human}.

Viewed more abstractly, users jointly explore a space of
many options (products, news stories, photos, birdwatching sites,
patches of sky to train their telescopes on, $\ldots$),
with the implicit goal of identifying the ``best'' ones.
Therefore, such scenarios can be fruitfully modeled in a bandit
learning framework.
However, contrary to standard bandit settings, the utilities of the
decision makers (the users) are not aligned with the overall utility.
Societally (i.e., in a suitable aggregate over all users),
it would be desirable to engage in considerable exploration of
different options, so as to provide higher rewards for a large number
of future users.
However, individual users only interact with the site a limited number
of times, and therefore have little incentive for exploration.
A particularly clean model is obtained when each user interacts with the
site only once, and hence has no intrinsic utility for exploration. 
This model has been the subject of prior work, and forms the basis of
the present submission.

To effect an outcome close to societally optimal,
it is necessary to provide exploration incentives to the individual users.
This was noted in two recent lines of work:
\citet{kremer2014implementing}
and \citet{mansour2015bayesian,mansour2016bayesian}
assume that the site (also called the \emph{principal}) has an
informational advantage in being the only one to observe the results
of past arm pulls.
(Such an assumption applies, for instance, to route recommendations in
driving.)
The principal can exploit her%
\footnote{We use male pronouns to refer to users and female pronouns
  to refer to the principal.}
advantage and make recommendations to the individual agents that 
are in their best interest to follow.
\citet{frazier2014incentivizing} and 
\citet{han2015incentivizing} instead assume that the results of all
past arm pulls are publicly observable
(such as reviews on an online retail site).
They instead suppose the principal can offer payments to users as
a reward for pulling particular arms.
We follow the model of \citet{frazier2014incentivizing} and
\citet{han2015incentivizing} 
and consider a multi-armed bandit model in which the principal can
offer payments to the users for pulling particular arms.

Past work has assumed that users are homogeneous, i.e., the expected
reward a user derives from an arm is the same for all users.%
\footnote{\citet{han2015incentivizing} assume that users are
heterogeneous in their tradeoff between utility derived from arm pulls
and utility derived from the principal's payment.}
In reality, users have different preferences, e.g., gastronomic,
political, aesthetic, practical, etc.
Indeed, the websites and mobile apps most widely used for joint discovery,
sharing, and recommendation of content tend to concern products and items 
with large amounts of heterogeneity in preferences (movies, restaurants,
videos, travel experiences), and not items which have a universally
agreed-on best order.
This is perhaps because regimes with heterogeneous preferences are the
ones where discovery of the best items is the most difficult for people, 
and thus where online
platforms tend to provide the greatest value.  Thus, we see an appropriate
accounting for heterogeneity as critical to an understanding of incentivizing
exploration in online communities.

Heterogeneity presents both a challenge and an opportunity.
On the one hand, unobserved heterogeneity hides critical
information about an agent's preferences from the principal.
On the other hand, heterogeneity also presents her an opportunity,
through the possibility of ``free exploration.''
Even when left unincentivized, agents will
play a variety of arms, revealing information about these arms' attributes.
This stands in sharp contrast to the case of homogeneous preferences,
where unincentivized agents will herd onto a single apparently best arm;
thus, effecting essentially any exploration at all requires incentives.

In order to take advantage of unobserved heterogeneity,
the principal has to give up some control, allowing the agent to
reveal his preferences through action, rather than obscuring them with
incentives.
However, she cannot give up control completely,
and must use incentives to force agents to explore against their
preferences, for the greater good. 

With these challenges and opportunities in mind, our goal is to understand the
impact of user heterogeneity on the principal's ability to achieve
high social utility with low incentive payments,
and on the best approaches for doing so.
We wish to understand whether incentivizing exploration with
heterogeneous preferences  is ``harder'' or ``easier'' than with
homogeneous ones,
and to understand how exploration strategies that do well in the
heterogeneous preference setting differ from those in the homogeneous one.

Toward that end, we model our setting as follows.
(We describe our model at a high level here,
with formal definitions given in Section~\ref{sec:prob}.)
Arms and users (or \emph{agents}) are characterized by payoff-relevant
\emph{attribute} (or \emph{feature}) vectors.
Arms' attributes are a priori unknown, and
  agents' attributes are drawn from a known distribution.
An agent's reward from pulling an arm is the inner product of his
vector with the arm's vector (plus noise).
When an arm is pulled, a noisy version of its attribute vector is
observed by everyone.
Agents are myopic and will pull the arm whose expected attribute
vector (based on past noisy observations) maximizes their reward.
The principal can incentivize agents to pull particular arms by
offering rewards for the specific arm.
The principal's goal is to keep the cumulative regret across all
agents small, while incurring only small total payments.
Our main theorem can be stated informally as follows:

\begin{theorem} \label{thm:main-intro}
Assume that for each arm, at least a constant fraction of the
population likes this arm best.
If furthermore, the density of ties in agent preferences between arms
vanishes (in a sense made precise in Section~\ref{subsec:discrete}),
there is a policy that achieves constant
expected regret $C$ and constant%
\footnote{The constants $C,C'$ depend on the number of arms and the
  smallest fraction preferring any one arm.} expected payment $C'$.
If near-ties have non-zero density,
then the regret is bounded by $O(C + C'' \log^3(T))$,
while the expected payment is still bounded by $C'$;
here, $C''$ depends on the ``tie density.''
\end{theorem}

The policy achieving the result of Theorem~\ref{thm:main-intro} is
quite simple: it mostly lets agents exploit arms, but incentivizes
them to explore when arms appear unlikely to be pulled without incentives.
It is presented in detail in Section~\ref{sec:ub}.


\section{Preliminaries}
\label{sec:prob}

We consider a multi-armed bandit setting with \ARMNUM arms.
Arm payoffs are determined by $d$ \emph{attributes} or \emph{features};
hence, arms can be identified with vectors $\ArmV{i} \in \R^d$.
The \ArmV{i} are (adversarially) fixed, and unknown to the agents
or the principal.
Whenever arm $i$ is pulled, its current utility-relevant features are
determined as $\ArmV{i} + \NoiseV$, where \NoiseV is a mean-zero independent Gaussian%
\footnote{For simplicity of notation, we assume that the variance
  $\sigma^2$ is uniform across time steps.
  This is not material for the analysis.
\dkedit{In fact, the analysis extends straightforwardly to any
  mean-zero sub-Gaussian noise.} \dkcomment{Is this true?}}
noise vector $\NoiseV \sim \Normal{\vc{0}}{\sigma^2 I_{d}}$.
Here, $I_d$ denotes the $d \times d$ identity matrix.

At each time $t$, a new user (or \emph{agent}) arrives,
whose feature vector (which we also call his \emph{type})
$\AgentV{t} \in \R^d$ is drawn from a known distribution \AgentDist.
Depending on context, we will identify agents with their arrival time
$t$ or their type \AgentV{t}.
When agent $t$ pulls arm $\Pull{t} := i$,
he and all future agents observe a vector
$\ObsV{t} = \ArmV{i} + \NoiseV[t]$ for arm $i$,
and his reward is $\AgentV{t} \cdot \ObsV{t}$,
i.e., agents have linear preferences.%
\footnote{\dkedit{Some of our results generalize to other forms of
    agent rewards.}}

For each time $t$ and arm $i$, let \NumPull{t}{i} be the number of
times that arm $i$ has been pulled (strictly) before time $t$.
An agent at time $t$ estimates arm $i$'s attribute vector as the
average of vectors observed during past pulls of the arm:
$\ArmEV{t}{i} = \frac{1}{\NumPull{t}{i}+1} \cdot
(\ArmEV{0}{i} + \sum_{t'<t: \Pull{t'} = i} \ObsV{t'})$;
here, \ArmEV{i}{0} is a single draw $\ArmV{i} + \NoiseV$ for arm $i$.
(In other words, we assume that each arm is pulled once for free at time 0.)

Since each user only pulls an arm once, users are \emph{myopic}:
in the absence of incentives, user $t$ will pull an arm from
$\argmax_i \AgentV{t} \cdot \ArmEV{t}{i}$.
To incentivize users to explore more, the principal can offer
\emph{payments} $\Pay{t}{i}$ to user $t$ for pulling arm $i$.
Then, user $t$ will pull an\footnote{We assume that ties are broken in
  favor of an arm with largest payment \Pay{t}{i}.}
% ; this assumption is
%   only for notational convenience, and can of course be avoided by
%   raising payments infinitesimally.
%   \pfcomment{In the finite preference setting this is clear, but it's a little bit delicate when there are continuum preferences, since regardless of the payment we offer there may always be support within our preference distribution for a $\theta$ that causes a tie.  In this continuum setting it should instead be possible to perturb infinitessimally so that we avoid ties with probability 1.}  
% }
arm $i$ maximizing $\argmax_i (\Pay{t}{i} + \AgentV{t} \cdot \ArmEV{t}{i})$.
The principal cannot observe \AgentV{t},
and only knows the distribution \AgentDist from which it is drawn.
Her goal is to achieve a good tradeoff between the cumulative
\emph{regret} experienced by all users up to time $T$,
and the total \emph{payment} she makes to the users.

We define the regret at time $t$ as
$\Regret{t} = (\max_{i} \AgentV{t} \cdot \ArmV{i}) - \AgentV{t} \cdot \ArmV{\Pull{t}}$,
and the cumulative regret up to time $T$ as
$\TotalRegret{T} = \sum_{t=1}^{T} \Regret{t}$.
Similarly, $\PayA{t} = \Pay{t}{\Pull{t}}$ is the actual incentive
payment at time $t$,
and the cumulative payment up to time $T$ is
$\TotalPay{T} = \sum_{t=1}^{T} \PayA{t}$.
More formally, the principal's goal is to find a policy
\POLICY for offering payments under which both the cumulative expected
regret
$\Expect{\TotalRegret{T}}$ the cumulative expected payment
$\Expect{\TotalPay{T}}$ are small.
\dkcomment{Deleted the \POLICY subscript for the expectation, as the
  expectation is also over the entire random history.}

To support the formulation of our results and the analysis,
we define the following additional notation.
We let
$\Best{\AgV} \in \argmax_i \AgV \cdot \ArmV{i}$
and
$ \Second{\AgV} \in \argmax_{i \neq \Best{\AgV}} \AgV \cdot \ArmV{i}$
denote the (indices of) the best and second-best arms for an agent
with attribute vector \AgV,
\dkedit{breaking ties arbitrarily (but consistently)}.

% \dkdelete{This behavior may be recovered if agents are Bayesian and
%   share a common non-informative prior distribution that is constant
%   over $\mathbb{R}^m$ and know $\sigma^2$.  In this case, the
%   posterior distribution on $u_{i}$ at time $t$ is multivariate normal
%   with mean $u_{i,t}$, and the expected value of $\theta_t \cdot u_i$
%   under this posterior conditioned on $\theta_t$ is $\theta_t \cdot
%   u_{i,t}$ (see Equation 2.13 in Section 2.5, \cite{Ge04}).
%   Alternatively, one may simply take our assumption that agents use
%   the average as their estimate of an attribute value directly without
%   such a Bayesian justification.}
% \dkcomment{I wonder what's the best way to discuss this.}

\subsection{Properties of \AgentDist}
Our algorithms rely on a few assumptions about the agent distribution
\AgentDist.

\begin{assumption}[Compact Support] \label{A2}
\AgentDist has a compact support set contained in $[0,\Diam]^d$.
\end{assumption}

Let $\MinProb = \min_{i} \Prob[\AgentDist]{\Set{\AgV}{\Best{\AgV} = i}}$
denote the minimum (over all arms) fraction of users that prefer any
particular arm.

\begin{assumption}[Every arm is someone's best] \label{A3}
Each arm $i$ has a strictly positive proportion of users for whom $i$
is the best arm; that is, $\MinProb > 0$.
\end{assumption}

%Let $\FirstTwo{i}{i'} = \Set{\AgV}{\Best{\AgV} = i, \Second{\AgV} = i'}$
%be the set of agent attribute vectors whose best arm is $i$ and
%second-best arm is $i'$.
% Let $F_{i,i'}$ be the cumulative density function
% (or cumulative mass function if $\AgentDist(\cdot)$ is a discrete distribution)
% $F_{i,i'}(z) = \Prob[\AgV \sim \AgentDist]{\AgV \in \FirstTwo{i}{i'}
%   \mbox{ and } (\ArmV{i}-\ArmV{i'}) \cdot \AgV \leq z}$
% of $(\ArmV{i}-\ArmV{i'}) \cdot \AgV$,
% on $\AgV \in \Omega_{i,i^{'}}$.
% In words, $F_{i,i'}$ is the distribution of the \emph{strength} of the
% preference of a random agent for arm $i$ over arm $i'$

% $\Best{\AgV} \in \argmax_i \AgV \cdot \ArmV{i}$
% and
% $ \Second{\AgV} \in \argmax_{i \neq \Best{\AgV}} \AgV \cdot \ArmV{i}$

\dkedit{Let \AlmostTied{z} be the cumulative density function
(or cumulative mass function if $\AgentDist(\cdot)$ is a discrete distribution)
$\AlmostTied{z} = \Prob[\AgV \sim \AgentDist]{%
  (\ArmV{\Best{\AgV}}-\ArmV{\Second{\AgV}}) \cdot \AgV \leq z}$.
In words, \AlmostTied{z} is the CDF of the \emph{strength} of the
preference of a random agent for his best arm over his second-best arm.}

\begin{assumption}[Not too many near-ties] \label{A1}
Near-ties have vanishing probability; 
that is, there exists a constant \TieDensity such that for all $z \in \R^+$,
%for all pairs $i,i'$ of arms and 
$\AlmostTied{z} \leq \TieDensity \cdot z$.
\end{assumption}

\dkedit{\paragraph{Role of parameters:}
Our problem setting is characterized by a fairly large number of
parameters.
Of these, we consider \ARMNUM, $\MinProb \leq 1/\ARMNUM$ and $T$ to be
the key parameters, while $\sigma, \Diam, d, \TieDensity$ should be
considered constant.
We will keep track of these constants throughout most of our proofs,
but report final results in terms of only the three key parameters
(except where we illustrate a specific point).}


\dkcomment{Stuff to discussion in conclusions:
\begin{enumerate}
\item Arms that are noone's best.
\item Other utilities besides inner products.
\item Other noise distributions.
\item Better bounds.
\end{enumerate}
}

\section{Overview of Results and Discussion}

Our main algorithm is presented as
Algorithm~\ref{alg:basic-incentivizing}
in Section~\ref{sec:ub}.
Our main result is the following pair of theorems%
\footnote{Recall that we omit the dependence on parameters other than
  \ARMNUM, \MinProb, $T$ unless making a particular point.},
analyzing the payments and regret of the algorithm.

\begin{theorem} \label{rst:budget}
The expected total payment of
Algorithm~\ref{alg:basic-incentivizing} is at most
$O \left(\ARMNUM^2 \cdot \e^{2/\MinProb} \right)$.
\end{theorem}

\begin{theorem} \label{rst:regret}
For any time horizon $T$, the expected cumulative regret for
Algorithm~\ref{alg:basic-incentivizing} up to time $T$ is bounded
above by 
$O \left(\ARMNUM \cdot \e^{2/\MinProb} + \TieDensity \ARMNUM \log^3(T) \right)$.
\end{theorem}

When $\TieDensity = 0$, the bound of Theorem~\ref{rst:regret} is
constant in $T$;
thus, the algorithm achieves constant regret using constant
\dkedit{expected payments}. 
As discussed in Section~\ref{sec:prob},
the case $\TieDensity = 0$ arises, for instance, for
discrete agent distributions.
In fact, in the case of a discrete agent distribution,
it is possible to modify the algorithm to also
reduce the dependence on \MinProb from exponential to polynomial.
Theorem~\ref{rst:discrete} (given later) states that the modified
algorithm achieves expected regret
$O \left(\frac{\ARMNUM}{\MinProb^4} \right)$
with expected payments of
$O \left(\frac{\ARMNUM^2}{\MinProb^{5/2}} \right)$.

The fact that constant regret can be achieved with constant payment
(independent of $T$) when $\TieDensity = 0$ suggests aiming for a
constant bound more generally, i.e., for $\TieDensity > 0$.
That such a bound is unachievable is shown in Appendix~\ref{sec:lb},
where we show a lower bound of $\Omega(\log(T))$ on
the expected regret of any algorithm.
The instance is simple: it has two arms, one \pfedit{with}
attributes; \dkcomment{Should that be ``with \emph{known}
  attributes''? I don't understand this edit.}
in addition, one draw from the other arm is observed in
each step $t$ even when it is not pulled.
While the probability of pulling the wrong arm decreases over time, it
does not do so fast enough, causing the stated regret.

The exponential dependence on $1/\MinProb$ implies an exponential
dependence on \ARMNUM (because $\MinProb \leq 1/\ARMNUM$).
This exponential dependence arises from a need to continue to
incentivize arm pulls to ensure that nearly tied agents learn their
best arms quickly.
Aside from the assumption that $\TieDensity = 0$, another assumption
allows us to eliminate this exponential dependence.
Namely, when \MinProb (or a lower bound on it) is known ahead of time,
the algorithm can be modified to incentivize arms less aggressively.
As shown in Theorem~\ref{rst:known-p},
the modified algorithm has expected regret at most
$O \left(\frac{\ARMNUM}{\MinProb^3}
+\frac{\ARMNUM \TieDensity \log^3(T)}{\MinProb} \right)$,
with expected payments of at most
$O \left(\frac{\ARMNUM^2}{\MinProb^{5/2}} \right)$.

We compare these bounds to those for standard bandits,
focusing on the dependence on $T$.  
The standard bandit setting is the case when the agent types \AgV
are concentrated on a single point, and agents pull arms at the
principal's direction without requiring payment.
(This setting violates our Assumption~\ref{A3},
so our bounds do not apply to it.)
Then, the payment is $0$ and the expected regret scales as
$\Theta(\log(T))$ \cite[Theorem 2.1]{bubeck2012regret}.

Our algorithm's payment is constant in $T$,
while its regret is $O(\log^3(T))$ in general with a lower bound of
$\Omega(\log(T))$;
when preferences are discrete, our algorithm's regret is
constant in $T$.
Thus, viewed solely in terms of the dependence on $T$,
the best performance achievable seems comparable to that in a standard
multi-armed bandit problem;
but when preferences are discrete, the constant regret
surpasses the $\Theta(\log(T))$ achievable in the standard multi-armed
bandit \dkedit{setting}.
This may seem surprising, because the principal in our setting has
both less control and less information than in the standard bandit setting.
The result arises because heterogeneity in preferences provides
free exploration, and allows all of the arms to be pulled infinitely
often without incurring regret once estimates are accurate enough.

While heterogeneity in preferences enables this free exploration,
heterogeneity alone is not always sufficient for enabling performance
improvements compared to the standard bandit setting.
Indeed, suppose that agents are still heterogeneous,
but the principal pulls arms directly.
Unless the principal can also observe the agents' types,
she will be unable to correctly choose each agent's preferred arm,
even with infinite exploration of arm attributes.
Regret will then grow as $\Omega(T)$. 

Thus, reaping the benefits of (unobserved) heterogeneous preferences
requires the principal to give up direct control of the arms,
providing agents the autonomy they need to express their private
information about their own preferences.
Our results show that simple arm-based incentives are sufficient
to overcome the apparent challenges created by this abdication of
control.


\section{Main Algorithm and Analysis}
\label{sec:ub}
The algorithm achieving the claimed bounds of
Theorems~\ref{rst:budget} and~\ref{rst:regret} is simple.
It mostly allows agents to exploit, but when an arm is sufficiently
unlikely to be pulled,
it incentivizes this arm with a payment high enough
to guarantee that the next agent pulls it.
This way, the algorithm ensures that each arm is pulled often enough.

More precisely, the algorithm divides time into \emph{phases}
$s = 1, 2, 3, \ldots$.
Phase $s$ starts when each arm has been pulled at least $s$ times.
We indicate the start time of phase $s$ by $t_s$. An arm $i$ is \emph{payment-eligible} at time $t$ (in phase $s$)
if both of the following hold:

\begin{itemize}
\item $i$ has been pulled at most%
\footnote{in fact: exactly, since the algorithm entered phase $s$}
$s$ times up to time $t$, i.e., $\NumPull{t}{i} \leq s$.
\item 
The conditional probability of pulling arm $i$ is less than
$1/\log(s)$ given current estimates \ArmEV{t}{i'} of arms'
attribute vectors.  In other words, we require
$\PullProb{t}{i} < 1/\log(s)$ where 
$\PullProb{t}{i} = \Prob[\AgV \sim \AgentDist]{\AgV \cdot \ArmEV{t}{i} > \AgV
\cdot \ArmEV{t}{i'} \mbox{ for all } i' \neq i \mid \ArmEV{t}{i'}\ \forall i'}$
is the probability that arm $i$ will be pulled
by the next (random) agent based on the current estimates. 
\end{itemize}

When multiple arms are payment-eligible, the algorithm picks one arbitarily to incentivize.
When the algorithm decides to incentivize an arm $i$,
it offers ``whatever it takes,'' i.e., offers a payment of
$\Pay{t}{i} = \max_{\AgV,i'} \AgV \cdot (\ArmEV{t}{i'} - \ArmEV{t}{i})$.
The maximum for \AgV is taken over the support of \AgentDist;
recall that we assumed this support to be compact.
The payment \Pay{t}{i} may appear unnecessarily high.
Indeed, it suffices to
incentivize only a $1/\log(s)$ fraction of the agents,
and our bounds also hold for an alternate version of our algorithm that 
offers payment
$\Pay{t}{i} = \min \Set{c \geq 0}{%
\Prob[\AgV \sim \AgentDist]{c + \AgV \cdot \ArmEV{t}{i} \geq \max_{i'\ne i} \AgV \cdot \ArmEV{t}{i'}} \ge 1/\log(s)}$.
However, we focus on the higher payments for simplicity of presentation.

Notice that \PullProb{t}{i} depends on the estimates for \emph{all}
arms; thus, by pulling another arm $i'$, an arm $i$ may become
payment-eligible, or cease to be so.
Algorithm~\ref{alg:basic-incentivizing} gives the full details.


\begin{algorithm}
\caption{Algorithm: Incentivizing Exploration \label{alg:basic-incentivizing}}
\begin{algorithmic}
\STATE Set the current phase number $s = 1$.
\COMMENT{Each arm is pulled once initially ``for free''.}
\FOR{time steps $t = 1, 2, 3, \ldots$}
\IF{$\NumPull{t}{i} \geq s+1$ for all arms $i$}
\STATE Increment the phase $s = s + 1$.
\ENDIF
\IF{there is a payment-eligible arm $i$}
\STATE Let $i$ be an arbitary payment-eligible arm.
\STATE Offer payment
$\Pay{t}{i} = \max_{\AgV,i'} \AgV \cdot (\ArmEV{t}{i'} - \ArmEV{t}{i})$
for pulling arm $i$
(and payment 0 for all other arms).
\ELSE
\STATE Let agent $t$ play myopically, i.e., offer payments 0 for all arms.
\ENDIF
\ENDFOR
\end{algorithmic}
\end{algorithm}

The high-level idea in the proofs of our main results, Theorems~\ref{rst:budget} and~\ref{rst:regret},
is the following.
Because the algorithm ensures that each arm is pulled
``frequently enough,''
the estimates \ArmEV{t}{i} become gradually more accurate in the
phase number $s$.
Thus, the fraction of agents who misidentify their best arm decreases.
Because for each arm, enough agents have a true preference for this
arm, once the arms' attribute vectors are learned well enough,
the algorithm will not need to incentivize any more,
resulting in a payment bound independent of $T$.
Similarly, the regret will decrease, and mostly accrue
due to ``problematic'' agents who are nearly tied in their preferences
between their top two arm choices.
The detailed analysis following this outline is somewhat subtle, though,
due to the dependencies between the agents' arm pulls and the
estimates which in turn are based on past arm pulls.
We begin with several technical lemmas that are used for both the
payment and regret bounds.

To formally reason about the event that the estimates of arms'
attributes vectors are accurate enough --- or fail to be so ---
we define the events
$\AccE{t}{i}{j}{x} := [|\ArmE{t}{i}{j} - \Arm{i}{j}| \leq x]$
that attribute $j$ of arm $i$ at time $t$ is estimated to
within accuracy $x$ or better.
Then, 
$\AccEU{t}{x} = \bigcap_{i,j} \AccE{t}{i}{j}{x}$
is the event that at time $t$, all arm attribute
estimates are accurate to within $x$ simultaneously.
We will then show that for suitable choices of $t, x$,
the events \AccE{t}{i}{j}{x}
(and hence, by union bound, \AccEU{t}{x})
have high probability,
and that when they do, myopic agents do not make large mistakes.

%We now prove our main results, Theorems~\ref{rst:budget} and~\ref{rst:regret}.



\subsection{General Lemmas}

% We begin by showing that under our assumptions, the measure of
% problematic arms cannot be too large.

% \begin{lemma} \label{lem:sdelta}
% $\AlmostTied{\delta} \leq \TieDensity \cdot \delta$ for all $\delta$.
% \end{lemma}

% \begin{emptyproof}
% Using the upper bound from Assumption~\ref{A1}, we can bound

% \begin{align*}
% \AlmostTied{\delta}
% & = \sum_{i,i'} \ProbC{\AgV \cdot (\ArmV{i} - \ArmV{i'}) \leq \delta}%
%     {\AgV \in \FirstTwo{i}{i'}}
%   \cdot \Prob{\AgV \in \FirstTwo{i}{i'}}\\
% & \leq \sum_{i,i'} \TieDensity \cdot \delta
%     \cdot \Prob{\AgV \in \FirstTwo{i}{i'}}
% \; = \; \TieDensity \cdot \delta. \QED
% \end{align*}
% \end{emptyproof}

We begin by bounding the length of any phase, which will support bounding
the regret of early rounds
(before tail bounds have kicked in). The proof of Lemma~\ref{lem:phase-length} and all other proofs missing in the main paper may be found in the appendix.
  
\begin{lemma} \label{lem:phase-length}
For any $s\geq 3$, the expected length of phase $s$ is at most
$\ARMNUM \cdot \log(s)$ time steps.
\end{lemma}
                  
We now state the key technical lemma which captures the intuition that
the estimates of the arms' attribute vectors become
more accurate with increasing phases $s$.

\begin{lemma} \label{lem:round-prob}
Recall the noise is a mean-zero sub-Gaussian($\sigma^2$) random variable.
Let \LatePhase be fixed, 
and let $x_n, x'_n > 0$ be sequences satisfying 
$\sqrt{0.6 n \cdot \log (\log_{1.1}(n) + 1) + \frac{n x_n^2}{16 \sigma^2}}
\leq \frac{n x'_n}{2 \sigma}$,
for all $n \geq \LatePhase$.
Let $\tau_s$ be a stopping time
(which may depend on the entire past history)
which is almost surely in phase $s$,
i.e., satisfying $\tau_s \in [t_s, t_{s+1})$ almost surely.
Then, for any arm $i$, attribute $j$, and phase $s \geq \LatePhase$,
we have 
$\Prob{\AccE{\tau_s}{i}{j}{x'_s}}
\geq 1 - 24 \exp\left(-\frac{1.8 s x_s^2}{16 \sigma^2} \right)$.
\end{lemma}

Next, we show the complementary result:
when the event \AccEU{t}{x} happens, no myopic agent incurs large regret.

\begin{lemma} \label{lem:right-choice}
Let $x > 0$ be arbitrary.
When \AccEU{t}{x} happens,
no agent \AgV will pull a highly suboptimal arm, i.e., an arm $i$ with 
$\AgV \cdot (\ArmV{\Best{\AgV}} - \ArmV{i}) > 2\Diam d x$.
\end{lemma}



\subsection{Bounding the Total Payment}

As a first step towards bound the total payment (and also regret),
we show that for sufficiently late phases,
under the event \AccEU{t}{x} for suitably small $x$,
the algorithm does not offer any payments.

\begin{lemma} \label{lem:no-incentives}
Fix an arm $i$.
Let $s \geq \exp(2/\MinProb)$, and let $\tau_s$ be the (random)
time when arm $i$ is pulled for the \Kth{s} time.
Under \AccEU{\tau_s}{\frac{p}{4\Diam d \TieDensity}},
this pull of arm $i$ is not incentivized.
\end{lemma}

Towards bounding the algorithm's total payment, we now bound the
\emph{number} of rounds in which the algorithm makes a payment by a
constant.
This bound also turns out to be useful for bounding the total regret.

\begin{lemma} \label{lem:numP}
The expected number of time steps in which
Algorithm~\ref{alg:basic-incentivizing}
makes any payment is at most $O\left( N\exp\left(\frac{2}{p}\right) \right)$.
\end{lemma}

\begin{proof}
We partition phases into early and late phases.
For each of the early phases,
we crudely bound the number of payments by \ARMNUM,
using that each arm is incentivized at most once per phase.
For later phases,
we use Lemma~\ref{lem:no-incentives},
which rules out any incentives unless large misestimates of the arm
locations occur, which is exponentially unlikely by 
Lemma~\ref{lem:round-prob}.

To make this intuition precise, we set
$\delta = \frac{\MinProb}{2 \TieDensity}$,
and $x = x' = \frac{\delta}{2 \Diam d}$
(which is independent of the phase number $s$).
The cutoff point between early and late phases is now set to
$\EvenLaterPhase = \max(2, \frac{30 \sigma^3}{x^3}, \exp(\frac{2}{\MinProb}))$.
We will verify below that
$\sqrt{0.6 n \cdot \log (\log_{1.1}(n) + 1) + \frac{n x^2}{16 \sigma^2}}
\leq \frac{n x}{2 \sigma}$
for all $n \geq \max(2, \frac{30 \sigma^3}{x^3})$.

Fix an arm $i$, and consider the \Kth{s} pull of an arm $i$,
for $s \geq \EvenLaterPhase$.
If this pull occurs before phase $s$, it was definitely not incentivized.
So we condition on the pull occurring in phase $s$,
and let $\tau_s$ be the time step when this pull occurred;
notice that this means that this must be the first pull of arm $i$ in
phase $s$.

By Lemma~\ref{lem:round-prob} (with our choice of $x = x'$),
and a union bound over all $i,j$, we bound the probability
$\Prob{\AccEU{\tau_s}{x}}
\geq 1 - 24 \ARMNUM d \cdot \exp\left(-\frac{1.8 s x^2}{16 \sigma^2} \right)$.
And by Lemma~\ref{lem:no-incentives},
under the event ${\mathcal E}_{\tau_s}(x)$,
arm $i$ is not payment-eligible.

Thus, for any $s \geq \EvenLaterPhase$,
the \Kth{s} pull of arm $i$ is incentivized  
with probability at most
$24 \ARMNUM d \cdot \exp \left(
  -\frac{1.8 s \cdot \MinProb^2}{256 \TieDensity^2 \Diam^2 d^2 \sigma^2}
\right)$.
Summing over all arms and phases $s$,
adding the at most $\ARMNUM \EvenLaterPhase$ incentivizations in
the first \EvenLaterPhase phases, 
and using that $\exp(-x) \leq 1-x/2$ for $x \leq 1$,
the expected total number of arm incentivizations is at most
\begin{align*}
\lefteqn{\ARMNUM \EvenLaterPhase
  + 24 \ARMNUM^2 d \cdot \frac{1}{1 - \exp \left(
    \frac{-1.8 \MinProb^2}{256 \TieDensity^2 \Diam^2 d^2 \sigma^2}
  \right)}}  \\
& = O\left( \max\left(\frac{\ARMNUM \TieDensity^3 \Diam^3 d^3 \sigma^3}{\MinProb^3}, N\exp(\frac{2}{p})\right)
  + \frac{\ARMNUM^2 d^3 \TieDensity^2 \sigma^2 \Diam^2}{\MinProb^2} \right)
\; = \; O\left( N\exp\left(\frac{2}{p}\right) \right).
\end{align*}

It remains to show that
$\sqrt{0.6 n \cdot \log (\log_{1.1}(n) + 1) + \frac{n x^2}{16 \sigma^2}}
\leq \frac{n x}{2 \sigma}$
for all $n \geq \max(2, \frac{30 \sigma^3}{x^3})$.
We first use that $\sqrt{\cdot}$ is sublinear to bound
\[
  \sqrt{0.6 n \cdot \log (\log_{1.1}(n) + 1) + \frac{n x^2}{16 \sigma^2}}
\; \leq \; \sqrt{0.6 n \cdot \log (\log_{1.1}(n) + 1)} + \frac{n x}{4 \sigma}
\]
(here we use the fact $\sqrt{n} \leq n$).
Next, we show that
$\sqrt{0.6 n \log (\log_{1.1}(n) + 1)} \leq \frac{n x}{4 \sigma}$;
then, adding the two terms gives the desired bound.

By squaring the claimed statement and rearranging,
it is equivalent to show that
$\frac{n}{\log(\log_{1.1}(n)+1)} \geq \frac{9.6\sigma^2}{x^2}$.
Because $n \geq 2$,
a numerical calculation and derivative test shows that
$\frac{n}{\log(\log_{1.1}(n)+1)} \geq n^{2/3}$,
and because $n \geq \frac{30 \sigma^3}{x^3}$,
we get that
$n^{2/3} \geq \frac{9.6 \sigma^2}{x^2}$, completing the proof.

\end{proof}

It would be desirable to simply identify a constant upper bound on the
payment made each time.
Unfortunately, while the agent types are drawn from a bounded support,
the noise in arm locations is not;
hence, with small probability, arm locations may be grossly
misestimated, resulting in high incentive payments.
As a result, the actual analysis of the total payment is significantly more
intricate;
the proof of Theorem~\ref{rst:budget} is given in
Appendix~\ref{sec:payment-proof}.

\subsection{Bounding the Total Regret}
In bounding the total regret, because agents' types are from a compact
set by Assumption~\ref{A2}, the maximum regret in any one round is
bounded by a constant.
We use $\MAXR = \max_{\AgV, i,i'} \AgV \cdot (\ArmV{i} - \ArmV{i'})$
to denote this constant upper bound on the maximum regret that can be
incurred in any one time step. 


\begin{emptyextraproof}{Theorem~\ref{rst:regret}}
Regret can arise in two ways:
(1) an agent was incentivized to pull a suboptimal arm, or
(2) an agent myopically pulled a suboptimal arm.
By Lemma~\ref{lem:numP}, Algorithm~\ref{alg:basic-incentivizing}
incentivizes agents at most 
$O\left( N\exp\left(\frac{2}{p}\right) \right)$
times in expectation, each time causing regret at most \MAXR,
for a total expected regret of at most
$O\left(\MAXR\cdot N\exp\left(\frac{2}{p}\right) \right)$.
For the rest of the proof, we focus on the regret incurred when agents
pull arms myopically and make mistakes.

We distinguish between agents incurring large regret
(which requires severe misestimates of arm locations;
such misestimates are exponentially unlikely to occur), 
and agents incurring small positive regret,
which requires these agents to be almost tied in their preference for
the best arm.
To be more precise, we define (with foresight) a phase-dependent
cutoff
$\EarlyCut{s} = \sqrt{\frac{128\log(s) \cdot \Diam^2 d^2 \sigma^2}{1.8 s}}$,
and consider a regret exceeding \EarlyCut{s} large.

For most of the proof, we will focus on the case $\EarlyCut{s} \leq \TieCutoff$.
The remaining phases are the ones with
$\frac{s}{\log s} \leq \frac{128}{1.8} \cdot \Diam^2 d^2 \sigma^2/\TieCutoff^2$.
Because $\frac{s}{\log s} = \Omega(s^{2/3})$, there are at most
$O(\Diam^3 d^3 \sigma^3/\TieCutoff^3)$ such phases,
each such phase lasts for at most $\ARMNUM \cdot \log(s)$ steps
in expectation by Lemma~\ref{lem:phase-length},
and each step incurs regret at most \MAXR.
Therefore, the total expected regret incurred in these phases is at most
$O(\ARMNUM \MAXR \cdot \Diam^3 d^3 \sigma^3/\TieCutoff^3
\cdot \log (\Diam^3 d^3 \sigma^3/\TieCutoff^3))
\leq O(\ARMNUM \MAXR \Diam^4 d^4 \sigma^4/\TieCutoff^4)
= O(\ARMNUM \MAXR)$,
by crudely bounding $\log y \leq O(y^{1/3})$.

For the remainder of the proof, we consider phases $s$
with $\EarlyCut{s} \leq \TieCutoff$.
We first consider agents \AgV incurring positive regret less than \EarlyCut{s}.
Let $A_s$ be the set of types satisfying 
$\AgV \cdot (\ArmV{\Best{\AgV}} - \ArmV{\Second{\AgV}}) \leq \EarlyCut{s}$.
By Assumption~\ref{A1}, and because $\EarlyCut{s} \leq \TieCutoff$,
the total measure of $A_s$
is $\AlmostTied{\EarlyCut{s}} \leq \TieDensity \cdot \EarlyCut{s}$.
Then, by Lemma~\ref{lem:phase-length},
the expected number of pulls in phase $s$ by agents in $A_s$  
is bounded above by $\TieDensity \EarlyCut{s} \cdot \ARMNUM \log(s)$.
Any agent \AgV incurring positive regret less than $\EarlyCut{s}$
must have a type in $A_s$,
so the expected total regret from such agents,
summed over all phases, is at most
\begin{equation}
\sum_{s=1}^{T} \TieDensity \EarlyCut[2]{s} \cdot \ARMNUM \log(s).
\label{equ:small_regret_bound}
\end{equation}

We next bound the regret incurred in time steps with large regret.
Define the stopping times $\tau_{s}^{k}$ to be the \Kth{k}
time step in the \Kth{s} phase,
with $\tau_{s}^{k} = \infty$ when phase $s$ has fewer than $k$ steps,
i.e., $k > t_{s+1}-t_{s}$.
By Lemma~\ref{lem:right-choice},
under \AccEU{\tau_s^k}{\frac{\EarlyCut{s}}{2\Diam d}},
no agent at time $t$ will incur regret more than \EarlyCut{s}.
  
To bound the probability of \AccEU{\tau_s^k}{\frac{\EarlyCut{s}}{2\Diam d}},
we use Lemma~\ref{lem:round-prob}
(with $x_s = x'_s = \frac{\EarlyCut{s}}{2 \Diam d}$
and a stopping time of $\tau_s^k$).
We first verify that this choice of $x_s, x'_s$
satisfies the assumption of Lemma~\ref{lem:round-prob} for all
$s \geq 2$,
i.e., that
$\sqrt{0.6 n \cdot \log (\log_{1.1}(n) + 1) + \frac{n x_n^2}{16 \sigma^2}}
\leq \frac{n x_n}{2 \sigma}$
for all $n \geq 2$.
Substituting that
$x_n = \frac{\EarlyCut{n}}{2 \Diam d} = \sqrt{\frac{32\log(n) \cdot \sigma^2}{1.8 n}}$,
the left-hand side is

\[
  \sqrt{0.6 n \cdot \log (\log_{1.1}(n) + 1) + \frac{2 \log n}{1.8}}
  \; \leq \;
  \sqrt{0.6 n \cdot \log (\log_{1.1}(n) + 1) + \frac{2 n \log n}{1.8}},
\]
while the right-hand side is $\sqrt{\frac{8 n \log(n)}{1.8}}$.
Squaring both sides and canceling out common terms,
the desired inequality is equivalent to
$\frac{50}{9} \log n \geq \log(\log_{1.1}(n) + 1)$.
This can be verified by numerical calculation for $n=2$ and a
derivative test.

Now, by Lemma~\ref{lem:round-prob} and a union bound over all arms $i$
and attributes $j$, we get that 
\begin{align*}
\Prob{\AccEU{\tau_s^k}{x_s}}
& \leq 24 \ARMNUM d \cdot \exp \left( \frac{-1.8 s x_s^2}{16 \sigma^2} \right)
\; = \; \frac{24 \ARMNUM d}{s^2}.
\end{align*}

In the low-probability event, we simply bound the maximum regret by \MAXR.
Summing over \dkedit{all} time periods,
the total expected regret from large-regret steps is at most
$24 \ARMNUM d \MAXR \cdot \frac{\pi^2}{6} = O(\ARMNUM \MAXR)$.

Adding all four types of regret terms,
and substituting 
$\EarlyCut[2]{s} = \frac{128 \dkedit{\log(s)} \cdot \Diam^2 d^2 \sigma^2}{1.8 s}$,
the cumulative regret up to time $T$ is at most
$
O \left(\ARMNUM \MAXR \cdot \exp \left( \frac{2}{p} \right)
+ \ARMNUM \MAXR
+ \ARMNUM \MAXR
+ \ARMNUM \TieDensity \cdot \sum_{s=1}^{T} \EarlyCut[2]{s} \log(s) \right)\\
= 
O \left(\ARMNUM  \cdot \exp \left( \frac{2}{p} \right)
+ \ARMNUM \TieDensity \cdot \log^3(T) \right)$.\QED
\end{emptyextraproof}


\subsection{Tighter Bounds Under Additional Assumptions}
%: Discrete Preferences and Known \MinProb}

The proofs of Theorem~\ref{rst:budget} and Theorem~\ref{rst:regret} incur exponential (in \ARMNUM) payment and regret in the initial phases because the threshold required for incentivization $1/\log(s)$ decreases slowly.
This slow decrease is needed to bound the regret in the later phases when the concentration inequality kicks in as in Equation~(\ref{equ:small_regret_bound}).
In this section, we discuss two restrictions on the problem setting under which  we can modify the algorithm slightly and provide stronger bounds, avoiding this exponential dependence.
Both problem settings are special cases of the more general problem setting previously considered.

\subsubsection{Discrete Preferences}
\label{subsec:discrete}
Discrete preferences by agents are captured by the following assumption (which states that the agents who are close to tied between two arms have measure 0):

\begin{assumption}[Discrete Preferences]
\label{a:discrete}
There is a positive $\delta_0$ such that
$\AlmostTied{\delta_0} = 0$.
\end{assumption}

When Assumption~\ref{a:discrete} holds, we restrict the payment-eligibility criterion by only incentivizing arms with much smaller probability to be pulled: an arm $i$ is \emph{payment-eligible} at time $t$ in phase $s$ when both of the following are true:
\begin{itemize}
\item $i$ has been pulled at most
$s$ times up to time $t$, i.e., $\NumPull{t}{i} \leq s$.
\item Based on the current estimates \ArmEV{t}{i'} of all arms' attribute vectors, the probability of pulling arm $i$ is less than $1/s$ (compared to $1/\log(s)$ in the general algorithm).
\end{itemize}

We refer to this modified version of Algorithm~\ref{alg:basic-incentivizing} as the \emph{Discrete-Preference Algorithm.}
We outline a proof of the following result for this algortithm under Assumption~\ref{a:discrete} in Appendix~\ref{sec:discussion-proof1}.

\begin{theorem}
\label{rst:discrete}
Under Assumption~\ref{a:discrete}, the Discrete-Preference Algorithm has expected payment budget bounded above by 
\begin{align*}
O \left(\frac{\ARMNUM^2 d \cdot (\MAXR + \Diam d \sigma)\sigma^{5/2}L^{5/2}\Diam^{5/2}d^{5/2}}{\MinProb^{5/2}} \right),
\end{align*}
and expected regret bounded above by 
\begin{align}
O \left(\MAXR\cdot \frac{\ARMNUM \TieDensity^4 \Diam^4 d^4 \sigma^4}{\MinProb^4}
  + \frac{\MAXR \ARMNUM^2 d^3 \TieDensity^2 \Diam^2 \sigma^2}{\MinProb^2}
  \right).  \nonumber
\end{align}
\end{theorem}



\subsubsection{Known \MinProb}
An alternative useful assumption is that \MinProb (or a strictly positive lower bound on \MinProb) is known.

\begin{assumption}[Known $\MinProb$]
\label{a:known-p}
A strictly positive lower bound on \MinProb is known in advance.
\end{assumption}

When this assumption holds, we choose a different modification in the definition of payment eligibility.
Let $s_0 = \exp(2/\MinProb)$.
An arm $i$ is \emph{payment-eligible} at time $t$ (in phase $s$)
if both of the following hold:
\begin{itemize}
\item $i$ has been pulled at most
$s$ times up to time $t$, i.e., $\NumPull{t}{i} \leq s$.
\item Based on the current estimates \ArmEV{t}{i'} of all arms' attribute vectors, the probability of pulling arm $i$ is less than $1/\log(s+s_0)$.
\end{itemize}

This knowledge of \MinProb allows the algorithm to incentivize
significantly fewer arm pulls.
We refer to this modified version of Algorithm~\ref{alg:basic-incentivizing} as the \emph{Known-\MinProb Algorithm.}
We outline a proof of the following result for this algortithm under Assumption~\ref{a:known-p} in Appendix~\ref{sec:discussion-proof2}.

\begin{theorem}
\label{rst:known-p}
Under Assumption~\ref{a:known-p}, the Known-\MinProb Algorithm has expected payment budget bounded above by 
\begin{align*}
O \left(\frac{\ARMNUM^2 d \cdot (\MAXR + \Diam d \sigma)\sigma^{5/2}L^{5/2}\Diam^{5/2}d^{5/2}}{\MinProb^{5/2}} \right),
\end{align*}
and expected regret bounded above by
\begin{align*}
O \left(\frac{\MAXR \ARMNUM \TieDensity^3 \Diam^3 d^3 \sigma^3}{\MinProb^3}
  + \frac{\MAXR \ARMNUM^2 d^3 \TieDensity^2 \Diam^2 \sigma^2}{\MinProb^2}
  + \frac{\ARMNUM^2 d\MAXR}{\MinProb}
  + \frac{\ARMNUM\Diam^2 d^2\sigma^2 \TieDensity(\log(T))^3}{\MinProb}\right).
\end{align*}
\end{theorem}



\section{Conclusion}
We study the problem of incentivizing exploration with heterogeneous
user preferences.
We proposed an algorithm that achieves expected cumulative regret of
$O(\ARMNUM \e^{2/\MinProb} + \ARMNUM \log^3(T))$,
using expected cumulative payments of $O(\ARMNUM^2 \e^{2/\MinProb})$.
It is possible to improve these bounds to polynomial (in \ARMNUM and
$1/\MinProb$) when \MinProb is known or the preference distribution is
discrete.
In fact, we conjecture that this should be possible even in the full
generality of our model.
As a first step towards such a polynomial bound, we believe that it
should be possible to obtain an exponential dependence on
$1/(\MinProb \ARMNUM)$, which gives polynomial dependence unless some
arm has a much smaller fraction of the population preferring it.

Taking this goal one step further, we would like to 
develop algorithms that do not require all arms to be preferred by a
strictly positive fraction of agents.
An alternate algorithm might only incentivize an arm if its estimated
attribute vector is close enough to a Pareto frontier.
The regret will then be $\Omega(\log(T))$ when at least one arm falls
below the Pareto frontier, as we no longer have free exploration of
all arms. 
It is likely that a bound will deteriorate as the number of such
unpreferred arms increases.

Finally, it would be desirable to generalize the class of utility
functions that can be handled beyond inner products.
We believe that similar results hold for arbitrary
Lipschitz-continuous utility functions of the arm's attribute vector,
and that only minor modifications are necessary to the algorithm and
proofs.


\acks{PF and BC were partially supported by NSF CMMI-1254298 and AFOSR
FA9550-15-1-0038.
DK was supported in part by NSF grant 1423618.}

\bibliography{reference}

\appendix

\section{A Lower Bound of $\Omega(\log(T))$} \label{sec:lb}

We saw in Theorem~\ref{rst:discrete} that 
when agent preferences for their best arm are sufficiently clear,
in the sense that $\TieDensity = 0$, the regret of
Algorithm~\ref{alg:basic-incentivizing} is bounded by a constant.
One may conjecture that this should hold more generally,
in that the regret of the (fewer and fewer) agents on the boundary
between close arms should go to 0, while their fraction also goes to 0.
In this section, we establish a lower bound,
showing that even in very simple settings, a (logarithmic) dependence
on $T$ is typically unavoidable for \emph{any} incentivization strategy.

We consider an instance with two arms,
whose attribute vectors are $(0,0)$ and $(0,1)$, respectively.
Agent types are distributed uniformly on the (edge of the)
two-dimensional unit square%
\footnote{While our model technically assumed that all agent coordinates are
non-negative, we could simply shift the unit square.
The present choice is solely for ease of notation.}  
$\Set{\AgV}{\max(|\Ag{1}|, |\Ag{2}|) = 1}$,
with density $\frac{1}{8}$.
Because $\AgV \cdot (0,0) = 0$ and $\AgV \cdot (0,1) = \Ag{2}$,
the best choice for agent \AgV is arm $(0,1)$ iff $\Ag{2} > 0$,
i.e., the top half of the unit square prefers the arm $(0,1)$,
and the bottom half prefers the arm $(0,0)$.

Since we are proving a lower bound, we give the algorithm the
following extra two advantages:
(1) there is no noise in the observations of the arm $(0,0)$,
and all agents know its location deterministically.
(2) in each time step, regardless of which arm is pulled, the algorithm
and all agents observe a pull from arm $(0,1)$.
For simplicity of notation, we set the standard deviation of the arm
$(0,1)$ to $\sigma = 1$;
different values only lead to a scaling of the results.

Under these advantages, myopic play is clearly optimal, so it suffices
to bound the regret of the myopic algorithm which never incentivizes
agents. 

Because a pull of arm $(0,1)$ is observed in each time step,
after $t$ rounds, the estimate \ArmEV{t}{(0,1)} is of the form
$(0,1) + (\Noise[t]{1}, \Noise[t]{2})$,
where $\Noise[t]{1} \sim \Normal{0}{\sqrt{1/t}}$
and $\Noise[t]{2} \sim \Normal{0}{\sqrt{1/t}})$.
We lower-bound the regret in step $t$ by focusing on the event that
both normal noise coordinates are non-negative,
which by symmetry has probability \quarter:

\begin{align*}
\Expect{\Regret{t}} 
  & \geq \ExpectC{\Regret{t}}{\Noise[t]{1} > 0, \Noise[t]{2} > 0}
         \cdot \Prob{\Noise[t]{1} > 0, \Noise[t]{2} > 0}
  \; = \; \quarter \ExpectC{\Regret{t}}{\Noise[t]{1} > 0, \Noise[t]{2} > 0}.
\end{align*}

For the moment, focus on some time step $t$,
and write $\NoiseV = \NoiseV[t]$.
Then, there are two types of agents who pull the wrong arm and incur
regret:
\begin{enumerate}
\item If $\Ag{2} > 0$ and $\Ag{1} \Noise{1} + \Ag{2} (1+\Noise{2}) < 0$
then \AgV should pull $(0,1)$,
but will wrongly pull $(0,0)$ and incur regret \Ag{2}.
The range of \AgV making the wrong choice is thus
$0 < \Ag{2} < \frac{- \Noise{1}}{1+\Noise{2}} \cdot \Ag{1}$.
Since we are proving a lower bound, we only focus on the
case $\Ag{1} = -1$, and ignore the case $\Ag{2} = 1$
(which is rare, since it requires \Noise{1} to be large).
Thus, the set of agents incurring regret contains the set
$\Set{\AgV = (-1, \Ag{2})}{0 < \Ag{2} < \frac{\Noise{1}}{1+\Noise{2}}}$.

% For any particular value of \Ag{2},
% the length of the interval of corresponding \Ag{1} is
% \begin{align*}
%   \sqrt{1-\Ag{2}^2} - \frac{1+\Noise{2}}{\Noise{1}} \cdot \Ag{2}
%   & \geq 1 - \Ag{2} - \frac{1+\Noise{2}}{\Noise{1}} \cdot \Ag{2}
%   \; = \; 1 - \frac{1+\Noise{1}+\Noise{2}}{\Noise{1}} \cdot \Ag{2}.
% \end{align*}

\item If $\Ag{2} < 0$ and $\Ag{1} \Noise{1} + \Ag{2} (1+\Noise{2}) > 0$,
then \AgV should pull $(0,0)$,
but will wrongly pull $(0,1)$ and incur regret $-\Ag{2}$.
This region and its regret are rotationally symmetric to the previous
case.
\end{enumerate}

Hence, the expected regret for given \Noise{1}, \Noise{2} is at least
%\begin{align*}
$2 \int_0^{\frac{\Noise{1}}{1+\Noise{2}}}
  \Ag{2} \cdot \frac{1}{8} \dd \Ag{2}
=
\frac{1}{8} \cdot \left(\frac{\Noise{1}}{1+\Noise{2}}\right)^2$.
%\end{align*}
The expected regret, conditioned on
$\Noise{1} > 0$ and $\Noise{2} > 0$, is therefore

\begin{align*}
\ExpectC{\Regret{t}}{\Noise{1} > 0, \Noise{2} > 0}
 & \geq \frac{1}{\Prob{\Noise{1}>0,\Noise{2}>0}} \cdot \frac{1}{8}
    \int_{0}^{\infty} \int_{0}^{\infty}
    \left( \frac{\Noise{1}}{1+\Noise{2}} \right)^2 \cdot
    \frac{\e^{-\frac{t \Noise{1}^2}{2}} \sqrt{t}}{\sqrt{2\pi}} \dd\Noise{1}
    \frac{\e^{-\frac{t \Noise{2}^2}{2}} \sqrt{t}}{\sqrt{2\pi}} \dd\Noise{2} \\
 & = \frac{1}{4\pi} \int_{0}^{\infty} \int_{0}^{\infty}
   \left( \frac{\Noise{1}}{\sqrt{t}+\Noise{2}} \right)^2
   \e^{-\Noise{1}^2/2} \e^{-\Noise{2}^2/2} \dd \Noise{1} \dd \Noise{2}.
\end{align*}

We want to show that the expected regret per time step decreases only
at a rate of $\Omega(1/t)$, and thereto consider the limit of the
ratio of the two quantities:

\begin{align*}
\lim_{t \to \infty} \frac{\ExpectC{\Regret{t}}{\Noise{1} > 0, \Noise{2} > 0}}{\frac{1}{t}}
  & \geq
    \lim_{t \to \infty} \left( \frac{1}{4\pi} \cdot
    \int_{0}^{\infty} \int_{0}^{\infty}
    t \cdot \left( \frac{\Noise{1}}{\sqrt{t}+\Noise{2}} \right)^2
    \e^{-\Noise{1}^2/2} \e^{-\Noise{2}^2/2} \dd \Noise{1} \dd \Noise{2} \right) \\
  & = \frac{1}{4\pi} \cdot \int_{0}^{\infty} \int_{0}^{\infty}
    \lim_{t \to \infty}
    \left( t \cdot \left( \frac{\Noise{1}}{\sqrt{t}+\Noise{2}} \right)^2 \right)
    \e^{-\Noise{1}^2/2} \e^{-\Noise{2}^2/2} \dd \Noise{1} \dd \Noise{2} \\
  & = \frac{1}{4\pi} \cdot \int_{0}^{\infty} \int_{0}^{\infty}
    \e^{-\Noise{1}^2/2} \e^{-\Noise{2}^2/2} \dd \Noise{1} \dd \Noise{2}
  \; = \; \frac{1}{4 \pi}; 
\end{align*}

the second line is due to the monotone convergence theorem,
which can be applied because
$t \left(\frac{\Noise{1}}{\sqrt{t}+\Noise{2}}\right)^2$
is strictly increasing in $t$.
Because the expected regret in each time step is at least
$\Omega(1/t)$, the total expected regret is at least
$\Omega(\sum_{t=1}^{T}\frac{1}{t}) = \Omega(\log(T))$.


\section{Proof of Lemma~\ref{lem:phase-length}}
\label{sec:lemma4-proof}

\begin{rlemma}{Lemma}{\ref{lem:phase-length}}
For any $s\geq 3$, the expected length of phase $s$ is at most
$\ARMNUM \cdot \log(s)$ time steps.

More generally, for any set of types $A$, the expected number of times that 
an agent with a type in $A$ appears in a phase $s$ is at most 
$f(A) \cdot \ARMNUM \cdot \log(s)$,
where $f(A) := \Prob[\AgV \sim \AgentDist]{\AgV \in A}$.
\end{rlemma}

\begin{proof}
We show the second claim for general $A$,
which implies the first claim by taking $A$ to be the support of $\AgentDist$.

Fix a phase $s$, and let $S_t$ be the set of arms that have been
pulled at most $s$ times by time $t$.
Let $\tau_m$ be the maximum of $t_s$ (the start of phase $s$)
and the time $t$ when $S_t$ first contains (at most) $m$ elements.
The phase ends when $S_t$ contains no elements, i.e., $\tau_0 = t_{s+1}$.
We consider the expected number of pulls during times
$t \in [\tau_m, \tau_{m-1})$ made by agents whose types are in $A$. 

Consider a counter starting from $0$ at time $\tau_m$ that increments
each time an agent arrives with a type in $A$,
and that is stopped the first time an arm in $S_t$ is pulled.
For any stopping time $t \in [\tau_m, \tau_{m-1})$, 
the conditional probability that the counter increments,
given the history of pulls, payments, and observations at time $t$,
is $f(A)$.
The conditional probability that the counter is stopped by that
agent's pull is at least $1/\log(s)$.
To see this, consider two cases.
In the first case, there is at least one payment-eligible arm,
in which case the principal incentivizes an arm in $S_t$.
In the second case, there are no payment-eligible arms.
Then, by definition of payment-eligibility,
each arm $i \in S_t$ has conditional probability at least $1/\log(s)$ of being pulled.

The expected value of this counter when it is stopped is bounded above
by the expected stopped value of another counter whose conditional
probability of being stopped is exactly $1/\log(s)$.
(This can be shown more formally via a coupling argument.) 
Let $V_\ell$ be the conditional expectation of this second counter's value when it is stopped,
given that its current value is $\ell$ and it has not been stopped.
Observe that $V_\ell = \ell + V_0$.
Enumerating outcomes
(either the counter increments or not; either it stops or it continues)
gives us the recurrence relation
$V_0 = f(A) \frac{1}{\log(s)} + f(A) (1-\frac{1}{\log(s)}) V_1
   + (1-f(A))(1-\frac{1}{\log(s)}) V_0$.
Using that $V_1 = 1 + V_0$, this equation simplifies to 
$V_0 = f(A) + (1-\frac{1}{\log(s)}) V_0$.
Noting that $V_0$ is finite and solving for $V_0$ gives us that
$V_0 = f(A) \log(s)$.
Thus, the conditional expected number of pulls (given the history at time $\tau_m$)
by an agent with a type in $A$ during times in $[\tau_m, \tau_{m-1})$
is no more than $f(A) \log(s)$.

We may write the total number of pulls by agents with types in $A$
during phase $s$ as a sum of this quantity over $m$ ranging from
$|S_{t_s}|$ down to 1.
Noting that $|S_{t_s}| \leq \ARMNUM$ and taking the expectation of
this sum completes the proof. 
\end{proof}







\section{Proof of Lemma~\ref{lem:round-prob}}
\label{sec:lemma5-proof}

\begin{rlemma}{Lemma}{\ref{lem:round-prob}}
Recall the noise is a mean-zero sub-Gaussian($\sigma^2$) random variable.
Let \LatePhase be a phase cutoff, 
and let $x_n, x'_n > 0$ be functions satisfying that
$\sqrt{0.6 n \cdot \log (\log_{1.1}(n) + 1) + \frac{n x_n^2}{16 \sigma^2}}
\leq \frac{n x'_n}{2 \sigma}$,
for all $n \geq \LatePhase$.
Let $\tau_s$ be a stopping time
(which may depend on the entire past history)
which is almost surely in phase $s$,
i.e., satisfying $\tau \in [t_s, t_{s+1})$ almost surely.

Then, for any arm $i$, attribute $j$, and phase $s \geq \LatePhase$,
we have that
$\Prob{\AccE{\tau_s}{i}{j}{x'_s}}
\geq 1 - 24 \exp\left(-\frac{1.8 s x_s^2}{16 \sigma^2} \right)$.
\end{rlemma}

The proof of Lemma~\ref{lem:round-prob} is based on an adaptive
concentration inequality due to \cite{zhao2016adaptive},
given as Lemma~\ref{lem:ACI-inequality}.

\begin{lemma}[Corollary 1 of \cite{zhao2016adaptive}]
\label{lem:ACI-inequality}
Let $X_i$ be zero-mean $1/2$-subgaussian random variables,
and $\SET{S_n = \sum_{i=1}^n X_i, n \geq 1}$ the corresponding random walk.
Let $J$ be any stopping time with respect to $\SET{X_1, X_2, \ldots}$.
(We allow $J$ to take the value $\infty$,
defining $\Prob{J = \infty} = 1 - \lim_{n \to \infty} \Prob{J \leq n}$.)
Define 
$g(n)  = \sqrt{0.6 n \cdot \log (\log_{1.1}(n) + 1) + n \cdot b}$.

Then, 
$\Prob{J < \infty \mbox{ and } S_J \geq g(J)} \leq 12 \e^{-1.8 b}$.
\end{lemma}

\begin{extraproof}{Lemma~\ref{lem:round-prob}}
Fix an arm $i$ and attribute $j$.
By the assumptions of the lemma,
the stopping time $\tau_s$ is such that almost surely,
each arm --- and in particular arm $i$ --- has been pulled at least
$s \geq \LatePhase$ times at time $\tau_s$.
Define $J$ to be the number of times that $i$ has been pulled at
time $\tau_s$.
For any $n \geq 1$, let $k_n$ be the time step right after arm $i$ has
been pulled for the \Kth{n} time. Define
$S_n := \frac{n \cdot (\ArmE{k_n}{i}{j} - \Arm{i}{j})}{2 \sigma}$
to be the sum of all attribute-$j$ noise components up to and
including the \Kth{n} pull of arm $i$,
renormalized to be a mean-zero sub-Gaussian($1/2$) random variable.
The $S_n$ define an unbiased half-subgaussian random walk,
and we can therefore apply Lemma~\ref{lem:ACI-inequality} to them and
the stopping time $J$.
Specifically, we set $b = \frac{x_J^2}{16 \sigma^2}$,
and obtain that

\begin{align*}
\Prob{J < \infty \mbox{ and } S_J \geq 
\sqrt{0.6 J \cdot \log (\log_{1.1}(J) + 1) + \frac{J x_J^2}{16 \sigma^2}}}
& \leq 12 \exp \left( \frac{-1.8 J x_J^2}{16 \sigma^2} \right).
\end{align*}

Applying Lemma~\ref{lem:ACI-inequality} to $-S_n$ with the same choice
of $b$, and taking a union bound over both cases, we obtain that

\begin{align*}
\Prob{J < \infty \mbox{ and } |S_J| \geq 
\sqrt{0.6 J \cdot \log (\log_{1.1}(J) + 1) + \frac{J x_J^2}{16 \sigma^2}}}
& \leq 24 \exp \left( \frac{-1.8 J x_J^2}{16 \sigma^2} \right).
\end{align*}

Because $J$ will be finite with probability 1, we can drop the
$J < \infty$ part of the event:
\begin{align*}
& \Prob{S_J \geq 
\sqrt{0.6 J \cdot \log (\log_{1.1}(J) + 1) + \frac{J^2 x_J^2}{16
    \sigma^2}}}\\
= & \Prob{J < \infty \mbox{ and } S_J \geq 
\sqrt{0.6 J \cdot \log (\log_{1.1}(J) + 1) + \frac{J^2 x_J^2}{16
    \sigma^2}}}.
\end{align*}

In the high-probability case, we now apply the assumed inequality
between $x_J$ and $x'_J$, to obtain that
\[
|S_J| \; \leq \;
\sqrt{0.6 J \cdot \log (\log_{1.1}(J) + 1) + \frac{J x_J^2}{16 \sigma^2}}
\; \leq \; \frac{J x'_J}{2 \sigma}.
\]

Substituting the definition of $S_J$ and canceling common terms,
the inequality implies that
$|\ArmE{k_J}{i}{j} - \Arm{i}{j}| \leq x'_J$.
The choice of $J$ ensures that
$\ArmE{k_J}{i}{j} = \ArmE{\tau_s}{i}{j}$,
and we have thus shown that
$|\ArmE{\tau_s}{i}{j} - \Arm{i}{j}| \leq x'_s$.
\end{extraproof}




\section{Proof of Lemma~\ref{lem:right-choice}}
\label{sec:lemma7-proof}

\begin{rlemma}{Lemma}{\ref{lem:right-choice}}
Let $x > 0$ be arbitrary.
When \AccEU{t}{x} happens,
no agent \AgV will pull a highly suboptimal arm, i.e., an arm $i$ with 
$\AgV \cdot (\ArmV{\Best{\AgV}} - \ArmV{i}) > 2\Diam d x$.
\end{rlemma}

\begin{proof}
By definition, when \AccEU{t}{x} happens,
for all arms $i$ and attributes $j$,
all arm attribute estimates are accurate to within $x$,
in the sense that
$|\ArmE{t}{i}{j} - \Arm{i}{j}| \leq x$.

Consider any agent type \AgV.
Let $i \neq \Best{\AgV}$ be any arm
with much smaller true reward than the best arm:
$\AgV \cdot (\ArmV{\Best{\AgV}} - \ArmV{i}) > 2\Diam d x$.
Because each coordinate of \ArmEV{t}{\Best{\AgV}} and of
\ArmEV{t}{i} is estimated accurately to within $x$, 
we get that 
$\AgV \cdot (\ArmEV{t}{\Best{\AgV}} - \ArmV{\Best{\AgV}})
\geq - \Diam d x$
and
$\AgV \cdot (\ArmEV{t}{i} - \ArmV{i}) \leq \Diam d x$.
Hence, 

\begin{align}
\AgV \cdot (\ArmEV{t}{\Best{\AgV}} - \ArmEV{t}{i})
& =
\AgV \cdot (\ArmEV{t}{\Best{\AgV}} - \ArmV{\Best{\AgV}})
+ \AgV \cdot (\ArmV{\Best{\AgV}} - \ArmV{i})
+ \AgV \cdot (\ArmV{i} - \ArmEV{t}{i}) \nonumber\\
& > -\Diam d x + 2 \Diam d x - \Diam d x
\; = \; 0, \label{equ:choice}
\end{align}
which means that the agent with type \AgV will not pull arm $i$.
\end{proof}


\section{Proof of Lemma~\ref{lem:no-incentives}}
\label{sec:lemma8-proof}

\begin{rlemma}{Lemma}{\ref{lem:no-incentives}}
Fix an arm $i$.
Let $s \geq \exp(2/\MinProb)$, and let $\tau_s$ be the (random)
time when arm $i$ is pulled for the \Kth{s} time.
Let $\hat{x} = \frac{1}{2\Diam d}
  \cdot \min(\TieCutoff, \frac{\MinProb}{2\TieDensity})$.
Under \AccEU{\tau_s}{\hat{x}},
this pull of arm $i$ is not incentivized.
\end{rlemma}

\begin{proof}
By Lemma~\ref{lem:right-choice},
under the event \AccEU{\tau_s}{\hat{x}},
all agent types \AgV with
$\AgV \cdot (\ArmV{\Best{\AgV}} - \ArmV{\Second{\AgV}})
> 2\Diam d \hat{x}$
will pull their best arm \Best{\AgV}.
Notice that $2\Diam d \hat{x}
= \min(\TieCutoff, \frac{\MinProb}{2 \TieDensity})$

By Assumption~\ref{A1},
the measure of agents (across all arms)
whose best and second-best arm differ in utility by less
than $\min(\TieCutoff, \frac{\MinProb}{2 \TieDensity})$
is at most
$\TieDensity \cdot \min(\TieCutoff, \frac{\MinProb}{2 \TieDensity})
\leq \frac{\MinProb}{2}$.
In particular, this bound holds for agents whose best arm is $i$.
By Assumption~\ref{A3}, at least a measure \MinProb of agents has $i$
as their best arm, and thus, at least a measure $\frac{\MinProb}{2}$
will myopically pull arm $i$.
Because $1/\log(s) \leq \frac{\MinProb}{2}$ for
$s \geq \exp(2/\MinProb)$, arm $i$ is not payment-eligible at time
$\tau_s$.
\end{proof}


\section{Proof of Theorem~\ref{rst:budget}}
\label{sec:payment-proof}

We restate the theorem here for convenience:

\begin{rtheorem}{Theorem}{\ref{rst:budget}}
\dkcomment{Copy here the final form of the theorem once it is finalized.}
\end{rtheorem}

The high-level idea of the proof is motivated by Lemma~\ref{lem:numP},
which shows that the expected \emph{number} of payments is constant.
Unfortunately, in contrast to the regret, there is no hard upper bound
on the payment in any one round.
If a draw of a particular arm comes out wildly inaccurate --- which is
an event of low but positive probability ---
then agents may require very large incentives to pull this arm again
in the future (and correct the inaccurate estimate).
The high payments are offset by the exceedingly low probability of
having to incur them, but a rigorous analysis requires some care:
if a high payment is required in one phase, this indicates a very
inaccurate estimate, which may require multiple phases to correct.
Hence, we need to handle dependency of payments across time steps and
phases.

To reason about such estimation errors formally, we define
\emph{envelopes} of sample paths.
A sample path \SP captures all the random events affecting the
algorithm, i.e., the random draws \AgentV{t} of agents and the
noise \NoiseV[t] in the draws the pulled arms,
with an infite \dkcomment{Is it infinite?} time horizon.
\dkcomment{We should be consistent about using filtration vs.~sample
  path.}
\pfcomment{I removed discussion of filtrations and replaced them by discussion of history.}

With foresight, we define $g(s, \ell) := \frac{12 \sigma \ell}{s^{2/5}}$.
Let $\ErrV{t}{i} = \ArmEV{t}{i} - \ArmV{i}$ be the estimation
error for the attribute vector \ArmV{i} at time $t$,
with components \Err{t}{i}{j}.
For any sample path \SP, let $s (t,\SP)$ be the phase number that the
algorithm is in at time $t$ with the sample path \SP.
Define the sets

\begin{align*}
\hat{L}_\ell
  & = \Set{\SP}{|\Err{t}{i}{j}(\SP)| \leq g(s(t,\SP),\ell)
    \mbox{ for all } i,j,t},\\
\Env{1} & = \hat{L}_1\\ 
\Env{\ell} & = \hat{L}_{\ell} - \hat{L}_{\ell-1}
  \qquad \mbox{ for } \ell \geq 2.
\end{align*}

% \begin{align*}
% L'[\ell](t)
%   & = \Set{\SP}{|\Err{t}{i}{j}(\SP)|\leq g(s(t,\SP),\ell), \forall i,j\}
% \end{align*}
We call \Env{\ell} the \Kth{\ell} envelope.
In words, \Env{\ell} consists of all sample paths such that at all times
$t$, all coordinates of all arm estimation errors are bounded by
$g(s, \ell)$,
but for at least one time $t$, at least one coordinate of one arm
estimation error is \emph{not} bounded by $g(s, \ell-1)$.
When \SP is clear, we omit it in the notation for
\Err{t}{i}{j}, payments, etc.
The importance of envelopes is that for small $\ell$, the payments are
tightly bounded, while for large $\ell$, the cumulative probability of
the sample paths in \Env{\ell} is small.
This is captured by the following two lemmas.

\begin{lemma} \label{lem:sample-path-payment}
If $\SP \in \Env{\ell}$ and $s(t,\SP) = s$, then
the payment in step $t$ is upper-bounded by
$\bar{c} (s,\ell) = \MAXR + 2 \Diam d \cdot g(s,\ell)$.
\dkcomment{Verify that the way I talk about $t$ vs.~$s$ is right.}
\end{lemma}

\begin{proof}
The maximum payment is upper-bounded by the maximum perceived
difference in value for any agent type and any two arms:
\begin{align*}
\bar{c} (s, \ell) & \leq 
\max_{\AgV} \left(  \max_{i} \AgV \cdot \ArmEV{t}{i}
                 - \min_{i'} \AgV \cdot \ArmEV{t}{i'} \right) \\
& = \max_{\AgV} \left( \max_{i}\AgV \cdot (\ErrV{t}{i}+\ArmV{i})
                    - \min_{i'}\AgV \cdot (\ErrV{t}{i'}+\ArmV{i'}) \right) \\
& \leq \max_{\AgV} \left(  \max_{i} \AgV \cdot \ArmV{i}
                        - \min_{i'} \AgV \cdot \ArmV{i'}
                        + \max_{i} \AgV \cdot \ErrV{t}{i}
                        - \min_{i'} \AgV \cdot \ErrV{t}{i'} \right) \\
& \leq \MAXR + 2 \Diam d \cdot g(s,\ell). 
\end{align*}
The final inequality used the definition of the envelope.
\end{proof}

\begin{lemma} \label{lem:envelope-probability}
For every $\ell \geq 2$, we have that
$\Prob{\SP \in \Env{\ell}} \leq 24 \ARMNUM d\exp(-1.8(\ell-1)^2)$. 
\end{lemma}

\begin{proof}
The proof idea is quite similar to the proof of
Lemma~\ref{lem:round-prob};
however, the specific form of the envelopes necessitates some subtle
changes in the specific forms of the bounds used.
For \SP to be in \Env{\ell}, by definition,
for at least one time $t$,
at least one coordinate of at least one arm's estimation error must
exceed $g(s(t,\SP),\ell-1)$.
For now, fix an arm $i$ and coordinate $j$.

Recall that $\NumPull{t}{i}(\SP) \geq s(t,\SP)$ is the number of times
that arm $i$ has been pulled by time $t$ under \SP.
We define the scaled estimation error
$S_{i,j}(t) := \frac{\NumPull{t}{i} \cdot \Err{t}{i}{j}}{2\sigma}$.
The advantage of $S_{i,j}(t)$ is that it is 
a summation of $1/2$-Gaussian random variables, 
so we will be able to apply Lemma~\ref{lem:ACI-inequality}.
First notice that we can express the desired probability as follows:

\begin{align*}
\Prob{|\Err{t}{i}{j}| \geq g(s(t,\SP), \ell-1)}
& = \Prob{S_{i,j}(t) \geq \frac{6\NumPull{t}{i}(\ell-1)}{s(t,\SP)^{2/5}}}\\
& \leq \; \Prob{S_{i,j}(t) \geq 6\NumPull{t}{i}^{3/5}(\ell-1)}. 
\end{align*}

We will show below that
$5 n^{3/5} \geq \sqrt{0.6n \log(\log_{1.1}(n)+1)}$
for all $n \geq 1$.
Applying this inequality and sublinearity of $\sqrt{\cdot}$,
we obtain that

\begin{align*}
6 \NumPull{t}{i}^{3/5}(\ell-1)
& \geq 5 \NumPull{t}{i}^{3/5} + \NumPull{t}{i}^{3/5}(\ell-1)  \\
& \geq \sqrt{0.6 \NumPull{t}{i} \log(\log_{1.1}(\NumPull{t}{i})+1)}
     + \sqrt{(\ell-1)^2 \NumPull{t}{i}}  \\
& \geq \sqrt{0.6\NumPull{t}{i}\log(\log_{1.1}(\NumPull{t}{i})+1)+(\ell-1)^2 \NumPull{t}{i}}
\; =: \; h(\NumPull{t}{i}).
\end{align*}

Let the random stopping time $\tau_{i,j}$ be the first value of
\NumPull{t}{i} (i.e., the first pull of arm $i$) such that 
$S_{i,j}(\tau_{i,j}) > h(\NumPull{t}{i})$.
$\tau_{i,j} = \infty$ means that the estimation error never exceeds 
$h(\NumPull{t}{i})$.
We apply Lemma~\ref{lem:ACI-inequality}
to $S_{i,j}(t)$, with a stopping time of $\tau_{i,j}$ and
with $b(n) = (\ell-1)^2$ \dkcomment{omit dependence on $n$ here?}.
We obtain that the probability that $\tau_{i,j} < \infty$ and 
$S_{i,j}(\tau_{i,j}) > h(\tau_{i,j})$ is at most
$12 \e^{-1.8 b(n)}$.
By definition of $\tau_{i,j}$, the event is the same as simply
$\tau_{i,j} < \infty$.
Applying the lemma again to $-S_{i,j}(\tau_{i,j})$,
and taking a union bound over both cases,
gives us that the probability that there is \emph{any} time $t$ with
$|S_{i,j}(\NumPull{t}{i})| > h (\NumPull{t}{i})$
is at most $24 \e^{-1.8 b(n)}$.
Hence, the probability that the error at \emph{any} time $t$ exceeds
$g(s(t,\SP), \ell-1)$ is bounded by 
\begin{align*}
\Prob{|\Err{t}{i}{j}| \geq g(s(t,\SP), \ell-1)}
& \leq 24 \e^{-1.8(\ell-1)^2}.
\end{align*}
Now, taking a union bound over all arms $i$ and
coordinates $j$ completes the proof.

It remains to show that
$5 n^{3/5} \geq \sqrt{0.6n \log(\log_{1.1}(n)+1)}$
for all $n \geq 1$.
By squaring the inequality and canceling out a factor $n$,
the statement is equivalent to showing that
$25 n^{1/5} \geq \log(\log_{1.1}(n)+1)$.
We will show the stronger statement that
$25 n^{1/5} \geq \log_{1.1}(n)+1$.  
To see this, notice that the derivative of the left-hand side is
always strictly larger than the derivative of the right-hand side,
so the difference between the sides is minimized at $n=1$,
where it is positive. 
\end{proof}

Next, we show that for any sample path \SP in the envelope \Env{\ell},
we can bound the total number of payments made in terms of $\ell$.
Define $h(\ell) := \max \left( \exp(\frac{2}{\MinProb}),
\left( \frac{48 \sigma \ell \TieDensity \Diam d}{\MinProb} \right)^{5/2}
\right)$.

\begin{lemma} \label{lem:envelope-payments}
Let \SP be a sample path in \Env{\ell}.
Then, under \SP, the algorithm makes payments at most for the first 
$h(\ell)$ phases. 
\end{lemma}

\begin{proof}
The proof is similar to that of Lemma~\ref{lem:numP}.
Fix a sample path $\SP \in \Env{\ell}$.
Consider a phase $s > h(\ell)$.

By definition of \Env{\ell}, the coordinate-wise estimation errors at
times $t$ in phase $s$ are
$|\Arm{i}{j} - \ArmE{t}{i}{j}| \leq g(s,\ell)$.
In particular, all agent typs \AgV whose preference for their best arm
is sufficiently clear
($\AgV \cdot \ArmV{\Best{\AgV}} - \AgV \cdot \ArmV{\Second{\AgV}}
> 2\Diam d g(s,\ell)$) will pull their best arm.

By Lemma~\ref{lem:sdelta}, the \emph{total} measure of agents (across
all arms) whose best and second-best arm differ in utility by at most
$2\Diam d \cdot g(s,\ell)$ is at most $2L \Diam d \cdot g(s,\ell)$.
Substituting the definition of $g$, because 
$s > h(\ell) \geq \left( \frac{48 \sigma \ell \TieDensity \Diam d}{\MinProb} \right)^{5/2}$,
the total measure of these agents is bounded by $\MinProb/2$.

By Assumption~\ref{A3}, at least a \MinProb fraction of agents have
$i$ as their best arm, so even with estimation errors, at least
$\MinProb - \MinProb/2 = \MinProb/2$ fraction will still prefer to
pull arm $i$.

Because $s > h(\ell) \geq \exp(\frac{2}{\MinProb})$, 
we get that $1/\log(s) \leq \MinProb/2$, so in phase $s$,
the algorithm will not incentivize any arms whose pulling probability
is more than $\MinProb/2$, which is all arms.
So no payments will be made in phase $s$.
\end{proof}


\begin{extraproof}{Theorem~\ref{rst:budget}}
We can write the total expected payment as
\begin{align*}
  \Expect{\sum_{t=1}^{\infty} \PayA{t}}
& = \sum_{\ell = 1}^{\infty} \sum_{\SP \in \Env{\ell}}
  \Prob{\SP} \cdot \sum_{t=1}^{\infty} \PayA{t}(\SP)
\; = \; \sum_{\ell = 1}^{\infty} \sum_{\SP \in \Env{\ell}}
  \Prob{\SP} \cdot \sum_{s=1}^{\infty} \sum_{t: s(t,\SP) = s} \PayA{t}(\SP).
\end{align*}

By Lemma~\ref{lem:envelope-payments},
the payments are 0 for $s > h(\ell)$,
and by Lemma~\ref{lem:sample-path-payment},
when $\SP \in \Env{\ell}$ and $s(t,\SP) = s$, we can bound
$\PayA{t}(\SP) \leq \MAXR + 2 \Diam d \cdot g(s,\ell)$.
Furthermore, in any one phase, because each arm is incentivized at
most once, there are at most \ARMNUM payments total.
Substituting these bounds, we obtain that

\begin{align*}
  \Expect{\sum_{t=1}^{\infty} \PayA{t}}
& \leq \sum_{\ell = 1}^{\infty} \sum_{\SP \in \Env{\ell}} \Prob{\SP} \cdot
  \sum_{s=1}^{h(\ell)} \ARMNUM \cdot (\MAXR + 2 \Diam d \cdot g(s,\ell))
\\ & = \ARMNUM \cdot
  \sum_{\ell = 1}^{\infty} \Prob{\SP \in \Env{\ell}} \cdot
  \sum_{s=1}^{h(\ell)} \left( \MAXR + \frac{24 \Diam d \sigma \ell}{s^{2/5}} \right).
\end{align*}

We now lower-bound $s^{2/5} \geq 1$, split off the term for $\ell=1$
and bound $\Prob{\SP \in \Env{1}} \leq 1$, and apply
Lemma~\ref{lem:envelope-probability} to the remaining
$\Prob{\SP \in \Env{\ell}}$ terms, to bound

\begin{align*}
\Expect{\sum_{t=1}^{\infty} \PayA{t}}
& \leq
\ARMNUM \cdot \sum_{s=1}^{h(1)} (\MAXR + 24 \Diam d \sigma)
+ \ARMNUM \cdot
  \sum_{\ell = 2}^{\infty} 24 \ARMNUM d\exp(-1.8(\ell-1)^2) \cdot
  \sum_{s=1}^{h(\ell)} (\MAXR + 24 \Diam d \sigma \ell)
\\ & \leq
\ARMNUM \cdot h(1) \cdot (\MAXR + 24 \Diam d \sigma)
+ 24 \ARMNUM^2 d \cdot \sum_{\ell = 2}^{\infty}
     \exp(-1.8(\ell-1)^2) \cdot h(\ell) \cdot (\MAXR + 24 \Diam d \sigma \ell).
\end{align*}
Because $h(\ell)$ and $(\MAXR + 24 \Diam d \sigma \ell)$ grow
polynomially in $\ell$, whereas $\exp(-1.8(\ell-1)^2)$ decreases
exponentially in $\ell$, the sum is dominated by its first term, and
the overall expected payment is bounded by

\begin{align*}
O \left(\ARMNUM^2 d \cdot (\MAXR + \Diam d \sigma) \cdot h(1) \right)
& = 
O \left(\ARMNUM^2 \MAXR \Diam^2 \TieDensity d^3 \sigma
\cdot \exp(\frac{2}{\MinProb}) \right).
\end{align*}
\dkcomment{$p^{-5/2}$ is dominated by $\exp(2/p)$.}
\pfcomment{the bound got tighter.  I didn't carry this forward anywhere where we report on / use the bound.}
\end{extraproof}


\section{Proof Sketch of Theorem~\ref{rst:discrete}}
\label{sec:discussion-proof1}

\begin{rtheorem}{Theorem}{\ref{rst:discrete}}
Under Assumption~\ref{a:discrete}, the Discrete-Preference Algorithm has expected payment \pfdelete{budget} bounded above by 
$\pfedit{O \left(\ARMNUM^2 /p \right)}$
and expected regret bounded above by \pfedit{$O(\ARMNUM / p)$}.
\end{rtheorem}

\begin{proof}
The proof of the expected payment bound is the same as the proof of Theorem~\ref{rst:budget}, except that we now define $h(\ell) := \max \left( \frac{2}{\MinProb},
\pfedit{
\left( \frac{24 \sigma \ell \Diam d}{\TieCutoff} \right)^{5/2}
}
\right).
$
%\pfedit{and $s_1 = \max(2,\frac{30\sigma^3}{x^3}, \frac2{\MinProb})$}.

To prove the bound on the expected regret, one can first prove tightened versions of Lemmas~\ref{lem:no-incentives} and \ref{lem:numP},
which replace the $\exp(2/\MinProb)$ term with \pfdelete{simply} $2/\MinProb$ \pfedit{and use $L=0$.}
In return, the length of phase $s$ is now bounded only by $\ARMNUM s$ instead of $\ARMNUM \log(s)$, \pfedit{and the expectetd number of times steps in which a payment is made is bounded by $O(N/p + N^2)$.}
Substituting these changes into the proof of Theorem~\ref{rst:regret}, we obtain the bound $O(R(N/p + N^2) + NR + NR) = O(N/p)$ since $p \le 1/N$.

\pfcomment{
The calculation I did for the bound on the expected payment is,
\begin{align*}
&O \left(\ARMNUM^2 d \cdot (\MAXR + \Diam d \sigma) \cdot h(1) \right) \\ 
&=O \left(\ARMNUM^2  d \cdot (\MAXR + \Diam d \sigma)\cdot
    \max\bigg\{\frac{2}{p},
    \left( \frac{\sigma \Diam d}{\TieCutoff} \right)^{5/2}
    \bigg\}
\right)
\end{align*}
}

\pfcomment{
The calculation I did for the bound on the regret is:

\begin{itemize}
\item regret from times when we pay is $RN(N + 1/p)$
\item regret from phases with $\gamma(s) >= \hat{z}$ is $O(NR D^6 d^6 \sigma^6 / \hat{z}^6)$
\item regret from pulls with small regret is 0
\item regret from pulls with big regret is $O(NR)$
\end{itemize}
}


\end{proof}

\section{Proof Sketch of Theorem~\ref{rst:known-p}}
\label{sec:discussion-proof2}


\begin{rtheorem}{Theorem}{\ref{rst:known-p}}
Under Assumption~\ref{a:known-p}, the Known-\MinProb Algorithm has an expected payment budget bounded above by 
    \bcedit{
\begin{align*}
    O\left(\ARMNUM^2 d\cdot(R+Dd\sigma)\cdot \max\bigg\{ \left(\frac{\sigma \Diam d}{\TieCutoff} \right)^{5/2},
        \left( \frac{\sigma \TieDensity \Diam d}{\MinProb} \right)^{5/2}\bigg\}\right)
\end{align*}
}
and an expected regret bounded above by
\bcedit{
\begin{align*}
O \left(\frac{\MAXR \ARMNUM \TieDensity^3 \Diam^3 d^3 \sigma^3}{\MinProb^3}
  + \frac{\MAXR \ARMNUM^2 d^3 \TieDensity^2 \Diam^2 \sigma^2}{\MinProb^2}
  + \frac{\ARMNUM\Diam^2 d^2\sigma^2 \TieDensity(\log(T))^3}{\MinProb}\right).
\end{align*}
}
\end{rtheorem}

\begin{proof}
  The proof of the expected payment bound follows Lemma~\ref{rst:budget} except for defining
    \bcedit{
    $h(\ell) := \max\big\{ \left(\frac{24\sigma \ell\Diam d}{\TieCutoff} \right)^{5/2},
        \left( \frac{48\sigma \ell\TieDensity \Diam d}{\MinProb} \right)^{5/2}\big\}$
    }
(without including the $\exp(2/\MinProb)$ term).

The proof of the expected regret bound first establishes tightened versions of Lemmas~\ref{lem:no-incentives} and \ref{lem:numP},
proving the following upper bound on the number of time steps in which a payment is made:
\begin{align}
O\left(\frac{\ARMNUM \TieDensity^3 \Diam^3 d^3 \sigma^3}{\MinProb^3}
  + \frac{\ARMNUM^2 d}{1 - \exp \left(
    \frac{-1.8 \cdot \MinProb^2}{256 \TieDensity^2 \Diam^2 d^2 \sigma^2}
  \right)} \right). \nonumber 
\end{align}
The proof of this result follows that of Lemma~\ref{lem:numP},
but the less aggressive incentivization allows us to define $\EvenLaterPhase = \max(2, \frac{30 \sigma^3}{x^3})$ since $\frac{1}{\log(s+s_0)} \leq \frac{\MinProb}{2}$ is true for all $s$.

Using this tighter bound on the number of incentiziations, and the fact that phases now last at most $\ARMNUM \log(s+s_0)$ steps in expectation, we can bound the regret in Equation~\ref{equ:small_regret_bound} by
%$71.11 \ARMNUM\Diam^2 d^2\sigma^2 \TieDensity\log(T)(\log(T)+1)\log(T+s_0)$
$O\left(\frac{\ARMNUM\Diam^2 d^2\sigma^2 \TieDensity(\log(T))^3}{\MinProb}\right)$.
\end{proof}


% \section{Proofs of Technical Lemmas}
\label{sec:n0-inequality-proof}
\label{sec:n1-inquality-proof}

In this section, we prove the two technical lemmas
\ref{lem:n0-inequality} and \ref{lem:n1-inequality}.
We restate both lemmas for convenience.

\dkcomment{I tightened these bounds a bit. Also, the proofs are quite
  a bit shorter now, so we might fold them into the main body?}

\begin{rtheorem}{Lemma}{\ref{lem:n0-inequality}}
For any $n \geq \LatePhase = \max (2, \frac{30 \sigma^3}{x^3})$, 
\begin{align*}
\sqrt{0.6 n \log (\log_{1.1}(n) + 1)} & \leq \frac{n x}{4 \sigma}. 
\end{align*}
\end{rtheorem}

\begin{proof}
By squaring the statement and rearranging, it is equivalent to show that

\begin{align*}
\frac{n}{\log(\log_{1.1}(n)+1)}
  & \geq \frac{9.6\sigma^2}{x^2}.
\end{align*}

A numerical calculation and derivative test shows that
$\frac{n}{\log(\log_{1.1}(n)+1)} \geq n^{2/3}$ for all $n \geq 2$,
and for $n \geq \frac{30 \sigma^3}{x^3}$, we get that
$n^{2/3} \geq \frac{9.6 \sigma^2}{x^2}$.
\end{proof}


\begin{rtheorem}{Lemma}{\ref{lem:n1-inequality}}
For any $T \geq 2$ and $1 \leq n \leq T$, we have
\begin{align*}
2\sqrt{\frac{2}{1.8} n \log(T)}
& \geq \sqrt{0.6 n \log(\log_{1.1}(n)+1) + \frac{2}{1.8} n \log(T)}.
\end{align*}
\dkcomment{This also holds for $n=1$, and I improved the constant.}
\end{rtheorem}

\begin{proof}
By squaring the desired inequality and canceling out common terms, the
statement to prove is equivalent to
$\frac{50}{9} \log T \geq \log(\log_{1.1}(n) + 1)$.
It is sufficient to show that
$\frac{50}{9} \log T \geq \log(\log_{1.1}(T) + 1)$,
which can be verified by numerical calculation for $T=2$ and a
derivative test.
\end{proof}


\end{document}
