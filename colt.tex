\documentclass[twoside,11pt]{article}
\usepackage[anon]{jmlr2e}


% Any additional packages needed should be included after jmlr2e.
% Note that jmlr2e.sty includes epsfig, amssymb, natbib and graphicx,
% and defines many common macros, such as 'proof' and 'example'.
%
% It also sets the bibliographystyle to plainnat; for more information on
% natbib citation styles, see the natbib documentation, a copy of which
% is archived at http://www.jmlr.org/format/natbib.pdf


\usepackage[]{algorithm}
\usepackage[noend]{algorithmic}

%\usepackage[utf8]{inputenc} % allow utf-8 input
%\usepackage[T1]{fontenc}    % use 8-bit T1 fonts
%\usepackage{microtype}      % microtypography
%\usepackage{xr} 

\usepackage{url}            % simple URL typesetting
\usepackage{amsfonts}       % blackboard math symbols
\usepackage{amsmath}
\usepackage{nicefrac}       % compact symbols for 1/2, etc.
%\usepackage{amsthm}
%\usepackage{mathtools}

%\usepackage{comment}
%\usepackage{subcaption}
%\usepackage{xcolor}
%\usepackage{float}
\usepackage{bm}
\usepackage{bbm}
\usepackage{ifthen}
\usepackage{xspace}

\usepackage{color-edits}
% home-made package by David that provides the following macros:
% 1. \pfedit{}: prints argument in blue
% 2. \pfcomment{}: prints argument in blue in [Peter: #1]
% 3. \pfmargincomment{}: prints argument in blue in the margin as
% [Peter: #1]
% 4. \pfdelete{}: instead of text, marks a note in the margin in blue that says
% "Peter deleted here"
% Package has option [suppress], which gets rid of all color and all comments
% and [showdeletions], which shows deleted text in color/strikeout.
\addauthor[Peter]{pf}{blue}
\addauthor[Bangrui]{bc}{green}
\addauthor[David]{dk}{red}

%\newcommand{\bccomment}[1]{{\color{blue}BC: #1}}
%\newcommand{\pfcomment}[1]{{\color{blue}PF: #1}}

\providecommand{\SET}[1]{\ensuremath{\{ #1 \}}\xspace}
\providecommand{\Abs}[1]{\ensuremath{| #1 |}\xspace}
\providecommand{\Set}[2]{\ensuremath{\SET{#1 \mid #2}}\xspace}

\providecommand{\PROB}{\ensuremath{{\rm Prob}}\xspace}
\providecommand{\Prob}[2][]{\ensuremath{%
\ifthenelse{\equal{#1}{}}{\PROB[#2]}{\PROB_{#1}[#2]}}\xspace}
\providecommand{\ProbC}[3][]{\Prob[#1]{#2\;|\;#3}}
\providecommand{\Expect}[2][]{\ensuremath{%
\ifthenelse{\equal{#1}{}}{\mathbb{E}}{\mathbb{E}_{#1}}%
\left[#2\right]}\xspace}
\providecommand{\ExpectC}[3][]{\Expect[#1]{#2\;|\;#3}}

\newcommand{\e}{\mathrm{e}}
\providecommand{\Kth}[1]{\ensuremath{{#1}^{\rm th}}}

\def\QED{{\phantom{x}} \hfill \ensuremath{\rule{1.3ex}{1.3ex}}}

\newcommand{\extraproof}[1]{\rm \trivlist \item[\hskip \labelsep{\bf Proof of #1. }]}
\def\endextraproof{\QED \endtrivlist}

\def\emptyproof{\rm \trivlist \item[\hskip \labelsep{\bf Proof. }]}
\def\endemptyproof{\endtrivlist}


\newcommand{\argmax}{\mathop{\mathrm{argmax}}}
\newcommand\floor[1]{\lfloor#1\rfloor}
\newcommand\ceil[1]{\lceil#1\rceil}
\newcommand{\fracpartial}[2]{\frac{\partial #1}{\partial  #2}}
\newcommand{\R}{\mathbb{R}} % real numbers

\newtheorem{assumption}{Assumption}

% Definitions of handy macros can go here

\newcommand{\vc}[1]{\bm{#1}} % vectors

\newcommand{\Normal}[2]{\ensuremath{{\mathcal N}(#1,#2)}\xspace}
% normal distribution

\newcommand{\dataset}{{\cal D}} % used anywhere?

%%%%%%%%%%%%%%%%%%%%%%%%%%%%%%%%%%%%
% Agreed upon symbols without macros
%%%%%%%%%%%%%%%%%%%%%%%%%%%%%%%%%%%%

% dimension for attribute vector: d
% stopping time: \tau
% phase of the algorithm: s


% To define and add in appropriate spots

% \Delta for utility difference
% \bar{\zeta} for error in utility difference
% q_{\delta} for mass of users


%%%%%%%%%%%%%%%%%%%%%%%%%%%%%%%%%%%%%
% Stuff related to Arms and Arm Pulls
%%%%%%%%%%%%%%%%%%%%%%%%%%%%%%%%%%%%%

\newcommand{\ARMNUM}{\ensuremath{N}\xspace} % number of arms

\newcommand{\ArmV}[1]{\ensuremath{\vc{\mu}_{#1}}\xspace}
% vector for location of an arm

\newcommand{\Arm}[2]{\ensuremath{\mu_{#1}^{(#2)}}\xspace}
% individual entries of the ith arm's location vector
% Argument 1: arm index
% Argument 2: coordinate

\newcommand{\NoiseV}[1][]{\ensuremath{\vc{\zeta}_{#1}}\xspace}
% vector of noise added to arm

\newcommand{\Noise}[2][]{\ensuremath{%
\ifthenelse{\equal{#1}{}}{\zeta_{#2}}{\zeta_{#1,#2}}\xspace}}
% individual entries of noise vector

\newcommand{\ObsV}[1]{\ensuremath{\vc{y}_{#1}}\xspace}
% observation of an arm: location plus noise

\newcommand{\ArmEV}[2]{\ensuremath{\vc{\hat{\mu}}_{#1,#2}}\xspace} 
% empirical estimator of the arm location based on samples
% Argument 1: time step
% Argument 2: arm

\newcommand{\ArmE}[3]{\ensuremath{\hat{\mu}_{#1,#2}^{(#3)}}\xspace}
% coordinates of empirical estimater of the arm location
% Argument 1: time step
% Argument 2: arm
% Argument 3: coordinate/attribute

\newcommand{\ErrV}[2]{\ensuremath{\vc{\epsilon}_{#1,#2}}\xspace} 
% error in the empirical estimator of the arm location
% Argument 1: time step
% Argument 2: arm

\newcommand{\Err}[3]{\ensuremath{\epsilon_{#1,#2}^{(#3)}}\xspace}
% coordinates of error in the empirical estimater of the arm location
% Argument 1: time step
% Argument 2: arm
% Argument 3: coordinate/attribute

\newcommand{\PullProb}[2]{\ensuremath{\phi_{#1,#2}}\xspace}
% probability that a random myopic agent will pull the arm
% Argument 1: time step
% Argument 2: arm
% \phi is a placeholder for now.


%%%%%%%%%%%%%%%%%%%%%%%%%
% Stuff related to agents
%%%%%%%%%%%%%%%%%%%%%%%%%

\newcommand{\AgV}{\ensuremath{\vc{\theta}}\xspace}
% vector for location of an agent, without subscript

\newcommand{\Ag}[1]{\ensuremath{\theta_{#1}}\xspace}
% individual entries of an agent location vector without time step

\newcommand{\AgentV}[1]{\ensuremath{\AgV_{#1}}\xspace}
% vector for location of an agent, subscripted by time of arrival

\newcommand{\Agent}[2]{\ensuremath{\theta_{#1,#2}}\xspace}
% individual entries of the agent location vector

\newcommand{\Best}[1]{\ensuremath{B_{#1}}\xspace}
% Best arm for a given agent

\newcommand{\Second}[1]{\ensuremath{B'_{#1}}\xspace}
% Second-best arm for a given agent

\newcommand{\FirstTwo}[2]{\ensuremath{\Omega_{#1,#2}}\xspace}
% Set of agents who have i as their first choice and i' as their second



%%%%%%%%%%%%%%%%%%%%%%%%%%%%%%%%%%%%%
% Stuff related to agent distribution
%%%%%%%%%%%%%%%%%%%%%%%%%%%%%%%%%%%%%

\newcommand{\AgentDist}{\ensuremath{f}\xspace}
% distribution of locations for agents

\newcommand{\Diam}{\ensuremath{D}\xspace}
% "diameter" of support of agent distribution, [0,D]^d

\newcommand{\MinProb}{\ensuremath{p}\xspace}
% minimum probability of support for any arm

\newcommand{\TieDensity}{\ensuremath{L}\xspace}
% upper bound on the derivative of the density of near-ties

\newcommand{\AlmostTied}[1]{\ensuremath{q(#1)}\xspace}
% probability mass of almost tied agents across all arms

%\newcommand{\ClearPref}[1]{\ensuremath{p(#1)}\xspace}
% probability mass of agents with a clear preference for their first choice,
% minimized over all arms.
% not used anywhere.


%%%%%%%%%%%%%%%%%%%%%%%%%%%
% Stuff related to policies
%%%%%%%%%%%%%%%%%%%%%%%%%%%

\newcommand{\POLICY}{\ensuremath{\mathcal A}\xspace} % algorithm/policy
\newcommand{\Pay}[2]{\ensuremath{c_{#1,#2}}\xspace}
% payment offered for arms:
% Argument 1: time step
% Argument 2: arm

\newcommand{\PayA}[1]{\ensuremath{c_{#1}}\xspace}
% payment actually made in step t, i.e., c_{t,\Pull{t}}

\newcommand{\TotalPay}[1]{\ensuremath{C_{#1}}\xspace}
% cumulative payment made up to and including step t.

\newcommand{\Pull}[1]{\ensuremath{i_{#1}}\xspace}
% arm that was actually pulled at time step t

\newcommand{\NumPull}[2]{\ensuremath{m_{#1,#2}}\xspace}
% number of pulls of arm i up to time t:
% Argument 1: time step
% Argument 2: arm

\newcommand{\Regret}[1]{\ensuremath{r_{#1}}\xspace}
% regret incurred in time step t

\newcommand{\TotalRegret}[1]{\ensuremath{R_{#1}}\xspace}
% total regret incurred up to and including time step t

\newcommand{\MAXR}{\ensuremath{R}\xspace}
% upper bound on the maximum regret in any one round

\newcommand{\History}[1]{H_{#1}}
% History of pulls, payments, and observations

%%%%%%%%%%%%%%%%%%%%%%%%%%%%%%%%%%
% Stuff related to specific bounds
%%%%%%%%%%%%%%%%%%%%%%%%%%%%%%%%%%

\newcommand{\LatePhase}{\ensuremath{s_0}\xspace}
% lower bound after which concentration lemma holds

\newcommand{\EvenLaterPhase}{\ensuremath{s_1}\xspace}
% lower bound after which payments become unlikely



%%%%%%%%%%%%%%%%%%%%%%%%%%%%%%%%
% Stuff related to payment proof
%%%%%%%%%%%%%%%%%%%%%%%%%%%%%%%%

\newcommand{\SP}{\ensuremath{\omega}\xspace}
% sample path

\newcommand{\Env}[1]{\ensuremath{{\mathcal L}_{#1}}\xspace}
% envelope
% Argument: level of envelope


% Heading arguments are {volume}{year}{pages}{date submitted}{date published}{paper id}{author-full-names}

\jmlrheading{1}{2000}{1-48}{4/00}{10/00}{meila00a}{Marina Meil\u{a} and Michael I. Jordan}

% Short headings should be running head and authors last names

\ShortHeadings{Incentivizing Exploration with Heterogeneous Utilities}{Chen, Frazier and Kempe}
\firstpageno{1}

\begin{document}

\title{Incentivizing Exploration by Heterogeneous Users}
% {Incentivizing Exploration with Heterogeneous Utilities}

\author{\name Bangrui Chen \email bc496@cornell.edu \\
\name Peter I.\ Frazier \email pf98@cornell.edu \\
\addr Operations Research and Information Engineering\\
Cornell University\\
New York, NY 14850, USA
\AND
\name David Kempe \email david.m.kempe@gmail.com  \\
\addr Department of Computer Science\\
University of Southern California\\
Los Angeles, CA 90089, USA}

\editor{}

\maketitle

\begin{abstract}
sdafsd
\end{abstract}

\begin{keywords}
Incentivizing Exploration
\end{keywords}

     

\section{Introduction}

Many websites and apps are designed to facilitate joint discovery,
sharing, and recommendations of content.
Such sites include news, photo, and video sharing sites,
sites to review restaurants, hotels, or travel experiences,
online stores at which users write reviews (such as Amazon),
and citizen science projects
(such as eBird \citep{sullivan2009ebird,xue-ebird} or Galaxy Zoo \citep{lintott-galaxy-zoo}).
By learning from the experiences of other users, individuals can
improve their own experience \citep{schmit2017human}.

Viewed more abstractly, users jointly explore a space of
many options (products, news stories, photos, birdwatching sites,
patches of sky to train their telescopes on, $\ldots$),
with the implicit goal of identifying the ``best'' ones.
Therefore, such scenarios can be fruitfully modeled in a bandit
learning framework.
However, contrary to standard bandit settings, the utilities of the
decision makers (the users) are not aligned with the overall utility.
Societally (i.e., in a suitable aggregate over all users),
it would be desirable to engage in considerable exploration of
different options, so as to provide higher rewards for a large number
of future users.
However, individual users only interact with the site a limited number
of times, and therefore have little incentive for exploration.
A particularly clean model is obtained when each user interacts with the
site only once, and hence has no intrinsic utility for exploration. 
This model has been the subject of prior work, and forms the basis of
the present submission.

To effect an outcome close to societally optimal,
it is necessary to provide exploration incentives to the individual users.
This was noted in two recent lines of work:
\citet{kremer2014implementing}
and \citet{mansour2015bayesian,mansour2016bayesian}
assume that the site (also called the \emph{principal}) has an
informational advantage in being the only one to observe the results
of past arm pulls.
(Such an assumption applies, for instance, to route recommendations in
driving.)
The principal can exploit her%
\footnote{We use male pronouns to refer to users and female pronouns
  to refer to the principal.}
advantage and make recommendations to the individual agents that 
are in their best interest to follow.
\citet{frazier2014incentivizing} and 
\citet{han2015incentivizing} instead assume that the results of all
past arm pulls are publicly observable
(such as reviews on an online retail site).
They instead suppose the principal can offer payments to users as
a reward for pulling particular arms.
We follow the model of \citet{frazier2014incentivizing} and
\citet{han2015incentivizing} 
and consider a multi-armed bandit model in which the principal can
offer payments to the users for pulling particular arms.

Past work has assumed that users are homogeneous, i.e., the expected
reward a user derives from an arm is the same for all users.%
\footnote{\citet{han2015incentivizing} assume that users are
heterogeneous in their tradeoff between utility derived from arm pulls
and utility derived from the principal's payment.}
In reality, users have different preferences, e.g., gastronomic,
political, aesthetic, practical, etc.
Indeed, the websites and mobile apps most widely used for joint discovery,
sharing, and recommendation of content tend to concern products and items 
with large amounts of heterogeneity in preferences (movies, restaurants,
videos, travel experiences), and not items which have a universally
agreed-on best order.
This is perhaps because regimes with heterogeneous preferences are the
ones where discovery of the best items is the most difficult for people, 
and thus where online
platforms tend to provide the greatest value.  Thus, we see an appropriate
accounting for heterogeneity as critical to an understanding of incentivizing
exploration in online communities.

Heterogeneity presents both a challenge and an opportunity.
On the one hand, unobserved heterogeneity hides critical
information about an agent's preferences from the principal.
On the other hand, heterogeneity also presents her an opportunity,
through the possibility of ``free exploration.''
Even when left unincentivized, agents will
play a variety of arms, revealing information about these arms' attributes.
This stands in sharp contrast to the case of homogeneous preferences,
where unincentivized agents will herd onto a single apparently best arm;
thus, effecting essentially any exploration at all requires incentives.

In order to take advantage of unobserved heterogeneity,
the principal has to give up some control, allowing the agent to
reveal his preferences through action, rather than obscuring them with
incentives.
However, she cannot give up control completely,
and must use incentives to force agents to explore against their
preferences, for the greater good. 

With these challenges and opportunities in mind, our goal is to understand the
impact of user heterogeneity on the principal's ability to achieve
high social utility with low incentive payments,
and on the best approaches for doing so.
We wish to understand whether incentivizing exploration with
heterogeneous preferences  is ``harder'' or ``easier'' than with
homogeneous ones,
and to understand how exploration strategies that do well in the
heterogeneous preference setting differ from those in the homogeneous one.

Toward that end, we model our setting as follows.
(We describe our model at a high level here,
with formal definitions given in Section~\ref{sec:prob}.)
Arms and users (or \emph{agents}) are characterized by payoff-relevant
\emph{attribute} (or \emph{feature}) vectors.
Arms' attributes are a priori unknown, and
  agents' attributes are drawn from a known distribution.
An agent's reward from pulling an arm is the inner product of his
vector with the arm's vector (plus noise).
When an arm is pulled, a noisy version of its attribute vector is
observed by everyone.
Agents are myopic and will pull the arm whose expected attribute
vector (based on past noisy observations) maximizes their reward.
The principal can incentivize agents to pull particular arms by
offering rewards for the specific arm.
The principal's goal is to keep the cumulative regret across all
agents small, while incurring only small total payments.
Our main theorem can be stated informally as follows:

\begin{theorem} \label{thm:main-intro}
Assume that for each arm, at least a constant fraction of the
population likes this arm best.
If furthermore, the density of ties in agent preferences between arms
vanishes (in a sense made precise in Section~\ref{subsec:discrete}),
there is a policy that achieves constant
expected regret $C$ and constant%
\footnote{The constants $C,C'$ depend on the number of arms and the
  smallest fraction preferring any one arm.} expected payment $C'$.
If near-ties have non-zero density,
then the regret is bounded by $O(C + C'' \log^3(T))$,
while the expected payment is still bounded by $C'$;
here, $C''$ depends on the ``tie density.''
\end{theorem}

The policy achieving the result of Theorem~\ref{thm:main-intro} is
quite simple: it mostly lets agents exploit arms, but incentivizes
them to explore when arms appear unlikely to be pulled without incentives.
It is presented in detail in Section~\ref{sec:ub}.


\section{Preliminaries}
\label{sec:prob}

We consider a multi-armed bandit setting with \ARMNUM arms.
Arm payoffs are determined by $d$ \emph{attributes} or \emph{features};
hence, arms can be identified with vectors $\ArmV{i} \in \R^d$.
The \ArmV{i} are (adversarially) fixed, and unknown to the agents
or the principal.
Whenever arm $i$ is pulled, its current utility-relevant features are
determined as $\ArmV{i} + \NoiseV$, where \NoiseV is a mean-zero independent Gaussian%
\footnote{For simplicity of notation, we assume that the variance
  $\sigma^2$ is uniform across time steps.
  This is not material for the analysis.
\dkedit{In fact, the analysis extends straightforwardly to any
  mean-zero sub-Gaussian noise.} \dkcomment{Is this true?}}
noise vector $\NoiseV \sim \Normal{\vc{0}}{\sigma^2 I_{d}}$.
Here, $I_d$ denotes the $d \times d$ identity matrix.

At each time $t$, a new user (or \emph{agent}) arrives,
whose feature vector (which we also call his \emph{type})
$\AgentV{t} \in \R^d$ is drawn from a known distribution \AgentDist.
Depending on context, we will identify agents with their arrival time
$t$ or their type \AgentV{t}.
When agent $t$ pulls arm $\Pull{t} := i$,
he and all future agents observe a vector
$\ObsV{t} = \ArmV{i} + \NoiseV[t]$ for arm $i$,
and his reward is $\AgentV{t} \cdot \ObsV{t}$,
i.e., agents have linear preferences.%
\footnote{\dkedit{Some of our results generalize to other forms of
    agent rewards.}}

For each time $t$ and arm $i$, let \NumPull{t}{i} be the number of
times that arm $i$ has been pulled (strictly) before time $t$.
An agent at time $t$ estimates arm $i$'s attribute vector as the
average of vectors observed during past pulls of the arm:
$\ArmEV{t}{i} = \frac{1}{\NumPull{t}{i}+1} \cdot
(\ArmEV{0}{i} + \sum_{t'<t: \Pull{t'} = i} \ObsV{t'})$;
here, \ArmEV{i}{0} is a single draw $\ArmV{i} + \NoiseV$ for arm $i$.
(In other words, we assume that each arm is pulled once for free at time 0.)

Since each user only pulls an arm once, users are \emph{myopic}:
in the absence of incentives, user $t$ will pull an arm from
$\argmax_i \AgentV{t} \cdot \ArmEV{t}{i}$.
To incentivize users to explore more, the principal can offer
\emph{payments} $\Pay{t}{i}$ to user $t$ for pulling arm $i$.
Then, user $t$ will pull an\footnote{We assume that ties are broken in
  favor of an arm with largest payment \Pay{t}{i}.}
% ; this assumption is
%   only for notational convenience, and can of course be avoided by
%   raising payments infinitesimally.
%   \pfcomment{In the finite preference setting this is clear, but it's a little bit delicate when there are continuum preferences, since regardless of the payment we offer there may always be support within our preference distribution for a $\theta$ that causes a tie.  In this continuum setting it should instead be possible to perturb infinitessimally so that we avoid ties with probability 1.}  
% }
arm $i$ maximizing $\argmax_i (\Pay{t}{i} + \AgentV{t} \cdot \ArmEV{t}{i})$.
The principal cannot observe \AgentV{t},
and only knows the distribution \AgentDist from which it is drawn.
Her goal is to achieve a good tradeoff between the cumulative
\emph{regret} experienced by all users up to time $T$,
and the total \emph{payment} she makes to the users.

We define the regret at time $t$ as
$\Regret{t} = (\max_{i} \AgentV{t} \cdot \ArmV{i}) - \AgentV{t} \cdot \ArmV{\Pull{t}}$,
and the cumulative regret up to time $T$ as
$\TotalRegret{T} = \sum_{t=1}^{T} \Regret{t}$.
Similarly, $\PayA{t} = \Pay{t}{\Pull{t}}$ is the actual incentive
payment at time $t$,
and the cumulative payment up to time $T$ is
$\TotalPay{T} = \sum_{t=1}^{T} \PayA{t}$.
More formally, the principal's goal is to find a policy
\POLICY for offering payments under which both the cumulative expected
regret
$\Expect{\TotalRegret{T}}$ the cumulative expected payment
$\Expect{\TotalPay{T}}$ are small.
\dkcomment{Deleted the \POLICY subscript for the expectation, as the
  expectation is also over the entire random history.}

To support the formulation of our results and the analysis,
we define the following additional notation.
We let
$\Best{\AgV} \in \argmax_i \AgV \cdot \ArmV{i}$
and
$ \Second{\AgV} \in \argmax_{i \neq \Best{\AgV}} \AgV \cdot \ArmV{i}$
denote the (indices of) the best and second-best arms for an agent
with attribute vector \AgV,
\dkedit{breaking ties arbitrarily (but consistently)}.

% \dkdelete{This behavior may be recovered if agents are Bayesian and
%   share a common non-informative prior distribution that is constant
%   over $\mathbb{R}^m$ and know $\sigma^2$.  In this case, the
%   posterior distribution on $u_{i}$ at time $t$ is multivariate normal
%   with mean $u_{i,t}$, and the expected value of $\theta_t \cdot u_i$
%   under this posterior conditioned on $\theta_t$ is $\theta_t \cdot
%   u_{i,t}$ (see Equation 2.13 in Section 2.5, \cite{Ge04}).
%   Alternatively, one may simply take our assumption that agents use
%   the average as their estimate of an attribute value directly without
%   such a Bayesian justification.}
% \dkcomment{I wonder what's the best way to discuss this.}

\subsection{Properties of \AgentDist}
Our algorithms rely on a few assumptions about the agent distribution
\AgentDist.

\begin{assumption}[Compact Support] \label{A2}
\AgentDist has a compact support set contained in $[0,\Diam]^d$.
\end{assumption}

Let $\MinProb = \min_{i} \Prob[\AgentDist]{\Set{\AgV}{\Best{\AgV} = i}}$
denote the minimum (over all arms) fraction of users that prefer any
particular arm.

\begin{assumption}[Every arm is someone's best] \label{A3}
Each arm $i$ has a strictly positive proportion of users for whom $i$
is the best arm; that is, $\MinProb > 0$.
\end{assumption}

%Let $\FirstTwo{i}{i'} = \Set{\AgV}{\Best{\AgV} = i, \Second{\AgV} = i'}$
%be the set of agent attribute vectors whose best arm is $i$ and
%second-best arm is $i'$.
% Let $F_{i,i'}$ be the cumulative density function
% (or cumulative mass function if $\AgentDist(\cdot)$ is a discrete distribution)
% $F_{i,i'}(z) = \Prob[\AgV \sim \AgentDist]{\AgV \in \FirstTwo{i}{i'}
%   \mbox{ and } (\ArmV{i}-\ArmV{i'}) \cdot \AgV \leq z}$
% of $(\ArmV{i}-\ArmV{i'}) \cdot \AgV$,
% on $\AgV \in \Omega_{i,i^{'}}$.
% In words, $F_{i,i'}$ is the distribution of the \emph{strength} of the
% preference of a random agent for arm $i$ over arm $i'$

% $\Best{\AgV} \in \argmax_i \AgV \cdot \ArmV{i}$
% and
% $ \Second{\AgV} \in \argmax_{i \neq \Best{\AgV}} \AgV \cdot \ArmV{i}$

\dkedit{Let \AlmostTied{z} be the cumulative density function
(or cumulative mass function if $\AgentDist(\cdot)$ is a discrete distribution)
$\AlmostTied{z} = \Prob[\AgV \sim \AgentDist]{%
  (\ArmV{\Best{\AgV}}-\ArmV{\Second{\AgV}}) \cdot \AgV \leq z}$.
In words, \AlmostTied{z} is the CDF of the \emph{strength} of the
preference of a random agent for his best arm over his second-best arm.}

\begin{assumption}[Not too many near-ties] \label{A1}
Near-ties have vanishing probability; 
that is, there exists a constant \TieDensity such that for all $z \in \R^+$,
%for all pairs $i,i'$ of arms and 
$\AlmostTied{z} \leq \TieDensity \cdot z$.
\end{assumption}

\dkedit{\paragraph{Role of parameters:}
Our problem setting is characterized by a fairly large number of
parameters.
Of these, we consider \ARMNUM, $\MinProb \leq 1/\ARMNUM$ and $T$ to be
the key parameters, while $\sigma, \Diam, d, \TieDensity$ should be
considered constant.
We will keep track of these constants throughout most of our proofs,
but report final results in terms of only the three key parameters
(except where we illustrate a specific point).}


\dkcomment{Stuff to discussion in conclusions:
\begin{enumerate}
\item Arms that are noone's best.
\item Other utilities besides inner products.
\item Other noise distributions.
\item Better bounds.
\end{enumerate}
}

\section{Overview of Results and Discussion}

Our main algorithm is presented as
Algorithm~\ref{alg:basic-incentivizing}
in Section~\ref{sec:ub}.
Our main result is the following pair of theorems%
\footnote{Recall that we omit the dependence on parameters other than
  \ARMNUM, \MinProb, $T$ unless making a particular point.},
analyzing the payments and regret of the algorithm.

\begin{theorem} \label{rst:budget}
The expected total payment of
Algorithm~\ref{alg:basic-incentivizing} is at most
$O \left(\ARMNUM^2 \cdot \e^{2/\MinProb} \right)$.
\end{theorem}

\begin{theorem} \label{rst:regret}
For any time horizon $T$, the expected cumulative regret for
Algorithm~\ref{alg:basic-incentivizing} up to time $T$ is bounded
above by 
$O \left(\ARMNUM \cdot \e^{2/\MinProb} + \TieDensity \ARMNUM \log^3(T) \right)$.
\end{theorem}

When $\TieDensity = 0$, the bound of Theorem~\ref{rst:regret} is
constant in $T$;
thus, the algorithm achieves constant regret using constant
\dkedit{expected payments}. 
As discussed in Section~\ref{sec:prob},
the case $\TieDensity = 0$ arises, for instance, for
discrete agent distributions.
In fact, in the case of a discrete agent distribution,
it is possible to modify the algorithm to also
reduce the dependence on \MinProb from exponential to polynomial.
Theorem~\ref{rst:discrete} (given later) states that the modified
algorithm achieves expected regret
$O \left(\frac{\ARMNUM}{\MinProb^4} \right)$
with expected payments of
$O \left(\frac{\ARMNUM^2}{\MinProb^{5/2}} \right)$.

The fact that constant regret can be achieved with constant payment
(independent of $T$) when $\TieDensity = 0$ suggests aiming for a
constant bound more generally, i.e., for $\TieDensity > 0$.
That such a bound is unachievable is shown in Appendix~\ref{sec:lb},
where we show a lower bound of $\Omega(\log(T))$ on
the expected regret of any algorithm.
The instance is simple: it has two arms, one \pfedit{with}
attributes; \dkcomment{Should that be ``with \emph{known}
  attributes''? I don't understand this edit.}
in addition, one draw from the other arm is observed in
each step $t$ even when it is not pulled.
While the probability of pulling the wrong arm decreases over time, it
does not do so fast enough, causing the stated regret.

The exponential dependence on $1/\MinProb$ implies an exponential
dependence on \ARMNUM (because $\MinProb \leq 1/\ARMNUM$).
This exponential dependence arises from a need to continue to
incentivize arm pulls to ensure that nearly tied agents learn their
best arms quickly.
Aside from the assumption that $\TieDensity = 0$, another assumption
allows us to eliminate this exponential dependence.
Namely, when \MinProb (or a lower bound on it) is known ahead of time,
the algorithm can be modified to incentivize arms less aggressively.
As shown in Theorem~\ref{rst:known-p},
the modified algorithm has expected regret at most
$O \left(\frac{\ARMNUM}{\MinProb^3}
+\frac{\ARMNUM \TieDensity \log^3(T)}{\MinProb} \right)$,
with expected payments of at most
$O \left(\frac{\ARMNUM^2}{\MinProb^{5/2}} \right)$.

We compare these bounds to those for standard bandits,
focusing on the dependence on $T$.  
The standard bandit setting is the case when the agent types \AgV
are concentrated on a single point, and agents pull arms at the
principal's direction without requiring payment.
(This setting violates our Assumption~\ref{A3},
so our bounds do not apply to it.)
Then, the payment is $0$ and the expected regret scales as
$\Theta(\log(T))$ \cite[Theorem 2.1]{bubeck2012regret}.

Our algorithm's payment is constant in $T$,
while its regret is $O(\log^3(T))$ in general with a lower bound of
$\Omega(\log(T))$;
when preferences are discrete, our algorithm's regret is
constant in $T$.
Thus, viewed solely in terms of the dependence on $T$,
the best performance achievable seems comparable to that in a standard
multi-armed bandit problem;
but when preferences are discrete, the constant regret
surpasses the $\Theta(\log(T))$ achievable in the standard multi-armed
bandit \dkedit{setting}.
This may seem surprising, because the principal in our setting has
both less control and less information than in the standard bandit setting.
The result arises because heterogeneity in preferences provides
free exploration, and allows all of the arms to be pulled infinitely
often without incurring regret once estimates are accurate enough.

While heterogeneity in preferences enables this free exploration,
heterogeneity alone is not always sufficient for enabling performance
improvements compared to the standard bandit setting.
Indeed, suppose that agents are still heterogeneous,
but the principal pulls arms directly.
Unless the principal can also observe the agents' types,
she will be unable to correctly choose each agent's preferred arm,
even with infinite exploration of arm attributes.
Regret will then grow as $\Omega(T)$. 

Thus, reaping the benefits of (unobserved) heterogeneous preferences
requires the principal to give up direct control of the arms,
providing agents the autonomy they need to express their private
information about their own preferences.
Our results show that simple arm-based incentives are sufficient
to overcome the apparent challenges created by this abdication of
control.



\section{Algorithm and Upper Bound}
\label{sec:ub}

In this section, we propose a simple policy that mostly exploits, and occasionally incentivizes exploration when the probability of an arm would be pulled by all agent types below a time-varying threshold given the current posterior. We prove that with the help of heterogeneous preferences, we can get a certain amount of exploration for free via heterogeneity. 

\subsection{Our Algorithm}
Our algorithm incentivizes pulling an arm $i$ at a time $t$ in round $n$ if and only if both of the following criteria are met:
\begin{itemize}
\item the probability of pulling arm $i$ would be below $n^{-1}$ without incentives; 
\item arm $i$ has not been played previously in the current round.
\end{itemize}
Ties are broken randomly.  This algorithm does not need to know the horizon $T$ in advance. 

If our algorithm decides to incentivize an arm $i$, it uses the ``pay whatever it takes'' strategy in which the payment offered is $\max_{\theta,j} \theta \cdot (u_{j,t} - u_{i,t})$. This maximum over $\theta$ is taken over the support of $F$, which we recall is assumed compact.  (We use this ``pay whatever it takes'' strategy for its simplicity, and in Section~\ref{sec:pi} we provide an alternate and smaller incentive payment that achieves the same payment budget bound and regret bound). 

We describe our algorithm in detail as follows:


\begin{algorithm}
\caption{Algorithm: Incentivizing Exploration}
\label{Alg1}
\begin{algorithmic}
\STATE Set n = 1 to denote the round number; Let $V=\emptyset$ be the set of arms that were pulled in the current round;
\FOR{ $t = 1, 2, 3, \cdots$}{
  \STATE Let $S = \{ i : P( \theta \cdot u_{i,t} > \theta \cdot u_{j,t}\ \forall j\ne i | u_{j,t}\ \forall j) < n^{-1}\}$ be the set of arms with unincentivized probability of being pulled below $n^{-1}$.
    \IF {$S\setminus V$ is non-empty}
  \STATE{Choose an arm $i$ uniformly at random from $S\setminus V$}
  \STATE{Pay whatever it takes to incentivize pulling arm $i$, i.e., offer payment 
    $c_{i,t} = \max_{\theta,j} \theta \cdot (u_{j,t} - u_{i,t})$ and $c_{j,t} = 0$ for $j \ne i$.}
  \ELSE
    \STATE {Let agents play myopically, i.e., offer payment $c_{j,t} = 0$ for all $j$}
  \ENDIF
    \STATE Denote $A_t$ as the pulled arm, update $V=V\cup\{A_t\}$, $u_{A_t,t}$ and $N(A_t,t)$
    \IF {$n\neq \min_{i}N(i,t)$} 
  \STATE $V=\emptyset$
    \ENDIF
    \STATE Update the round number, $n = \min_{i} N(i,t)$
}\ENDFOR

\end{algorithmic}
\end{algorithm}


\subsection{Assumptions}
In this section, we state several assumptions assumed by our analysis.  First define
\begin{align}
\Omega_{i,j}=\{\theta:B(\theta)=i, \hat{B}(\theta)=j\}, \nonumber 
\end{align}
which is the set of agent preferences whose best arm is arm $i$ and second best arm is arm $j$. With this definition, our analysis makes the following assumptions:

\begin{assumption} Let $F_{i,j}(y)$ be the marginal cumulative density function (or cumulative mass function if $F(\cdot)$ is a discrete distribution) of $(u_i-u_j)\cdot\theta$ conditioned on $\theta \in \Omega_{i,j}$. We assume $F_{i,j}(y)\leq My$ for all $y\in R^{+}$, $\forall i,j$. 
\label{A1}
\end{assumption}

As we can see later in our proof, we only need $\max_{i,j}\limsup_{y\rightarrow 0^{+}}\frac{F_{i,j}(y)}{y}$ to be finite. Intuitively, assumption~\ref{A1} states that there are not many agents who are indifferent between their best arm and the second best arm. 

\begin{assumption} We assume $F$ has a compact support set. Without loss of generality, we assume $\theta\in [0,W]^m$.
\label{A2}
\end{assumption}

We use $R = \max_{\theta, i,j} \theta \cdot (u_i - u_j)$ to denote the maximum regret that can be incurred at each time.  Assumption~\ref{A2} shows that $R<\infty$.

\begin{assumption}
Denote $p=\min_{i}P(\{\theta: B(\theta)=i\})$. We assume $p>0$.
\label{A3}
\end{assumption}

Assumption~\ref{A3} means each arm $i$ has a strictly positive proportion of users for which that arm is best. 

\subsection{General Results}

In this section, we prove Algorithm~\ref{Alg1} achieves $O(N^2+M(\log(T))^2)$ cumulative regret with $O(N^2)$ payment budget.  This is stated in the following pair of theorems, which together constitute our main results.

\begin{theorem}
The payment budget for Algorithm~\ref{Alg1} is bounded above by $O(N^2)$. 
\label{rst:budget}
\end{theorem}


\begin{theorem}
The cumulative regret for Algorithm~\ref{Alg1} is bounded above by $O(N^2 m + M m^2(\log(T))^2)$.
\label{rst:regret}
\end{theorem}



Before we prove these two theorems, we must first introduce two additional pieces of notation, which will be used in preliminary lemmas.  Let $S(\delta)$ be the proportion of users whose utility difference between their best and second best arm is less than $\delta$. Formally, $S(\delta)=P(\theta: \theta \cdot u_{B(\theta)}-\theta\cdot u_{\hat{B}(\theta)}\leq \delta)$. Then, let $p(\delta)=\min_{i}P(\{\theta:B(\theta)=i,\theta\cdot u_{B(\theta)}-\theta\cdot u_{\hat{B}(\theta)}>\delta\})$. We know $p(0)=p$. 

With this additional notation, we now prove several lemmas.
First, based on Assumption~\ref{A1}, we have the following bound for $S(\delta)$.

\begin{lemma}
$S(\delta)\leq M\delta$.
\label{lemma:sdelta}
\end{lemma}

\begin{proof}
\begin{align*}
S(\delta)
  &=\sum_{i,j}P(\theta\cdot(u_{i}-u_{j})\le \delta|\theta\in \Omega_{i,j})P(\theta\in \Omega_{i,j}) \\
    &\leq \sum_{i,j}M\delta \times P(\theta\in \Omega_{i,j}) \\
    &=M\delta.
    \end{align*}
    \end{proof}

    The following lemma bounds the probability of making a mistake if we let the agents play myopically in the $n^{th}$ round, given that the utility difference between his/her best and second best arm is bounded below by a constant. 

    \begin{lemma}
    Define $\tau$ to be any stopping time that is almost surely between $t_n$ and $t_{n+1}-1$ with respect to the filtration $\mathcal{F}_{t}=\sigma(A_1,\cdots,A_t,c_1,\cdots,c_t,O_1,\cdots,O_t)$, we have 
    \begin{align}
    P(\arg\max\{\theta_{\tau}\cdot u_{i,\tau}\}\neq B(\theta_{\tau})|\theta_{\tau}\cdot(u_{B(\theta_{\tau})}-u_{\hat{B}(\theta_{\tau})})> 2Wm\lambda)\leq 24Nm\exp\left(-\frac{1.8n\lambda^2}{16\sigma^2}\right), \nonumber
    \end{align}
    for $n\geq n_{0}=\max\{50, \frac{92.16\sigma^4}{\lambda^4}\}$.
    \label{round:prob}
    \end{lemma}


    We need the following lemma in order to prove Lemma~\ref{round:prob}.

    \begin{lemma}
    For $n\geq n_{0}=\max\{50, \frac{92.16\sigma^4}{\lambda^4}\}$, we have
    \begin{align}
    \frac{n\lambda}{4\sigma}\geq \sqrt{0.6n\log(\log_{1.1}(n)+1)}. \nonumber
    \end{align}
    \label{n0-inequality}
    \end{lemma}


    \begin{proof}
    First, we observe that
    \begin{align}
    &\frac{n\lambda}{4\sigma}\geq \sqrt{0.6n\log(\log_{1.1}(n)+1)} \nonumber \\
      \iff &\frac{n}{\log(\log_{1.1}(n)+1)}\geq \frac{9.6\sigma^2}{\lambda^2}. \nonumber 
      \end{align}
      Since $\log(x)\leq x-1$ for $x>0$, we know 
      \begin{align}
      \log(\log_{1.1}(n)+1)=\log\left(\frac{\log(n)}{\log(1.1)}+1\right)\leq \log(11\log(n)+1)\leq \log(11n)\leq 3+\log(n). \nonumber
      \end{align}

      Thus, we know
      \begin{align}
      \frac{n}{\log(\log_{1.1}(n)+1)}\geq \frac{n}{3+\log(n)}. \nonumber 
      \end{align}

      To prove the lemma, we just need to show for $n\geq n_{0}$, we have
      \begin{align}
      \frac{n}{3+\log(n)}\geq \frac{9.6\sigma^2}{\lambda^2}. \label{n0-equ}
      \end{align}
      Inequality~(\ref{n0-equ}) is true because of the following two observations:
      \begin{itemize}
      \item for $n\geq 50$, we have $\frac{n}{3+\log(n)}\geq n^{0.5}$;
      \item for $n\geq \frac{92.16\sigma^4}{\lambda^4}$, we have $n^{0.5}\geq {9.6\sigma^{2}}{\lambda^{2}}$.
      \end{itemize}

      Thus, we know our lemma is true.

      \end{proof}

      To prove Lemma~\ref{round:prob}, we also need to use an adaptive concentration inequality due to \cite{zhao2016adaptive}. For reference, we state it here as a Lemma.

      \begin{lemma}[Corollary 1 in \cite{zhao2016adaptive}]
      Let $X_{i}$ be zero mean $1/2$-subgaussian random variables. $\{S_{n}=\sum_{i=1}^{n}X_{i},n\geq 1\}$ be a random walk. Let $J$ be any stopping time with respect to $\{X_1,X_2,\cdots\}$. We allow $J$ to take the value of $\infty$ where $P(J=\infty)=1-\lim_{n\rightarrow \infty}P(J\leq n)$. If
      \begin{align}
      f(n)=\sqrt{0.6n\log(\log_{1.1}(n)+1)+bn}, \nonumber
      \end{align}
      then
      \begin{align}
      Pr[\{S_{J}\geq f(J)\}\cap \{J<\infty\}]\leq 12e^{-1.8b}. \nonumber
      \end{align}
      \label{ACI-inequality}
      \end{lemma}


      We now prove Lemma~\ref{round:prob}.

      \begin{proof}[Proof of Lemma~\ref{round:prob}]
      In the $n^{th}$ round, we know all arms have been pulled at least $n$ times. For all the agents $\theta$ whose utility difference between their best and second best arm is greater than $2mW\lambda$, denote $K(\theta)=\max_{i\neq B(\theta)}\{\theta\cdot u_{i,t}\}$. If $|u_{i,t}^{j}-u_{i}^{j}|\leq \lambda$ for all $i,j$, then
      \begin{align}
      &\theta\cdot(u_{B(\theta),t}-u_{K(\theta),t}) \nonumber \\
        \geq & \theta\cdot(u_{B(\theta),t}-u_{B(\theta)}) + \theta\cdot(u_{K(\theta)}-u_{K(\theta),t}) + \theta\cdot(u_{B(\theta)}-u_{K(\theta)}) \nonumber \\
        > & -Wm\lambda - Wm\lambda + 2Wm\lambda = 0,\nonumber
        \end{align}
        which means their myopic action would incur no regret.


        Define $\epsilon_{i,\tau}=u_{i,\tau}-u_i$ and $\epsilon_{i,\tau}^{j}$ to be the $j^{th}$ component of $\epsilon_{i,\tau}$. Thus, we have
        \begin{align}
        &P(\arg\max\{\theta_{\tau}\cdot u_{i,\tau}\}\neq B(\theta_{\tau})|\theta_{\tau}\cdot(u_{B(\theta_{\tau})}-u_{\hat{B}(\theta_{\tau})})> 2Wm\lambda)\nonumber \\
          \leq &P(\exists i, \exists j, |u_{i,\tau}^{j}-u_{i}^{j}|\geq\lambda |\theta_{\tau}\cdot(u_{B(\theta_{\tau})}-u_{\hat{B}(\theta_{\tau})})> 2Wm\lambda) \nonumber \\
          \leq & \sum_{i}\sum_{j} P(|u_{i,\tau}^{j}-u_{i}^{j}|\geq\lambda|\theta_{\tau}\cdot(u_{B(\theta_{\tau})}-u_{\hat{B}(\theta_{\tau})})> 2Wm\lambda) \nonumber \\
          = &  \sum_{i}\sum_{j} P(|\epsilon_{i,\tau}^{j}|\geq\lambda|\theta_{\tau}\cdot(u_{B(\theta_{\tau})}-u_{\hat{B}(\theta_{\tau})})> 2Wm\lambda). \label{ACI}
          \end{align}

          To bound equation~(\ref{ACI}), we use Lemma~\ref{ACI-inequality}. Define
          \begin{align}
          S_{N(i,\tau)}^{i,j}=\frac{\epsilon_{i,\tau}^{j}}{2\sigma}. \nonumber
          \end{align}

          Based on Lemma~\ref{n0-inequality}, for $n_{0}=\max\{50, \frac{92.16\sigma^2}{\lambda^2}\}$ and $n\geq n_{0}$, we have
          \begin{align}
          \frac{n\lambda}{4\sigma}\geq \sqrt{0.6n\log(\log_{1.1}(n)+1)}. \nonumber
          \end{align}

          Thus, if we set $b=\frac{n\lambda^2}{16\sigma^2}$ in Lemma~\ref{ACI-inequality}, for any $N(i,\tau)\geq n\geq n_{0}$, we have
          \begin{align}
          \frac{N(i,\tau)\lambda}{2\sigma}\geq & \sqrt{0.6N(i,\tau)\log(\log_{1.1}(N(i,\tau))+1)}+\frac{\lambda}{4\sigma}\sqrt{n N(i,\tau)} \nonumber \\
            \geq & \sqrt{0.6N(i,\tau)\log(\log_{1.1}(N(i,\tau))+1)+bN(i,\tau)}, \nonumber 
            \end{align}
            where the last inequality is because $\sqrt{x}+\sqrt{y}\geq \sqrt{x+y}$. Thus, we have
            \begin{align}
            &P(\epsilon_{i,\tau}^{j}\geq\lambda|\theta_{\tau}\cdot(u_{B(\theta_{\tau})}-u_{\hat{B}(\theta_{\tau})})> 2Wm\lambda) \nonumber \\
              =&P\left(S_{N(i,\tau)}^{i,j}\geq \frac{N(i,\tau)\lambda}{2\sigma}\bigg|\theta_{\tau}\cdot(u_{B(\theta_{\tau})}-u_{\hat{B}(\theta_{\tau})})> 2Wm\lambda\right) \nonumber \\
              \leq & P\left(S_{N(i,\tau)}^{i,j}\geq \sqrt{0.6 N_{i,\tau}\log(\log_{1.1}(N(i,\tau))+1)+b N(i,\tau)}\bigg|\theta_{\tau}\cdot(u_{B(\theta_{\tau})}-u_{\hat{B}(\theta_{\tau})})> 2Wm\lambda\right) \nonumber \\
              \leq & 12\exp( -1.8b) = 12\exp\left(\frac{-1.8 n\lambda^2}{16\sigma^2}\right). \nonumber
              \end{align}
              Similarily, we can bound 
              \begin{align}
              &P(\epsilon_{i,\tau}^{j}\leq-\lambda|\theta_{\tau}\cdot(u_{B(\theta_{\tau})}-u_{\hat{B}(\theta_{\tau})})> 2Wm\lambda) \nonumber \\
                =&P(-\epsilon_{i,\tau}^{j}\geq \lambda|\theta_{\tau}\cdot(u_{B(\theta_{\tau})}-u_{\hat{B}(\theta_{\tau})})> 2Wm\lambda) \nonumber \\
                \leq & 12\exp\left(\frac{-1.8 n\lambda^2}{16\sigma^2}\right). \nonumber 
                \end{align}

                Therefore, we know $P(|\epsilon_{i,\tau}^{j}|\geq \lambda|\theta_{\tau}\cdot(u_{B(\theta_{\tau})}-u_{\hat{B}(\theta_{\tau})})> 2Wm\lambda)\leq 24\exp\left(\frac{-1.8 n\lambda^2}{16\sigma^2}\right)$. Thus, we know
                \begin{align}
                &\sum_{i}\sum_{j} P(|\epsilon_{i,\tau}^{j}|\geq \lambda|\theta_{\tau}\cdot(u_{B(\theta_{\tau})}-u_{\hat{B}(\theta_{\tau})})> 2Wm\lambda)  \nonumber \\
                  \leq& 24Nm \exp\left(\frac{-1.8 n\lambda^2}{16\sigma^2}\right). \nonumber
                  \end{align}

                  \end{proof}

                  Before we start analyzing the cumulative regret, we first prove the following lemma which bounds the expected length of each round.

                  \begin{lemma}
                  Using our algorithm, we have $\mathbb{E}[t_{n+1}-t_{n}]\leq Nn$, $\forall n\geq 1$.
                  \label{round:length}
                  \end{lemma}


                  \begin{proof}
                  A round completes when each arm is pulled at least once in that round. Let $X_{i}$ be the number of agents who come to the system between the time after the $(i-1)^{th}$ unique arm was pulled, up to and including the time when the $i^{th}$ unique arm was pulled. Then we know 
                  \begin{equation*}
                  \mathbb{E}[t_{n+1}-t_{n}]=\sum_{i=1}^{N}E[X_{i}].
                  \end{equation*}


                  Fix $i$. In bounding $X_i$, we think of agents as ``trials'', where each trial can result in a new unique arm being pulled (which we call a ``successful'' trial), or not.  There are two ways a trial can be successful:
                  \begin{itemize}
                  \item If there is at least one arm that has not been pulled and the probability of an agent utility function that would pull this arm without incentives is less than $n^{-1}$, then the principal will offer an incentive that causes this arm to be pulled (or one of these arms if there is more than one). In this case, the probability that the trial is succesful is $1$.  
                  \item The probability of an agent utility function that would pull each un-pulled arm without incentives is at least $n^{-1}$. In this case, the probability that the trial is successful is at least $n^{-1}$.
                  \end{itemize}

                  Thus, $X_{i}$ is stochastically dominated below by a geometric random variable with success probability $n^{-1}$, the expected number of trials up to and including the first success, $E[X_i]$, is bounded above by $n$.  Thus,
                  \begin{align}
                  E[t_{n+1}-t_{n}]\leq Nn. \nonumber
                  \end{align}
                  \end{proof}


                  \begin{lemma}
                  The expected number of payments for Algorithm~\ref{Alg1} is bounded above by $O(N^2)$.
                  \label{lemma:numP}
                  \end{lemma}

                  \begin{proof}
                  If $|u_{i}^{j}-u_{i,t}^{j}|\leq \lambda$ is true $\forall i$, $\forall j$, then we know for those $\theta\in \{\theta:\theta\cdot u_{B(\theta)}-\max_{j\neq B(\theta)}\{\theta \cdot u_{j}\}> 2mW\lambda\}$, they will correctly identify their best arm. Thus we know, in the $n^{th}$ round, if $|u_{i}^{j}-u_{i,t}^{j}|\leq \frac{p^{-1}(\frac{p}{2})}{2Wm}$ $\forall i$ and $\forall j$, and $n^{-1}\leq p/2$, we do not need to incentivize any arms. In order to have $n^{-1}\leq \frac{p}{2}$, we need $n\geq \frac{2}{p}$. Denote $n_1=\max\{n_{0}, \frac{2}{p}\}$. Denote $\delta_{0}=p^{-1}(\frac{p}{2})>0$ (because of Assumption~\ref{A1}).

                  Define $\tau_{n}^{i}$ to be the first time we pull arm $i$ in the $n^{th}$ round. Then
                  \begin{align}
                  \sum_{t=1}^{\infty}\mathbbm{1}\{c(t)>0\} =\sum_{n=1}^{\infty}\sum_{i=1}^{N}\mathbbm{1}\{c(\tau_{n}^{i})>0\}. \nonumber
                  \end{align}

                  The cumulative expected number of payments is bounded above by:
                  \begin{align}
                  &E\left[\sum_{t=1}^{\infty}\mathbbm{1}\{c(t)>0\}\right] \nonumber \\
                    =&\sum_{n=1}^{\infty}\sum_{i=1}^{N}P(c(\tau_{n}^{i})>0) \nonumber \\
                    \leq &\sum_{n=n_{1}}^{\infty}\sum_{i=1}^{N}P\left(\exists i,j:|u_{i}^{j}-u_{i,\tau_{n}^{i}}^{j}|>\frac{p^{-1}(\frac{p}{2})}{2Wm}\right)+\sum_{n=1}^{n_1}N \nonumber \\
                    \leq &\sum_{n=n_{1}}^{\infty}\sum_{i=1}^{N}24Nm \exp\left(\frac{-1.8 n\delta_{0}^2}{64W^2 m^2\sigma^2}\right) +\sum_{n=1}^{n_1}N \nonumber \\
                    \leq & \sum_{n=n_{1}}^{\infty}24Nm \exp\left(\frac{-1.8 n\delta_{0}^2}{64W^2 m^2\sigma^2}\right)\times N+\sum_{n=1}^{n_1}N \nonumber  \\
                    \leq&24N^2 m \frac{1}{\exp(\frac{1.8\delta_{0}}{64W^2 m^2\sigma^2})-1} + Nn_1, \nonumber
                    \end{align}

                    Thus, we know the expected number of payments is bounded above by $O(N^2)$.

                    \end{proof}


                    Now we are ready to prove our second main result, Theorem~\ref{rst:regret}.

                    \begin{proof}
                    For regret incurred in the first $n_0$ round, it is bounded above by $\sum_{n=1}^{n_{0}}NRn$.

                    For regret incurred after the first $n_0$ round, it has two different components: the regret incurred when we let the agents play myopically and the regret incurred when we incentivize the agents. Using Lemma~\ref{lemma:numP}, the expected regret incurred when we incentivize the agents is bounded above by: $\left[24N^2 m \frac{1}{\exp(\frac{1.8\delta_{0}}{64W^2 m^2\sigma^2})-1} + Nn_1\right]R$.

                    For the regret incurred when we let the agents play myopically at time $t\geq t_{n_0}$, it consists of the following two components:
                    \begin{itemize}
                    \item For those users whose utility difference between their best and the second best arm is greater than $f(t)$: we define a sequence of stopping time $\tau_{n}^{k}$ to be the $k^{th}$ time period in the $n^{th}$ round. For $k>t_{n+1}-t_{n}$, we define $\tau_{n}^{k}=\infty$. For $\tau_{n}^{k}=t$, the probability of these users making a mistake is bounded above by $24Nm\exp\left(-\frac{1.8n f(\tau_{n}^{k})^2}{64 W^2 m^2\sigma^2}\right)$ and the expected regret is bounded above by $24Nm\exp\left(-\frac{1.8n f(\tau_{n}^{k})^2}{64 W^2 m^2\sigma^2}\right)\times R$. We denote the regret incurred by these agents as $r_1(\tau_{n}^{k})$. For $k>t_{n+1}-t_{n}$, we define $r_1(\tau_{n}^{k})=0$.
                    \item For those user whose utility difference between their best and the second best arm is smaller than $f(t)$: this happens with probability $S(f(t))$ at each time and regret is bounded above by $S(f(t)) \times f(t)=Mf(t)^2$. We denote the regret incurred by these agents as $r_2(t)$.
                    \end{itemize}

                    Thus, the cumulative expected regret incurred up to time $T$ when we let the agent play myopically is bounded above by:
                    \begin{align}
                    &E\left[\sum_{t=1}^{T}r(t)\right] \nonumber \\
                      =&E\left[\sum_{t=1}^{t_{n_{0}}} r(t) + \sum_{t=t_{n_{0}}}^{T}(r_1(t)+r_2(t))\right]  \nonumber \\
                      \leq & \sum_{n=1}^{n_{0}}NRn + E\left[\sum_{n=n_{0}}^{T}\sum_{t=t_{n}}^{t_{n+1}-1}r_1(t)\right]+ E\left[\sum_{t=1}^{T}r_2(t)\right] \nonumber \\
                      =& \sum_{n=1}^{n_{0}}NRn + E\left[\sum_{n=n_{0}}^{T}\sum_{k=1}^{\infty}r_1(\tau_{n}^{k})\right]+ E\left[\sum_{t=1}^{T}r_2(t)\right]. \label{chap5:equ:r1}
                      \end{align}
                      Since
                      \begin{align}
                      & E\left[\sum_{n=n_{0}}^{T}\sum_{k=1}^{\infty}r_1(\tau_{n}^{k})\right] \nonumber\\
                        = &  \sum_{n=n_{0}}^{T}\sum_{k=1}^{\infty}E[r_1(\tau_{n}^{k})] \nonumber \\
                        = &  \sum_{n=n_{0}}^{T}\sum_{k=1}^{\infty}(E[r_1(\tau_{n}^{k})|\tau_{n}^{k}<\infty]\times P(\tau_{n}^{k}<\infty)+E[r_1(\tau_{n}^{k})|\tau_{n}^{k}=\infty]\times P(\tau_{n}^{k}=\infty)) \nonumber \\
                        = & \sum_{n=n_{0}}^{T}\sum_{k=1}^{\infty}E[r_1(\tau_{n}^{k})|\tau_{n}^{k}<\infty]\times P(\tau_{n}^{k}<\infty), \nonumber
                        \end{align}

                        we have
                        \begin{align}
                        \eqref{chap5:equ:r1}= & \sum_{n=1}^{n_{0}}NRn + \sum_{n=n_{0}}^{T}\sum_{k=1}^{\infty}E[r_1(\tau_{n}^{k})|\tau_{n}^{k}<\infty]\times P(\tau_{n}^{k}<\infty) + E\left[\sum_{t=1}^{T}r_2(t)\right] \nonumber \\
                          \leq & \sum_{n=1}^{n_{0}}NRn + \sum_{n=n_{0}}^{T}\left[\sum_{k=1}^{\infty}24Nm\exp\left(-\frac{1.8n f(\tau_{n}^{k})^2}{64 W^2 m^2\sigma^2}\right) R\times P(\tau_{n}^{k}<\infty) \right]+ \sum_{k=1}^{T}Mf(t)^2 \nonumber \\
                          \leq & \sum_{n=1}^{n_{0}}NRn + \sum_{n=1}^{T} 24Nm\exp\left(-\frac{1.8n f(n)^2}{64 W^2 m^2\sigma^2}\right) R \times Nn+ \sum_{t=1}^{T}Mf(t)^2. \label{chap5:equ:regret}
                          \end{align}


                          Thus the cumulative regret at time $T$ is bounded above by
                          \begin{align}
                          &\sum_{n=1}^{n_{0}}NRn + \sum_{n=1}^{T} 24Nm\exp\left(-\frac{1.8n f(n)^2}{64 W^2 m^2\sigma^2}\right)\times R \times Nn+ \sum_{t=1}^{T}Mf(t)^2 \nonumber \\
                            + & 24N^2 m \frac{1}{e^{\frac{1.8\delta_{0}}{64W^2 m^2\sigma^2}}-1}R+N\left(\max\left\{n_{0},\frac{2}{p}\right\}\right)R. \nonumber
                            \end{align}

                            For a fixed $T$, we only need to minimize the following two terms since all others are constant:

                            \begin{align}
                            \sum_{n=1}^{T} 24Nm\exp\left(-\frac{1.8n f(n)^2}{64 W^2 m^2\sigma^2}\right)\times R \times Nn+ \sum_{t=1}^{T}Mf(t)^2. \label{equ:regret}
                            \end{align}


                            If we set $f^2(t)=\frac{2\log(T)\times 64W^2 m^2\sigma^2}{1.8t}$, then
                            \begin{align}
                            &\sum_{n=1}^{T} 24Nm\exp\left(-\frac{1.8n f(n)^2}{64 W^2 m^2\sigma^2}\right)\times R \times Nn+ \sum_{t=1}^{T}Mf(t)^2 \nonumber \\ 
                            \leq & \sum_{n=1}^{T} 24N^2 mnR \exp\left(-2\log(T)\right)  + \frac{128W^2 m^2\sigma^2 M\log(T)}{1.8}\sum_{t=1}^{T}\frac{1}{n} \nonumber \\
                              \leq &  24N^2 m R\frac{T(T-1)}{2T^2}  + 71.12 W^2 m^2\sigma^2 M\log(T)(\log(T)+1) \nonumber \\
                              \leq &  12 N^2 m R  + 71.12 W^2 m^2\sigma^2 M\log(T)(\log(T)+1). \nonumber
                              \end{align}

                              Thus, the cumulative expected regret is bounded by $O(N^2 m + M m^2\log(T))$.
                              \end{proof}

                              \begin{corollary}
                              If $\exists \delta>0$ such that $F_{i,j}(\delta)=0$ for all $i,j$, then the cumulative expected regret is bounded by $O(N^2)$.
                              \end{corollary}


                              \begin{proof}
                              The proof of this corollary is similar to the proof of Theorem~\ref{rst:regret}. If we set $f^2(t)=\frac{2\log(T)\times 64W^2 m^2\sigma^2}{1.8t}$, then there exists a $t_{0}$ such that for $t>t_{0}$, $S(f(t))=0$. Thus, similar to equation~\eqref{equ:regret}, we know the cumulative expected regret when we let the agents play myopically is bounded above by
                              \begin{align}
                              \sum_{n=1}^{n_{0}}NRn + \sum_{n=1}^{T} 24Nm\exp\left(-\frac{1.8n f(n)^2}{64 W^2 m^2\sigma^2}\right)\times R \times Nn+ \sum_{t=1}^{t_{0}}Mf(t)^2 \nonumber
                              \end{align}
                              Therefore, based on the same analysis of Theorem~\ref{rst:regret}, we know the cumulative regret is bounded by $O(N^2)$.
                              \end{proof}


                              \subsection{Practical Issues}
                              \label{sec:pi}

                              In Algorithm~\ref{Alg1}, we use ``pay whatever it takes'' strategy when we decide to incentivize the agent. However, ''pay whatever it takes'' only shows up in the proof of Lemma~\ref{round:length}. Without loss of generality, suppose we want to incentivize arm $i$ at time $t$ at the $n^{th}$ round. Based on the proof of Lemma~\ref{round:length}, as long as we offer a payment $c_{i,t}$ such that arm $i$ has at least $n^{-1}$ probability being pulled at time $t$, our results still hold true. We could compute this $c_{i,t}$ dynamically based on $F(\cdot)$ as well as our current estimate $u_{i,t}$. Here is the revised algorithm which would work well in practice:


                              \begin{algorithm}
                              \caption{Algorithm: Incentivizing Exploration}
                              \label{Alg2}
                              \begin{algorithmic}
                              \STATE Set n = 1 to denote the round number; Let $V=\emptyset$ be the set of arms that were pulled in the current round;
                              \FOR{ $t = 1, 2, 3, \cdots$}{
                                \STATE Let $S = \{ i : P( \theta \cdot u_{i,t} > \theta \cdot u_{j,t}\ \forall j\ne i | u_{j,t}\ \forall j) < n^{-1}\}$ be the set of arms with unincentivized probability of being pulled below $n^{-1}$.
                                  \IF {$S\setminus V$ is non-empty}
                                \STATE{Choose an arm $i$ uniformly at random from $S\setminus V$}
                                \STATE{Offer payment $c_{i,t}=\inf\{c: P(\theta\sim F: \theta\cdot u_{i,t}+c>max_{j}\theta\cdot u_{j,t})>n^{-1}\}$}
                                \ELSE
                                  \STATE {Let agents play myopically, i.e., offer payment $c_{j,t} = 0$ for all $j$}
                                \ENDIF
                                  \STATE Denote $A_t$ as the pulled arm, update $V=V\cup\{A_t\}$, $u_{A_t,t}$ and $N(A_t,t)$
                                  \IF {$n\neq \min_{i}N(i,t)$} 
                                \STATE $V=\emptyset$
                                  \ENDIF
                                  \STATE Update the round number, $n = \min_{i} N(i,t)$
                              }\ENDFOR

\end{algorithmic}
\end{algorithm}


The same proof would work and we can get the exact same results as Algorithm~\ref{Alg1}. 




\section{A Lower Bound of $\Omega(\log(T))$} \label{sec:lb}

We saw in Theorem~\ref{rst:discrete} that 
when agent preferences for their best arm are sufficiently clear,
in the sense that $\TieDensity = 0$, the regret of
Algorithm~\ref{alg:basic-incentivizing} is bounded by a constant.
One may conjecture that this should hold more generally,
in that the regret of the (fewer and fewer) agents on the boundary
between close arms should go to 0, while their fraction also goes to 0.
In this section, we establish a lower bound,
showing that even in very simple settings, a (logarithmic) dependence
on $T$ is typically unavoidable for \emph{any} incentivization strategy.

We consider an instance with two arms,
whose attribute vectors are $(0,0)$ and $(0,1)$, respectively.
Agent types are distributed uniformly on the (edge of the)
two-dimensional unit square%
\footnote{While our model technically assumed that all agent coordinates are
non-negative, we could simply shift the unit square.
The present choice is solely for ease of notation.}  
$\Set{\AgV}{\max(|\Ag{1}|, |\Ag{2}|) = 1}$,
with density $\frac{1}{8}$.
Because $\AgV \cdot (0,0) = 0$ and $\AgV \cdot (0,1) = \Ag{2}$,
the best choice for agent \AgV is arm $(0,1)$ iff $\Ag{2} > 0$,
i.e., the top half of the unit square prefers the arm $(0,1)$,
and the bottom half prefers the arm $(0,0)$.

Since we are proving a lower bound, we give the algorithm the
following extra two advantages:
(1) there is no noise in the observations of the arm $(0,0)$,
and all agents know its location deterministically.
(2) in each time step, regardless of which arm is pulled, the algorithm
and all agents observe a pull from arm $(0,1)$.
For simplicity of notation, we set the standard deviation of the arm
$(0,1)$ to $\sigma = 1$;
different values only lead to a scaling of the results.

Under these advantages, myopic play is clearly optimal, so it suffices
to bound the regret of the myopic algorithm which never incentivizes
agents. 

Because a pull of arm $(0,1)$ is observed in each time step,
after $t$ rounds, the estimate \ArmEV{t}{(0,1)} is of the form
$(0,1) + (\Noise[t]{1}, \Noise[t]{2})$,
where $\Noise[t]{1} \sim \Normal{0}{\sqrt{1/t}}$
and $\Noise[t]{2} \sim \Normal{0}{\sqrt{1/t}})$.
We lower-bound the regret in step $t$ by focusing on the event that
both normal noise coordinates are non-negative,
which by symmetry has probability \quarter:

\begin{align*}
\Expect{\Regret{t}} 
  & \geq \ExpectC{\Regret{t}}{\Noise[t]{1} > 0, \Noise[t]{2} > 0}
         \cdot \Prob{\Noise[t]{1} > 0, \Noise[t]{2} > 0}
  \; = \; \quarter \ExpectC{\Regret{t}}{\Noise[t]{1} > 0, \Noise[t]{2} > 0}.
\end{align*}

For the moment, focus on some time step $t$,
and write $\NoiseV = \NoiseV[t]$.
Then, there are two types of agents who pull the wrong arm and incur
regret:
\begin{enumerate}
\item If $\Ag{2} > 0$ and $\Ag{1} \Noise{1} + \Ag{2} (1+\Noise{2}) < 0$
then \AgV should pull $(0,1)$,
but will wrongly pull $(0,0)$ and incur regret \Ag{2}.
The range of \AgV making the wrong choice is thus
$0 < \Ag{2} < \frac{- \Noise{1}}{1+\Noise{2}} \cdot \Ag{1}$.
Since we are proving a lower bound, we only focus on the
case $\Ag{1} = -1$, and ignore the case $\Ag{2} = 1$
(which is rare, since it requires \Noise{1} to be large).
Thus, the set of agents incurring regret contains the set
$\Set{\AgV = (-1, \Ag{2})}{0 < \Ag{2} < \frac{\Noise{1}}{1+\Noise{2}}}$.

% For any particular value of \Ag{2},
% the length of the interval of corresponding \Ag{1} is
% \begin{align*}
%   \sqrt{1-\Ag{2}^2} - \frac{1+\Noise{2}}{\Noise{1}} \cdot \Ag{2}
%   & \geq 1 - \Ag{2} - \frac{1+\Noise{2}}{\Noise{1}} \cdot \Ag{2}
%   \; = \; 1 - \frac{1+\Noise{1}+\Noise{2}}{\Noise{1}} \cdot \Ag{2}.
% \end{align*}

\item If $\Ag{2} < 0$ and $\Ag{1} \Noise{1} + \Ag{2} (1+\Noise{2}) > 0$,
then \AgV should pull $(0,0)$,
but will wrongly pull $(0,1)$ and incur regret $-\Ag{2}$.
This region and its regret are rotationally symmetric to the previous
case.
\end{enumerate}

Hence, the expected regret for given \Noise{1}, \Noise{2} is at least
%\begin{align*}
$2 \int_0^{\frac{\Noise{1}}{1+\Noise{2}}}
  \Ag{2} \cdot \frac{1}{8} \dd \Ag{2}
=
\frac{1}{8} \cdot \left(\frac{\Noise{1}}{1+\Noise{2}}\right)^2$.
%\end{align*}
The expected regret, conditioned on
$\Noise{1} > 0$ and $\Noise{2} > 0$, is therefore

\begin{align*}
\ExpectC{\Regret{t}}{\Noise{1} > 0, \Noise{2} > 0}
 & \geq \frac{1}{\Prob{\Noise{1}>0,\Noise{2}>0}} \cdot \frac{1}{8}
    \int_{0}^{\infty} \int_{0}^{\infty}
    \left( \frac{\Noise{1}}{1+\Noise{2}} \right)^2 \cdot
    \frac{\e^{-\frac{t \Noise{1}^2}{2}} \sqrt{t}}{\sqrt{2\pi}} \dd\Noise{1}
    \frac{\e^{-\frac{t \Noise{2}^2}{2}} \sqrt{t}}{\sqrt{2\pi}} \dd\Noise{2} \\
 & = \frac{1}{4\pi} \int_{0}^{\infty} \int_{0}^{\infty}
   \left( \frac{\Noise{1}}{\sqrt{t}+\Noise{2}} \right)^2
   \e^{-\Noise{1}^2/2} \e^{-\Noise{2}^2/2} \dd \Noise{1} \dd \Noise{2}.
\end{align*}

We want to show that the expected regret per time step decreases only
at a rate of $\Omega(1/t)$, and thereto consider the limit of the
ratio of the two quantities:

\begin{align*}
\lim_{t \to \infty} \frac{\ExpectC{\Regret{t}}{\Noise{1} > 0, \Noise{2} > 0}}{\frac{1}{t}}
  & \geq
    \lim_{t \to \infty} \left( \frac{1}{4\pi} \cdot
    \int_{0}^{\infty} \int_{0}^{\infty}
    t \cdot \left( \frac{\Noise{1}}{\sqrt{t}+\Noise{2}} \right)^2
    \e^{-\Noise{1}^2/2} \e^{-\Noise{2}^2/2} \dd \Noise{1} \dd \Noise{2} \right) \\
  & = \frac{1}{4\pi} \cdot \int_{0}^{\infty} \int_{0}^{\infty}
    \lim_{t \to \infty}
    \left( t \cdot \left( \frac{\Noise{1}}{\sqrt{t}+\Noise{2}} \right)^2 \right)
    \e^{-\Noise{1}^2/2} \e^{-\Noise{2}^2/2} \dd \Noise{1} \dd \Noise{2} \\
  & = \frac{1}{4\pi} \cdot \int_{0}^{\infty} \int_{0}^{\infty}
    \e^{-\Noise{1}^2/2} \e^{-\Noise{2}^2/2} \dd \Noise{1} \dd \Noise{2}
  \; = \; \frac{1}{4 \pi}; 
\end{align*}

the second line is due to the monotone convergence theorem,
which can be applied because
$t \left(\frac{\Noise{1}}{\sqrt{t}+\Noise{2}}\right)^2$
is strictly increasing in $t$.
Because the expected regret in each time step is at least
$\Omega(1/t)$, the total expected regret is at least
$\Omega(\sum_{t=1}^{T}\frac{1}{t}) = \Omega(\log(T))$.


\section{Conclusion}
We study the problem of incentivizing exploration with heterogeneous
user preferences.
We proposed an algorithm that achieves expected cumulative regret of
$O(\ARMNUM \e^{2/\MinProb} + \ARMNUM \log^3(T))$,
using expected cumulative payments of $O(\ARMNUM^2 \e^{2/\MinProb})$.
It is possible to improve these bounds to polynomial (in \ARMNUM and
$1/\MinProb$) when \MinProb is known or the preference distribution is
discrete.
In fact, we conjecture that this should be possible even in the full
generality of our model.
As a first step towards such a polynomial bound, we believe that it
should be possible to obtain an exponential dependence on
$1/(\MinProb \ARMNUM)$, which gives polynomial dependence unless some
arm has a much smaller fraction of the population preferring it.

Taking this goal one step further, we would like to 
develop algorithms that do not require all arms to be preferred by a
strictly positive fraction of agents.
An alternate algorithm might only incentivize an arm if its estimated
attribute vector is close enough to a Pareto frontier.
The regret will then be $\Omega(\log(T))$ when at least one arm falls
below the Pareto frontier, as we no longer have free exploration of
all arms. 
It is likely that a bound will deteriorate as the number of such
unpreferred arms increases.

Finally, it would be desirable to generalize the class of utility
functions that can be handled beyond inner products.
We believe that similar results hold for arbitrary
Lipschitz-continuous utility functions of the arm's attribute vector,
and that only minor modifications are necessary to the algorithm and
proofs.



\newpage

\appendix

\section*{Proof of Theorem~\ref{rst:budget}}

We also need the following lemma in part of the proof of Theorem~\ref{rst:budget}.
\begin{lemma}
For all $n\geq 1$, we have
\begin{align}
0.9n^{5/6} \geq \sqrt{0.6n \log(\log_{1.1}(n)+1)}. \nonumber
\end{align}
\label{lemma:cal2}
\end{lemma}
\begin{proof}
\begin{align}
&0.9n^{5/6} \geq \sqrt{0.6n \log(\log_{1.1}(n)+1)} \nonumber \\
\iff & 0.81 n^{5/3} \geq 0.6n \log(\log_{1.1}(n)+1) \nonumber \\
\iff &  \frac{81}{60} n^{2/3} \geq  \log(\log_{1.1}(n)+1) \nonumber
\end{align}

Denote $f(x) = \frac{81}{60}x^{2/3} - \log(\log_{1.1}(x)+1)$. It's easy to compute $f^{'}(x)=0$ has a unique solution $x_{0}=e^{2/3w(\frac{20e^{20000/314763}}{27})-\frac{10000}{104921}}$ (here $w(\cdot)$ is the Lambert W-Function) and it is the global minimum. Since $f(x_{0})\approx 0.0252 >0$, we know $f(x)>0$ for all $x\geq 1$. Thus, our lemma holds true.
\end{proof}


Now we are ready to prove our first main result, Theorem~\ref{rst:budget}.

\begin{proof}

\noindent\textbf{Step 1: Categorize measurement errors into different radius envelopes}


Denote $\epsilon_{i,t}=u_{i,t}-u_{i}$ to be the estimation error for the attribute vector $u_i$ at time $t$. Denote $\epsilon_{i,t}^{j}$ to be the $j^{th}$ component of $\epsilon_{i,t}$. Denote $\omega$ to be a sample path and $n(t,\omega)$ to be the round number for sample path $\omega$ at time $t$. For a fixed time $t$, define
\begin{align}
L^{'}[\ell](t) = \{\omega:|\epsilon_{i,t}^{j}(\omega)|\leq g(n(t,\omega),\ell), \forall i,j\}\nonumber
\end{align}
where $g(n,\ell)$ is a function which we will define later. Define $L[1](t) = L^{'}[1](t)$ and $L[i](t) = L^{'}[i](t)\setminus L^{'}[i-1](t)$ for $i\geq 2$. We call $L[\ell](t)$ the $\ell^{th}$ envelope at time $t$. We often simplify the notation and use $L[\ell]$ instead of $L[\ell](t)$ without confusion.

In the calculation below, we omit the dependency on $\omega$ when refering to variables $c(t)$, $\epsilon_{i,t}^{j}$ and $t_n$. Based on the definition of $L[\ell]$, we know if $\omega\in L[\ell]$, the maximum payment we need to offer at time $t$ is bounded above by 
\begin{align}
&\max_{i}\theta_t\cdot u_{i,t} - \min_{j}\theta_t\cdot u_{j,t} \nonumber \\
= &\max_{i}\theta_t\cdot (\epsilon_{i,t}+u_i) - \min_{j}\theta_t\cdot (\epsilon_{j,t}+u_j) \nonumber \\
\leq &\max_{i}\theta_t\cdot u_i - \min_{j}\theta_t\cdot u_j +\max_{i}\theta_t\cdot \epsilon_{i,t} - \min_{j}\theta_t\cdot \epsilon_{j,t}\nonumber \\
\leq & R + 2Wmg(n,\ell). \nonumber
\end{align}

We denote $\bar{c}(n,\ell)=R + 2Wmg(n,\ell)$ as the upper bound of payment for round $n$ if our measurement error lie in the $\ell^{th}$ envelope.


\noindent\textbf{Step 2: Introduce a new stochastic process which bounds the total payment in a round}

Denote $\mathcal{F}_t$ as the filtration up to time $t$. Based on our algorithm, we know

\[ P(\text{play a new arm}|\mathcal{F}_t) =
  \begin{cases}
      1       & \quad \text{if we incentivize}\\
          \sum_{i\notin (V\cup S)} P(\theta \cdot u_{i,t}>\theta\cdot u_{j,t} \forall j\neq i)  & \quad \text{if }  S\setminus V\neq \emptyset 
            \end{cases} \label{dom_stoc}
            \] 
            \hspace{1cm}$\geq (n-|V|)\times \frac{1}{n}=1-\frac{|V|}{n}$.

            Let $(Z_{n,v,m}:m\geq 0)$ be a sequence of independent Bernoulli random variable with success probability $(1-\frac{v}{n})$. We will construct an alternative stochastic process for selecting which arm gets played that has the same distribution as the original process, but under which
            \begin{align}
            t_{n+1}-t_{n}\leq \bar{T}_{n}:=\sum_{v=0}^{N-1}\bar{T}_{n,v}, \nonumber 
            \end{align}
            where $\bar{T}_{n,v}:=\inf\{m\geq 0: Z_{n,v,m}=1\}+1$.

            The new stochastic process will have the property that whenever $Z_{n,v(n,t),m(n,t)}=1$, we will play a new arm at time t for $t\in [t_{n}, t_{n+1}]$, where $v(n,t)$ is the number of unique arms played in round $n$ strictly before time $t$, and $m(n,t)$ is the number of times we have pulled a previously pulled arm for the current value of $v(n,t)$. At time $t$, to determine what arm to pull, calculate \eqref{dom_stoc} and let $q_t$ be the probability computed. Note $q_t\geq 1-\frac{v(n,t)}{n}$.

            If $Z_{n,v(n,t),m(n,t)}$ is 1, decide to play a new arm. Otherwise, draw a second Bernoulli random variable with probability $\frac{q_t-(1-\frac{v(n,t)}{n})}{1-(1-\frac{v(n,t)}{n})}$, and if it is 1, decide to play a new arm, and otherwise decide to play an old arm. Note that

            \begin{align}
            P(\text{play a new arm}|\mathcal{F}_{t})=\left[1-\frac{v(n,t)}{n}\right]+\left[1-(1-\frac{v(n,t)}{n})\right]\times \frac{q_t-(1-\frac{v(n,t)}{n})}{1-(1-\frac{v(n,t)}{n})}=q_t.
            \end{align}

            Since we pull a new arm only when $Z_{n,v(n,t),m(n,t)}=1$, the number of times we pull an arm when $v(n,t)=v$ is bounded above by $\bar{T}_{n,v}$. Thus,

            \begin{align}
            t_{n+1}-t_{n}\leq \sum_{v=0}^{n-1}\bar{T}_{n,v}=\bar{T}_{n}.
            \end{align}
            Finally, we want to bound $\sum_{t}E[c(t)]$. Let $C(n)$ be the cost incurred in round $n$. Thus $\sum_{t}E[c(t)]=\sum_{n}E[C(n)]$. We know
            \begin{align}
            C(n)&=\sum_{t=t_n}^{t_{n+1}-1}c(t)=\sum_{t=t_{n}}^{t_{n+1}-1}\sum_{\ell}\mathbbm{1}\{\omega\in L(\ell)\}c(t) \nonumber \\
            &\leq \sum_{t=t_{n}}^{t_{n+1}-1}\sum_{l}\mathbbm{1}\{\omega\in L(\ell)\}\bar{c}(n,\ell) \nonumber \\
            &\leq (t_{n+1}-t_{n})\sum_{\ell} \mathbbm{1}\{\omega\in L(\ell)\}\bar{c}(n,\ell) \nonumber \\
            &\leq \bar{T}_{n}\sum_{l}\mathbbm{1}\{\omega\in L(\ell)\}\bar{c}(n,\ell). \nonumber
            \end{align}

            Since $\bar{T}_{n}$ is independent of $\omega$, we know

            \begin{align}
            E[C(n)]\leq E[\bar{T}_{n}]\sum_{\ell}P(\omega\in L(\ell))\bar{c}(n,\omega). \nonumber
            \end{align}

            \noindent\textbf{Step 3: Rewrite the total payment expression}.

            Based on the above notations, we can rewrite the cumulative payment as follows:
            \begin{align}
            \sum_{t=1}^{\infty}c(t) =\sum_{n=1}^{\infty}C(n)
            \leq  \sum_{n=1}^{\infty}\sum_{\ell=1}^{\infty}\bar{T}_{n}\mathbbm{1}\{\omega\in L(\ell)\}\bar{c}(n,\ell). \nonumber
            \end{align}

            Set $g(n,\ell)$ to be $\frac{2\sigma \ell}{n^{1/6}}$. Since if $|u_{i}^{j}-u_{i,t}^{j}|\leq \lambda$ is true $\forall i$, $\forall j$, then we know for those $\theta\in \{\theta:\theta\cdot u_{B(\theta)}-\max_{j\neq B(\theta)}\{\theta \cdot u_{j}\}> 2Wm\lambda\}$, they will correctly identify their best arm. Thus, if $|u_{i}^{j}-u_{i,t}^{j}|\leq \frac{2\sigma l}{n^{1/6}} \leq \frac{p^{-1}(\frac{p}{2})}{2Wm}$ $\forall i$ and $\forall j$, then the probability that an unincentivized agent would pull arm $i$ is at least $\frac{p}{2}$. Further, if time $t$ is in a round $n$ that satisfies $n^{-1}\leq p/2$, then our algorithm will not incentivize pulling any arms. Denote $a_0=\frac{4Wm\sigma}{p^{-1}(\frac{p}{2})}$. In order to have $\frac{2\sigma l}{n^{1/6}}\leq \frac{p^{-1}(\frac{p}{2})}{2Wm}$, it is sufficient to have $n\geq \lceil (a_{0} l)^6 \rceil$. In order to have $n^{-1}\leq \frac{p}{2}$, we need $n\geq \frac{2}{p}$. Denote $n_2=\frac{2}{p}$. Thus, we know we can only incur regret for sample paths $\omega$ in the $\ell^{th}$ envelope in the first $\max\{n_2,\lceil (a_0 \ell)^6 \rceil\}$ rounds.

            Thus,
            \begin{align}
            \sum_{t=1}^{\infty}c(t)\leq\sum_{\ell=1}^{\infty}\sum_{n=1}^{\max\{n_2,\lceil (a_0 \ell)^6 \rceil\}}\bar{c}(n,\ell)\mathbbm{1}\{\omega\in L[\ell]\}\bar{T}_{n}. \nonumber
            \end{align}

            Therefore,

            \begin{align}
            &E\left[\sum_{t=1}^{\infty}c(t)\right] \nonumber\\
            \leq &\sum_{\ell=1}^{\infty}\sum_{n=1}^{\max\{n_2,\lceil (a_0 \ell)^6 \rceil\}}\bar{c}(n,\omega)P(\omega \in L(\ell))E[\bar{T}_{n}] \nonumber \\
            =&\sum_{\ell=1}^{\infty}\sum_{n=1}^{\max\{n_2,\lceil (a_0 \ell)^6 \rceil\}}\left[R+2Wm\frac{2\sigma \ell}{n^{1/6}}\right]P(\omega\in L[\ell])E[\bar{T}_{n}] \nonumber \\
            \leq &\sum_{\ell=1}^{\infty}\sum_{n=1}^{\max\{n_2,\lceil (a_0 \ell)^6 \rceil\}}\left[R+4Wm\sigma l\right]P(\omega\in L[\ell])E[\bar{T}_{n}]. \nonumber
            \end{align}



            \noindent\textbf{Step 4: Bound $P(\omega\in L(\ell))$}.

            We now bound $P(\omega\in L[\ell])$ for $n\geq n_0$ and $\ell\geq 2$. As a reminder, we omit the dependency between $\epsilon_{i,t}^{j}$, $n$ and $\omega$. We know 
            \begin{align}
            &P(\omega\in L[\ell]) \nonumber \\
            =&P(\omega\in L^{'}[\ell])- P(\omega\in L^{'}[\ell-1]) \nonumber \\
            \leq & 1-P\left(|\epsilon_{i,t}^{j}|<\frac{2\sigma (\ell-1)}{n^{1/6}}, \forall i,j\right) \nonumber \\
            =&P\left(\exists i,j, s.t. |\epsilon_{i,t}^{j}|\geq \frac{2\sigma (\ell-1)}{n^{1/6}}\right) \nonumber  \\
            \leq &\sum_{i,j}P\left(|\epsilon_{i,t}^{j}|\geq \frac{2\sigma (\ell-1)}{n^{1/6}}\right) \nonumber
            \end{align}

            Define $S_{i,t}^{j}=\frac{N(i,t)\epsilon_{i,t}^{j}}{2\sigma}$, then we know $S_{i,t}^{j}$ is a summation of $1/2$ gaussian random numbers. Therefore,

            \begin{align}
            &\sum_{i,j}P\left(|\epsilon_{i,t}^{j}|\geq \frac{2\sigma(\ell-1)}{n^{1/6}}\right) \nonumber \\ 
            =&\sum_{i,j}P\left(|S_{i,t}^{j}|\geq \frac{N(i,t)(\ell-1)}{n^{1/6}}\right) \nonumber \\
            \leq &\sum_{i,j}P(|S_{i,t}^{j}|\geq N(i,t)^{5/6}(\ell-1)). \nonumber
            \end{align}

            Based on Lemma~\ref{lemma:cal2}, we know
            \begin{align}
            &N(i,t)^{5/6}(\ell-1) \nonumber \\
            =& 0.9N(i,t)^{5/6} + N(i,t)^{5/6}(\ell-1.9) \nonumber \\
            \geq & \sqrt{0.6N(i,t)\log(\log_{1.1}(N(i,t))+1)} + \sqrt{(\ell-1.9)^2 N(i,t)} \nonumber \\
            \geq & \sqrt{0.6N(i,t)\log(\log_{1.1}(N(i,t))+1)+(\ell-1.9)^2 N(i,t)}. \nonumber
            \end{align}

            Based on Lemma~\ref{ACI-inequality}, we know
            \begin{align}
            & P(\omega\in L(\ell) \nonumber \\
            \leq &\sum_{i,j}P(|S_{i,t}^{j}|\geq N(i,t)^{5/6}(\ell-1)) \nonumber \\
            \leq & \sum_{i,j}24e^{-1.8(\ell-1.9)^2} = 24Nme^{-1.8(\ell-1.9)^2}. \nonumber
            \end{align}

            \noindent\textbf{Step 5: Final Step}

            Thus,
            \begin{align}
            &\sum_{t=1}^{\infty}c(t) \nonumber \\
            \leq&\sum_{\ell=1}^{\infty}\sum_{n=1}^{\max\{n_2,\lceil (a_0 \ell)^6 \rceil\}}\left[R+4Wm\sigma \ell\right]P(\omega\in L[\ell])E[\bar{T}_{n}] \nonumber \\
            \leq& \sum_{n=1}^{\max\{n_2,\lceil (a_0)^6 \rceil\}}\left[R+4Wm\sigma \right]P(\omega\in L[1])E[\bar{T}_{n}] +\sum_{l=2}^{\infty}\sum_{n=1}^{\max\{n_2,\lceil (a_0 \ell)^6 \rceil\}}\left[R+4Wm\sigma \ell\right]P(\omega\in L[\ell])E[\bar{T}_{n}] \nonumber \\
            \leq& \sum_{n=1}^{\max\{n_2,\lceil (a_0)^6 \rceil\}}\left[R+4Wm\sigma \right]E[\bar{T}_{n}] +\sum_{\ell=2}^{\infty}\sum_{n=1}^{\max\{n_2,\lceil (a_0 \ell)^6 \rceil\}}\left[R+4Wm\sigma \ell\right]24Nme^{-1.8(\ell-1.9)^2}E[\bar{T}_{n}]\nonumber \\
            \leq& \sum_{n=1}^{\max\{n_2,\lceil (a_0)^6 \rceil\}}\left[R+4Wm\sigma \right]Nn +\sum_{\ell=2}^{\infty}\sum_{n=1}^{\max\{n_2,\lceil (a_0 \ell)^6 \rceil\}}\left[R+4Wm\sigma \ell\right]24Nme^{-1.8(\ell-1.9)^2}Nn \nonumber \\
            \leq & \left[R+4Wm\sigma \right]N (\max\{n_2,\lceil (a_0)^6 \rceil\})^2+\sum_{\ell=2}^{\infty}24N^2 m[R+4Wm\ell\sigma](\max\{n_2,\lceil (a_0 \ell)^6 \rceil\})^2 e^{-1.8(\ell-1.9)^2} \nonumber \\
            =& O(N^2). \nonumber
            \end{align}

            \end{proof}

            \section*{Calculation for Equation~\eqref{ex:limit}}

            In this section, we prove
            \begin{align}
            \lim_{t\rightarrow\infty} t\left[1-\frac{\sqrt{t}+z_{t,2}}{\sqrt{z_{t,1}^2+(\sqrt{t}+z_{t,2})^2}}\right]=\frac{z_{t,1}^2}{2}. \nonumber 
            \end{align}

            Denote $w(x)=\sqrt{x}$. Then $w(x+y)=\sqrt{x} + y\frac{1}{2\sqrt{x}}+O(x^{-3/2})$. Thus, we know

            \begin{align}
            &\lim_{t\rightarrow\infty} t\left[1-\frac{\sqrt{t}+z_{t,2}}{\sqrt{z_{t,1}^2+(\sqrt{t}+z_{t,2})^2}}\right] \nonumber \\
            =& \lim_{t\rightarrow\infty} t\left[\frac{\sqrt{z_{t,1}^2+(\sqrt{t}+z_{t,2})^2}-\sqrt{t}-z_{t,2}}{\sqrt{z_{t,1}^2+(\sqrt{t}+z_{t,2})^2}}\right]  \nonumber \\
            = & \lim_{t\rightarrow\infty} t\left[\frac{\sqrt{t}+z_{t,2}+z_{t,1}^{2}\frac{1}{2\sqrt{z_{t,1}^2+(\sqrt{t}+z_{t,2})^2}}+O((z_{t,1}^2+(\sqrt{t}+z_{t,2})^2)^{-3/2})-\sqrt{t}-z_{t,2}}{\sqrt{z_{t,1}^2+(\sqrt{t}+z_{t,2})^2}}\right]  \nonumber \\
            = & \lim_{t\rightarrow\infty} t \times \frac{z_{t,1}^2}{2(z_{t,1}^2+(\sqrt{t}+z_{t,2})^2)} \rightarrow \frac{z_{t,1}^2}{2}. \nonumber 
            \end{align}

            Therefore, we know Equation~\eqref{ex:limit} holds true.




\bibliography{reference}
\end{document}
