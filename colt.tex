\documentclass[twoside,11pt]{article}
\usepackage[anon]{jmlr2e}


% Any additional packages needed should be included after jmlr2e.
% Note that jmlr2e.sty includes epsfig, amssymb, natbib and graphicx,
% and defines many common macros, such as 'proof' and 'example'.
%
% It also sets the bibliographystyle to plainnat; for more information on
% natbib citation styles, see the natbib documentation, a copy of which
% is archived at http://www.jmlr.org/format/natbib.pdf


\usepackage[]{algorithm}
\usepackage[noend]{algorithmic}

%\usepackage[utf8]{inputenc} % allow utf-8 input
%\usepackage[T1]{fontenc}    % use 8-bit T1 fonts
%\usepackage{microtype}      % microtypography
%\usepackage{xr} 

\usepackage{url}            % simple URL typesetting
\usepackage{amsfonts}       % blackboard math symbols
\usepackage{amsmath}
\usepackage{nicefrac}       % compact symbols for 1/2, etc.
%\usepackage{amsthm}
%\usepackage{mathtools}

%\usepackage{comment}
%\usepackage{subcaption}
%\usepackage{xcolor}
%\usepackage{float}
\usepackage{bm}
\usepackage{bbm}
\usepackage{ifthen}
\usepackage{xspace}

\usepackage{color-edits}
% home-made package by David that provides the following macros:
% 1. \pfedit{}: prints argument in blue
% 2. \pfcomment{}: prints argument in blue in [Peter: #1]
% 3. \pfmargincomment{}: prints argument in blue in the margin as
% [Peter: #1]
% 4. \pfdelete{}: instead of text, marks a note in the margin in blue that says
% "Peter deleted here"
% Package has option [suppress], which gets rid of all color and all comments
% and [showdeletions], which shows deleted text in color/strikeout.
\addauthor[Peter]{pf}{blue}
\addauthor[Bangrui]{bc}{green}
\addauthor[David]{dk}{red}

%\newcommand{\bccomment}[1]{{\color{blue}BC: #1}}
%\newcommand{\pfcomment}[1]{{\color{blue}PF: #1}}

\providecommand{\SET}[1]{\ensuremath{\{ #1 \}}\xspace}
\providecommand{\Abs}[1]{\ensuremath{| #1 |}\xspace}
\providecommand{\Set}[2]{\ensuremath{\SET{#1 \mid #2}}\xspace}

\providecommand{\PROB}{\ensuremath{{\rm Prob}}\xspace}
\providecommand{\Prob}[2][]{\ensuremath{%
\ifthenelse{\equal{#1}{}}{\PROB[#2]}{\PROB_{#1}[#2]}}\xspace}
\providecommand{\ProbC}[3][]{\Prob[#1]{#2\;|\;#3}}
\providecommand{\Expect}[2][]{\ensuremath{%
\ifthenelse{\equal{#1}{}}{\mathbb{E}}{\mathbb{E}_{#1}}%
\left[#2\right]}\xspace}
\providecommand{\ExpectC}[3][]{\Expect[#1]{#2\;|\;#3}}

\newcommand{\argmax}{\mathop{\mathrm{argmax}}}
\newcommand\floor[1]{\lfloor#1\rfloor}
\newcommand\ceil[1]{\lceil#1\rceil}
\newcommand{\fracpartial}[2]{\frac{\partial #1}{\partial  #2}}
\newcommand{\R}{\mathbb{R}} % real numbers

\newtheorem{assumption}{Assumption}

% Definitions of handy macros can go here

\newcommand{\vc}[1]{\bm{#1}} % vectors

\newcommand{\Normal}[2]{\ensuremath{{\mathcal N}(#1,#2)}\xspace}
% normal distribution

\newcommand{\dataset}{{\cal D}} % used anywhere?

%%%%%%%%%%%%%%%%%%%%%%%%%%%%%%%%%%%%
% Agreed upon symbols without macros
%%%%%%%%%%%%%%%%%%%%%%%%%%%%%%%%%%%%

% dimension for attribute vector: d
% stopping time: \tau
% phase of the algorithm: s


% To define and add in appropriate spots

% \Delta for utility difference
% \bar{\zeta} for error in utility difference
% q_{\delta} for mass of users


%%%%%%%%%%%%%%%%%%%%%%%%%%%%%%%%%%%%%
% Stuff related to Arms and Arm Pulls
%%%%%%%%%%%%%%%%%%%%%%%%%%%%%%%%%%%%%

\newcommand{\ARMNUM}{\ensuremath{N}\xspace} % number of arms
\newcommand{\ArmV}[1]{\ensuremath{\vc{\mu}_{#1}}\xspace}
% vector for location of an arm

\newcommand{\Arm}[2]{\ensuremath{\mu_{#1,#2}}\xspace}
% individual entries of the ith arm's location vector

\newcommand{\NoiseV}[1][]{\ensuremath{\vc{\zeta}_{#1}}\xspace}
% vector of noise added to arm

\newcommand{\Noise}[2][]{\ensuremath{%
\ifthenelse{\equal{#1}{}}{\epsilon_{#2}}{\zeta_{#1,#2}}\xspace}}
% individual entries of noise vector

\newcommand{\ObsV}[1]{\ensuremath{\vc{y}_{#1}}\xspace}
% observation of an arm: location plus noise

\newcommand{\ArmEV}[2]{\ensuremath{\hat{mu}_{#1,#2}}\xspace} 
% empirical estimator of the arm location based on samples
% Argument 1: time step
% Argument 2: arm



%%%%%%%%%%%%%%%%%%%%%%%%%
% Stuff related to agents
%%%%%%%%%%%%%%%%%%%%%%%%%

\newcommand{\AgentV}[1]{\ensuremath{\vc{\theta}_{#1}}\xspace}
% vector for location of an agent

\newcommand{\Agent}[2]{\ensuremath{\theta_{#1,#2}}\xspace}
% individual entries of the agent location vector

\newcommand{\Best}[1]{\ensuremath{B_{#1}}\xspace}
% Best arm for a given agent

\newcommand{\Second}[1]{\ensuremath{B'_{#1}}\xspace}
% Second-best arm for a given agent

\newcommand{\FirstTwo}[2]{\ensuremath{\Omega_{#1,#2}}\xspace}
% Set of agents who have i as their first choice and i' as their second



%%%%%%%%%%%%%%%%%%%%%%%%%%%%%%%%%%%%%
% Stuff related to agent distribution
%%%%%%%%%%%%%%%%%%%%%%%%%%%%%%%%%%%%%

\newcommand{\AgentDist}{\ensuremath{f}\xspace}
% distribution of locations for agents

\newcommand{\Diam}{\ensuremath{D}\xspace}
% "diameter" of support of agent distribution, [0,D]^d

\newcommand{\MinProb}{\ensuremath{p}\xspace}
% minimum probability of support for any arm

\newcommand{\TieDensity}{\ensuremath{L}\xspace}
% upper bound on the derivative of the density of near-ties



%%%%%%%%%%%%%%%%%%%%%%%%%%%
% Stuff related to policies
%%%%%%%%%%%%%%%%%%%%%%%%%%%

\newcommand{\POLICY}{\ensuremath{\mathcal A}\xspace} % algorithm/policy
\newcommand{\Pay}[2]{\ensuremath{c_{#1,#2}}\xspace}
% payment offered for arms:
% Argument 1: time step
% Argument 2: arm

\newcommand{\PayA}[1]{\ensuremath{c_{#1}}\xspace}
% payment actually made in step t, i.e., c_{t,\Pull{t}}

\newcommand{\TotalPay}[1]{\ensuremath{C_{#1}}\xspace}
% cumulative payment made up to and including step t.

\newcommand{\Pull}[1]{\ensuremath{i_{#1}}\xspace}
% arm that was actually pulled at time step t

\newcommand{\NumPull}[2]{\ensuremath{m_{#1,#2}}\xspace}
% number of pulls of arm i up to time t:
% Argument 1: time step
% Argument 2: arm

\newcommand{\Regret}[1]{\ensuremath{r_{#1}}\xspace}
% regret incurred in time step t

\newcommand{\TotalRegret}[1]{\ensuremath{R_{#1}}\xspace}
% total regret incurred up to and including time step t



% Heading arguments are {volume}{year}{pages}{date submitted}{date published}{paper id}{author-full-names}

\jmlrheading{1}{2000}{1-48}{4/00}{10/00}{meila00a}{Marina Meil\u{a} and Michael I. Jordan}

% Short headings should be running head and authors last names

\ShortHeadings{Incentivizing Exploration with Heterogeneous Utilities}{Chen, Frazier and Kempe}
\firstpageno{1}

\begin{document}

\title{Incentivizing Exploration by Heterogeneous Users}
% {Incentivizing Exploration with Heterogeneous Utilities}

\author{\name Bangrui Chen \email bc496@cornell.edu \\
\name Peter I.\ Frazier \email pf98@cornell.edu \\
\addr Operations Research and Information Engineering\\
Cornell University\\
New York, NY 14850, USA
\AND
\name David Kempe \email david.m.kempe@gmail.com  \\
\addr Department of Computer Science\\
University of Southern California\\
Los Angeles, CA 90089, USA}

\editor{}

\maketitle

\begin{abstract}
sdafsd
\end{abstract}

\begin{keywords}
Incentivizing Exploration
\end{keywords}

     

\section{Introduction}

Many websites and apps are designed to facilitate joint discovery,
sharing, and recommendations of content.
Such sites include news, photo, and video sharing sites,
sites to review restaurants, hotels, or travel experiences,
online stores at which users write reviews (such as Amazon),
and citizen science projects
(such as eBird \citep{sullivan2009ebird,xue-ebird} or Galaxy Zoo \citep{lintott-galaxy-zoo}).
By learning from the experiences of other users, individuals can
improve their own experience \citep{schmit2017human}.

Viewed more abstractly, users jointly explore a space of
many options (products, news stories, photos, birdwatching sites,
patches of sky to train their telescopes on, $\ldots$),
with the implicit goal of identifying the ``best'' ones.
Therefore, such scenarios can be fruitfully modeled in a bandit
learning framework.
However, contrary to standard bandit settings, the utilities of the
decision makers (the users) are not aligned with the overall utility.
Societally (i.e., in a suitable aggregate over all users),
it would be desirable to engage in considerable exploration of
different options, so as to provide higher rewards for a large number
of future users.
However, individual users only interact with the site a limited number
of times, and therefore have little incentive for exploration.
A particularly clean model is obtained when each user interacts with the
site only once, and hence has no intrinsic utility for exploration. 
This model has been the subject of prior work, and forms the basis of
the present submission.

To effect an outcome close to societally optimal,
it is necessary to provide exploration incentives to the individual users.
This was noted in two recent lines of work:
\citet{kremer2014implementing}
and \citet{mansour2015bayesian,mansour2016bayesian}
assume that the site (also called the \emph{principal}) has an
informational advantage in being the only one to observe the results
of past arm pulls.
(Such an assumption applies, for instance, to route recommendations in
driving.)
The principal can exploit her%
\footnote{We use male pronouns to refer to users and female pronouns
  to refer to the principal.}
advantage and make recommendations to the individual agents that 
are in their best interest to follow.
\citet{frazier2014incentivizing} and 
\citet{han2015incentivizing} instead assume that the results of all
past arm pulls are publicly observable
(such as reviews on an online retail site).
They instead suppose the principal can offer payments to users as
a reward for pulling particular arms.
We follow the model of \citet{frazier2014incentivizing} and
\citet{han2015incentivizing} 
and consider a multi-armed bandit model in which the principal can
offer payments to the users for pulling particular arms.

Past work has assumed that users are homogeneous, i.e., the expected
reward a user derives from an arm is the same for all users.%
\footnote{\citet{han2015incentivizing} assume that users are
heterogeneous in their tradeoff between utility derived from arm pulls
and utility derived from the principal's payment.}
In reality, users have different preferences, e.g., gastronomic,
political, aesthetic, practical, etc.
Indeed, the websites and mobile apps most widely used for joint discovery,
sharing, and recommendation of content tend to concern products and items 
with large amounts of heterogeneity in preferences (movies, restaurants,
videos, travel experiences), and not items which have a universally
agreed-on best order.
This is perhaps because regimes with heterogeneous preferences are the
ones where discovery of the best items is the most difficult for people, 
and thus where online
platforms tend to provide the greatest value.  Thus, we see an appropriate
accounting for heterogeneity as critical to an understanding of incentivizing
exploration in online communities.

Heterogeneity presents both a challenge and an opportunity.
On the one hand, unobserved heterogeneity hides critical
information about an agent's preferences from the principal.
On the other hand, heterogeneity also presents her an opportunity,
through the possibility of ``free exploration.''
Even when left unincentivized, agents will
play a variety of arms, revealing information about their attributes.
This stands in sharp contrast to the case of homogeneous preferences,
where unincentivized agents will herd onto a single apparently best arm;
thus, effecting essentially any exploration at all requires incentives.

In order to take advantage of unobserved heterogeneity,
the principal has to give up some control, allowing the agent to
reveal his preferences through action, rather than obscuring them with
incentives.
However, she cannot give up control completely,
and must use incentives to force agents to explore against their
preferences, for the greater good. 

With these challenges and opportunities in mind, our goal is to understand the
impact of user heterogeneity on the principal's ability to achieve
high social utility with low incentive payments,
and on the best approaches for doing so.
We wish to understand whether incentivizing exploration with
heterogeneous preferences  is ``harder'' or ``easier'' than with
homogeneous ones,
and to understand how exploration strategies that do well in the
heterogeneous preference setting differ from those in the homogeneous one.

Toward that end, we model our setting as follows.
(We describe our model at a high level here,
with formal definitions given in Section~\ref{sec:prob}.)
The \ARMNUM arms and users (or \emph{agents})
are characterized by payoff-relevant
\emph{attribute} (or \emph{feature}) vectors.
Arms' attributes are a priori unknown,
and agents' attributes are drawn from a known distribution.
An agent's reward from pulling an arm is the inner product of his
vector with the arm's vector (plus noise).
When an arm is pulled, a noisy version of its attribute vector is
observed by everyone.
\pfedit{This models full-text product reviews on websites like Amazon, or ratings of ``service'', ``value'', and other restaurant attributes on websites like Tripadvisor.  }
Agents are myopic and will pull the arm whose expected attribute
vector (based on past noisy observations) maximizes their reward.
\pfedit{Agents' ability to effectively base decisions on all past observations models platforms' abilities to aggregate and effectively display past feedback, through simple averages of ratings and automatic summarization of full-text reviews \citep{wang2010product,liu2012movie,abulaish2009feature}.  While platforms can manipulate the display of user data, most have an incentive to be seen as truthful.}
The principal can incentivize agents to pull particular arms by
offering rewards for the specific arm.
The principal's goal is to keep the cumulative regret across all
agents small, while incurring only small total payments.
Our main theorem can be stated informally as follows:

%One example where agents give noisy observations of items’ feature vectors is restaurant reviews on TripAdvisor.  Users provide not just an overall rating, but also separate ratings for Service, Value, Food and Atmosphere.  We may treat these as our feature vector.  Platforms collect similar explicit feature vector observations whenever their users rate not just satisfaction with an item but the more granular components that contribute to it.

%As another example, consider websites like Amazon and Yelp whose users provide full-text reviews of items.  Reviews on Yelp describe a restaurants’ attributes (service, value, etc.), and reviews on Amazon similarly describe products’ attributes.  Thus, we can think of a user reading past reviews as aggregating past users’ observations of items’ feature vectors.  In this way, platforms with full-text reviews implicitly provide feature vector observations.

%As a third example, a platform with full-text reviews could run an NLP algorithm to extract features and sentiments from reviews, and then surface them to users as rated attributes.


\begin{theorem} \label{thm:main-intro}
Assume that for each arm, at least a constant fraction \MinProb
of the population likes this arm best,
and let \TieDensity be a measure of the ``density of near-ties''
between agents' arm preferences
(in a sense made precise in Section~\ref{subsec:discrete}).
There is a policy that achieves expected 
cumulative regret $O (\ARMNUM \e^{2/\MinProb} + \ARMNUM \log^3(T))$,
using expected cumulative payments of $O(\ARMNUM^2 \e^{2/\MinProb})$.
In particular, when agents who are close to tied between two arms have measure $0$,
both the expected regret and expected payment are bounded by constants
(with respect to $T$). 
\end{theorem}

The policy achieving the result of Theorem~\ref{thm:main-intro} is
quite simple: it mostly lets agents exploit arms, but incentivizes
them to explore when arms appear unlikely to be pulled without incentives.
It is presented in detail in Section~\ref{sec:ub}.


\section{Problem Setting}
\label{sec:prob}

We have $N$ arms. Arm $i$ has a fixed but unknown attribute vector $u_i\in \mathbb{R}^{m}$. 
A stream of myopic selfish agents come to our system.  Agent $t$ has linear preferences over attributes described by a vector $\theta_t \in \mathbb{R}^m$ that is unknown to the principal and drawn i.i.d. from a known distribution $F(\cdot)$ with compact support. We refer synonomously to an agent and that agent's preference vector: when we say ``an agent $\theta$'', we mean ``an agent with preference vector $\theta$.''

Each agent $t$ chooses an arm to pull $A_t$, according to a process described below, and obtains utility $\theta_t \cdot u_{A_{t}}$.  The principal and all agents then see a noisy observation of the attribute vector of the pulled arm of the form $O_t=u_{A_{t}}+\epsilon_{t}$, where $\epsilon_t\sim N(0, \sigma^2 I_{m})$ is independent normally distributed noise, and $I_m$ denotes an $m$-dimensional identity matrix.  Although we assume a common variance across attributes for simplicity of presentation, our theoretical results hold if the variance differs.

At each time $t$, for each arm $i$, we (the principal) offer a non-negative payment $c_{i,t}\geq 0$ based on previous observations.
We assume that agent $t$ chooses to pull the arm that myopically maximizes the sum of this payment and an estimate of the utility obtained $\theta_t \cdot u_{i,t}$ where $u_{i,t}$ denotes the simple average of $O_s$ over all previous pulls of arm $i$. In this paper, we assume all arms have been pulled once at time $t=0$ and $u_{i,0}$ denotes a random draw from the arm attribute vector. For $t>0$, denote $u_{i,t} = \frac{\sum_{s<t} O_s 1\{A_s = i\} + u_{i,0}}{\sum_{s<t} 1\{A_s = i\}+1}$ and $A_t=\argmax_{i}\{\theta_t\cdot u_{i,t}+c_{i,t}\}$, breaking ties in favor of the arm with the highest incentive.  We use $c_t=c_{A_{t},t}$ to denote the actual incentive payment at time $t$.

This behavior may be recovered if agents are Bayesian and share a common non-informative prior distribution that is constant over $\mathbb{R}^m$ and know $\sigma^2$.  In this case, the posterior distribution on $u_{i}$ at time $t$ is multivariate normal with mean $u_{i,t}$, and the expected value of $\theta_t \cdot u_i$ under this posterior conditioned on $\theta_t$ is $\theta_t \cdot u_{i,t}$ (see equation $2.13$ in section $2.5$, \cite{Ge04}).  Alternatively, one may simply take our assumption that agents use the average as their estimate of an attribute value directly without such a Bayesian justification.

We define the regret at time $t$ as $r(t)=\max_{i}\{\theta_{t}\cdot u_{i}\}-\theta_t\cdot u_{A_t}$, and the cumulative regret up to time T as $R(T)=\sum_{t=1}^{T}r(t)$. Define the cumulative payment up to time T similarly as $C(t)=\sum_{t=1}^{T}c(t)$. 
As the principal, we want to find a strategy $\mathcal{A}$ under which both the cumulative expected regret $\mathbb{E}_{\mathcal{A}}[R(T)]$ and the cumulative expected payment $\mathbb{E}_{\mathcal{A}}[C(T)]$ are small.

To support later development, we define some additional notation.
We let $B(\theta)$ and $\hat{B}(\theta)$ refer to the index of the arm that is best and second best for an agent with preference vector $\theta$, $B(\theta) \in \argmax_i \theta \cdot u_i$ and $\hat{B}(\theta)=\argmax_{i\neq B(\theta)}\theta\cdot u_{i}$, breaking ties uniformly at random. We let $N(i,t)$ denote the number of pulls of arm $i$ at times up to and including $t$ plus $1$ (because of the initial pull), i.e. $N(i,t)=\sum_{s<t} 1\{A_s = i\}+1$.
We call time $t_{n}=\min_{t}\{\forall i, N(i,t)\geq n\}$ the {\it starting point of the $n^{th}$ round}. We call the set of times $[t_{n}, t_{n+1})$ the {\it $n^{th}$ round}.




\section{Overview of Results and Discussion}

\dkcomment{In the results discussion clarify which results need to
  know $T$ and which don't.}

%
% State the main theorem(s) formally.
%

We will show below that our algorithm (described formally below as Algorithm~\ref{alg:basic-incentivizing}) achieves $O \left(\MAXR\cdot  N\exp(\frac{2}{p})\right) + O(\ARMNUM\Diam^2 d^2\sigma^2 \TieDensity(\log(T))^3)$ cumulative regret with $O \left(\ARMNUM^2 \MAXR \Diam^2 \TieDensity d^3 \sigma\cdot \exp(\frac{2}{\MinProb}) \right)$ payment budget.  We state these results formally, which together constitute our main results.

%\begin{theorem}
%The payment budget for Algorithm~\ref{alg:basic-incentivizing} is bounded above by 
%\begin{align}
%O \left(\ARMNUM^2 \MAXR \Diam^2 \TieDensity d^3 \sigma\cdot \exp(\frac{2}{\MinProb}) \right). \nonumber 
%\end{align} 
%\label{rst:budget}
%\end{theorem}

%\begin{theorem}
%For any given $T$, the cumulative regret for Algorithm~\ref{alg:basic-incentivizing}
%is bounded above by 
%\begin{align}
%O \left(\MAXR\cdot  N\exp(\frac{2}{p})\right) +
%O(\ARMNUM\Diam^2 d^2\sigma^2 \TieDensity(\log(T))^3).
%\end{align}
%\label{rst:regret}
%\end{theorem}

% Discuss lower bounds or impossibility results.
We also show a lower bound of $\Omega(\log(T))$ (section~\ref{sec:lb}) on the regret of any algorithm when agents have a density. 

In the specific case of discrete distributions over agent preferences, in which $L=0$ \pfcomment{check whether this is the right description}, we obtain tighter bounds:

%Discuss extensions.

% Discuss the justification or comparison to standard bandits.
We compare these results to those possible for standard bandits. In standard bandits, the payment is obviously $0$, and the regret scales as $N \sqrt{T}$ \pfcomment{add cite}.

% Discuss the comparison in the 2x2 square.

\pfcomment{Below this is old text}

As we discuss in more detail in Section~\ref{sec:main-discussion},
the reason we obtain stronger regret bounds than in the standard bandit
setting is that the agents' heterogeneity in preferences will
naturally lead to exploration.

Here we pursue an alternate direction to our analysis of incentivizing exploration with heterogeneous preferences: we consider approximation-style results comparing achievable algorithm performance across four different problem settings, described in the table below.

\begin{center}
\begin{tabular}{ c|c|c| } 
 \hline
     & Pulls arms directly & Needs to incentivize \\ 
\hline
 Perfect Info on Preferences & $C11$: The ``God'' Policy & $C12$ \\ 
 Partial Info on Preferences & $C21$ & $C22$: actual algorithm \\ 
 \hline
\end{tabular}
\end{center}

Columns indicate whether the principal pulls arms directly (left column) or incentivizes agents who in turn actually pull the arms. Rows indicate whether the principal knows the preferences of the agents (top row) or not (bottom row).   Problem settings correspond to cells and are described by the names C11, C12, and so on. An optimal algorithm for problem setting C11 has the best performance possible.

We wish to make two high-level points through this analysis:

\begin{itemize}
\item First, better performance is possible in setting C22 than in setting C21, in the sense we can find tight approximation guarantees relative to C11 that are better for C22 than for C21.  This is achieved by setting up incentives in C22 that still allow agents to select arms aligned with their preferences, providing selections of arms that contain more information than in C21.
\item Second, heterogeneity in preferences improves performance over what can be achieved in a setting with homogeneous preferences.  This would be in the spirit of David K.'s results that heterogeneity in preference for money can be exploited to reduce the budget required.
\end{itemize}



Section~\ref{sec:lb} constructs an example showing regret is $\Omega(\log(T))$ in the worst case, regardless of incentive budget.




\section{Algorithm and Upper Bound}
\label{sec:ub}

In this section, we propose a simple policy that mostly exploits, and occasionally incentivizes exploration when the probability of an arm would be pulled by all agent types below a time-varying threshold given the current posterior. We prove that with the help of heterogeneous preferences, we can get a certain amount of exploration for free via heterogeneity. 

\subsection{Our Algorithm}
Our algorithm incentivizes pulling an arm $i$ at a time $t$ in round $n$ if and only if both of the following criteria are met:
\begin{itemize}
\item the probability of pulling arm $i$ would be below $n^{-1}$ without incentives; 
\item arm $i$ has not been played previously in the current round.
\end{itemize}
Ties are broken randomly.  This algorithm does not need to know the horizon $T$ in advance. 

If our algorithm decides to incentivize an arm $i$, it uses the ``pay whatever it takes'' strategy in which the payment offered is $\max_{\theta,j} \theta \cdot (u_{j,t} - u_{i,t})$. This maximum over $\theta$ is taken over the support of $F$, which we recall is assumed compact.  (We use this ``pay whatever it takes'' strategy for its simplicity, and in Section~\ref{sec:pi} we provide an alternate and smaller incentive payment that achieves the same payment budget bound and regret bound). 

We describe our algorithm in detail as follows:


\begin{algorithm}
\caption{Algorithm: Incentivizing Exploration}
\label{Alg1}
\begin{algorithmic}
\STATE Set n = 1 to denote the round number; Let $V=\emptyset$ be the set of arms that were pulled in the current round;
\FOR{ $t = 1, 2, 3, \cdots$}{
  \STATE Let $S = \{ i : P( \theta \cdot u_{i,t} > \theta \cdot u_{j,t}\ \forall j\ne i | u_{j,t}\ \forall j) < n^{-1}\}$ be the set of arms with unincentivized probability of being pulled below $n^{-1}$.
    \IF {$S\setminus V$ is non-empty}
  \STATE{Choose an arm $i$ uniformly at random from $S\setminus V$}
  \STATE{Pay whatever it takes to incentivize pulling arm $i$, i.e., offer payment 
    $c_{i,t} = \max_{\theta,j} \theta \cdot (u_{j,t} - u_{i,t})$ and $c_{j,t} = 0$ for $j \ne i$.}
  \ELSE
    \STATE {Let agents play myopically, i.e., offer payment $c_{j,t} = 0$ for all $j$}
  \ENDIF
    \STATE Denote $A_t$ as the pulled arm, update $V=V\cup\{A_t\}$, $u_{A_t,t}$ and $N(A_t,t)$
    \IF {$n\neq \min_{i}N(i,t)$} 
  \STATE $V=\emptyset$
    \ENDIF
    \STATE Update the round number, $n = \min_{i} N(i,t)$
}\ENDFOR

\end{algorithmic}
\end{algorithm}


\subsection{Assumptions}
In this section, we state several assumptions assumed by our analysis.  First define
\begin{align}
\Omega_{i,j}=\{\theta:B(\theta)=i, \hat{B}(\theta)=j\}, \nonumber 
\end{align}
which is the set of agent preferences whose best arm is arm $i$ and second best arm is arm $j$. With this definition, our analysis makes the following assumptions:

\begin{assumption} Let $F_{i,j}(y)$ be the marginal cumulative density function (or cumulative mass function if $F(\cdot)$ is a discrete distribution) of $(u_i-u_j)\cdot\theta$ conditioned on $\theta \in \Omega_{i,j}$. We assume $F_{i,j}(y)\leq My$ for all $y\in R^{+}$, $\forall i,j$. 
\label{A1}
\end{assumption}

As we can see later in our proof, we only need $\max_{i,j}\limsup_{y\rightarrow 0^{+}}\frac{F_{i,j}(y)}{y}$ to be finite. Intuitively, assumption~\ref{A1} states that there are not many agents who are indifferent between their best arm and the second best arm. 

\begin{assumption} We assume $F$ has a compact support set. Without loss of generality, we assume $\theta\in [0,W]^m$.
\label{A2}
\end{assumption}

We use $R = \max_{\theta, i,j} \theta \cdot (u_i - u_j)$ to denote the maximum regret that can be incurred at each time.  Assumption~\ref{A2} shows that $R<\infty$.

\begin{assumption}
Denote $p=\min_{i}P(\{\theta: B(\theta)=i\})$. We assume $p>0$.
\label{A3}
\end{assumption}

Assumption~\ref{A3} means each arm $i$ has a strictly positive proportion of users for which that arm is best. 

\subsection{General Results}

In this section, we prove Algorithm~\ref{Alg1} achieves $O(N^2+M(\log(T))^2)$ cumulative regret with $O(N^2)$ payment budget.  This is stated in the following pair of theorems, which together constitute our main results.

\begin{theorem}
The payment budget for Algorithm~\ref{Alg1} is bounded above by $O(N^2)$. 
\label{rst:budget}
\end{theorem}


\begin{theorem}
The cumulative regret for Algorithm~\ref{Alg1} is bounded above by $O(N^2 m + M m^2(\log(T))^2)$.
\label{rst:regret}
\end{theorem}



Before we prove these two theorems, we must first introduce two additional pieces of notation, which will be used in preliminary lemmas.  Let $S(\delta)$ be the proportion of users whose utility difference between their best and second best arm is less than $\delta$. Formally, $S(\delta)=P(\theta: \theta \cdot u_{B(\theta)}-\theta\cdot u_{\hat{B}(\theta)}\leq \delta)$. Then, let $p(\delta)=\min_{i}P(\{\theta:B(\theta)=i,\theta\cdot u_{B(\theta)}-\theta\cdot u_{\hat{B}(\theta)}>\delta\})$. We know $p(0)=p$. 

With this additional notation, we now prove several lemmas.
First, based on Assumption~\ref{A1}, we have the following bound for $S(\delta)$.

\begin{lemma}
$S(\delta)\leq M\delta$.
\label{lemma:sdelta}
\end{lemma}

\begin{proof}
\begin{align*}
S(\delta)&=\sum_{i,j}P(\theta\cdot(u_{i}-u_{j})\le \delta|\theta\in \Omega_{i,j})P(\theta\in \Omega_{i,j}) \\
&\leq \sum_{i,j}M\delta \times P(\theta\in \Omega_{i,j}) \\
&=M\delta.
\end{align*}
\end{proof}

The following lemma bounds the probability of making a mistake if we let the agents play myopically in the $n^{th}$ round, given that the utility difference between his/her best and second best arm is bounded below by a constant. 

\begin{lemma}
Define $\tau$ to be any stopping time that is almost surely between $t_n$ and $t_{n+1}-1$ with respect to the filtration $\mathcal{F}_{t}=\sigma(A_1,\cdots,A_t,c_1,\cdots,c_t,O_1,\cdots,O_t)$, we have 
\begin{align}
P(\arg\max\{\theta_{\tau}\cdot u_{i,\tau}\}\neq B(\theta_{\tau})|\theta_{\tau}\cdot(u_{B(\theta_{\tau})}-u_{\hat{B}(\theta_{\tau})})> 2Wm\lambda)\leq 24Nm\exp\left(-\frac{1.8n\lambda^2}{16\sigma^2}\right), \nonumber
\end{align}
for $n\geq n_{0}=\max\{50, \frac{92.16\sigma^4}{\lambda^4}\}$.
\label{round:prob}
\end{lemma}


We need the following lemma in order to prove Lemma~\ref{round:prob} and we include the proof in the appendix.

\begin{lemma}
For $n\geq n_{0}=\max\{50, \frac{92.16\sigma^4}{\lambda^4}\}$, we have
\begin{align}
\frac{n\lambda}{4\sigma}\geq \sqrt{0.6n\log(\log_{1.1}(n)+1)}. \nonumber
\end{align}
\label{n0-inequality}
\end{lemma}

To prove Lemma~\ref{round:prob}, we also need to use an adaptive concentration inequality due to \cite{zhao2016adaptive}. For reference, we state it here as a Lemma.

\begin{lemma}[Corollary 1 in \cite{zhao2016adaptive}]
Let $X_{i}$ be zero mean $1/2$-subgaussian random variables. $\{S_{n}=\sum_{i=1}^{n}X_{i},n\geq 1\}$ be a random walk. Let $J$ be any stopping time with respect to $\{X_1,X_2,\cdots\}$. We allow $J$ to take the value of $\infty$ where $P(J=\infty)=1-\lim_{n\rightarrow \infty}P(J\leq n)$. If
\begin{align}
f(n)=\sqrt{0.6n\log(\log_{1.1}(n)+1)+bn}, \nonumber
\end{align}
then
\begin{align}
Pr[\{S_{J}\geq f(J)\}\cap \{J<\infty\}]\leq 12e^{-1.8b}. \nonumber
\end{align}
\label{ACI-inequality}
\end{lemma}


We now prove Lemma~\ref{round:prob}.

\begin{proof}[Proof of Lemma~\ref{round:prob}]
In the $n^{th}$ round, we know all arms have been pulled at least $n$ times. For all the agents $\theta$ whose utility difference between their best and second best arm is greater than $2mW\lambda$, denote $K(\theta)=\max_{i\neq B(\theta)}\{\theta\cdot u_{i,t}\}$. If $|u_{i,t}^{j}-u_{i}^{j}|\leq \lambda$ for all $i,j$, then
\begin{align}
&\theta\cdot(u_{B(\theta),t}-u_{K(\theta),t}) \nonumber \\
\geq & \theta\cdot(u_{B(\theta),t}-u_{B(\theta)}) + \theta\cdot(u_{K(\theta)}-u_{K(\theta),t}) + \theta\cdot(u_{B(\theta)}-u_{K(\theta)}) \nonumber \\
> & -Wm\lambda - Wm\lambda + 2Wm\lambda = 0,\nonumber
\end{align}
which means their myopic action would incur no regret.


Define $\epsilon_{i,\tau}=u_{i,\tau}-u_i$ and $\epsilon_{i,\tau}^{j}$ to be the $j^{th}$ component of $\epsilon_{i,\tau}$. Thus, we have
\begin{align}
&P(\arg\max\{\theta_{\tau}\cdot u_{i,\tau}\}\neq B(\theta_{\tau})|\theta_{\tau}\cdot(u_{B(\theta_{\tau})}-u_{\hat{B}(\theta_{\tau})})> 2Wm\lambda)\nonumber \\
\leq &P(\exists i, \exists j, |u_{i,\tau}^{j}-u_{i}^{j}|\geq\lambda |\theta_{\tau}\cdot(u_{B(\theta_{\tau})}-u_{\hat{B}(\theta_{\tau})})> 2Wm\lambda) \nonumber \\
\leq & \sum_{i}\sum_{j} P(|u_{i,\tau}^{j}-u_{i}^{j}|\geq\lambda|\theta_{\tau}\cdot(u_{B(\theta_{\tau})}-u_{\hat{B}(\theta_{\tau})})> 2Wm\lambda) \nonumber \\
= &  \sum_{i}\sum_{j} P(|\epsilon_{i,\tau}^{j}|\geq\lambda|\theta_{\tau}\cdot(u_{B(\theta_{\tau})}-u_{\hat{B}(\theta_{\tau})})> 2Wm\lambda). \label{ACI}
\end{align}

To bound equation~(\ref{ACI}), we use Lemma~\ref{ACI-inequality}. Define
\begin{align}
S_{N(i,\tau)}^{i,j}=\frac{\epsilon_{i,\tau}^{j}}{2\sigma}. \nonumber
\end{align}

Based on Lemma~\ref{n0-inequality}, for $n_{0}=\max\{50,\frac{92.16\sigma^2}{\lambda^2}\}$ and $n\geq n_{0}$, we have
\begin{align}
\frac{n\lambda}{4\sigma}\geq \sqrt{0.6n\log(\log_{1.1}(n)+1)}. \nonumber
\end{align}

Thus, if we set $b=\frac{n\lambda^2}{16\sigma^2}$ in Lemma~\ref{ACI-inequality}, for any $N(i,\tau)\geq n\geq n_{0}$, we have
\begin{align}
\frac{N(i,\tau)\lambda}{2\sigma}\geq & \sqrt{0.6N(i,\tau)\log(\log_{1.1}(N(i,\tau))+1)}+\frac{\lambda}{4\sigma}\sqrt{n N(i,\tau)} \nonumber \\
\geq & \sqrt{0.6N(i,\tau)\log(\log_{1.1}(N(i,\tau))+1)+bN(i,\tau)}, \nonumber 
\end{align}
where the last inequality is because $\sqrt{x}+\sqrt{y}\geq \sqrt{x+y}$. Thus, we have
\begin{align}
&P(\epsilon_{i,\tau}^{j}\geq\lambda|\theta_{\tau}\cdot(u_{B(\theta_{\tau})}-u_{\hat{B}(\theta_{\tau})})> 2Wm\lambda) \nonumber \\
=&P\left(S_{N(i,\tau)}^{i,j}\geq \frac{N(i,\tau)\lambda}{2\sigma}\bigg|\theta_{\tau}\cdot(u_{B(\theta_{\tau})}-u_{\hat{B}(\theta_{\tau})})> 2Wm\lambda\right) \nonumber \\
\leq & P\left(S_{N(i,\tau)}^{i,j}\geq \sqrt{0.6 N_{i,\tau}\log(\log_{1.1}(N(i,\tau))+1)+b N(i,\tau)}\bigg|\theta_{\tau}\cdot(u_{B(\theta_{\tau})}-u_{\hat{B}(\theta_{\tau})})> 2Wm\lambda\right) \nonumber \\
\leq & 12\exp( -1.8b) = 12\exp\left(\frac{-1.8 n\lambda^2}{16\sigma^2}\right). \nonumber
\end{align}
Similarily, we can bound 
\begin{align}
&P(\epsilon_{i,\tau}^{j}\leq-\lambda|\theta_{\tau}\cdot(u_{B(\theta_{\tau})}-u_{\hat{B}(\theta_{\tau})})> 2Wm\lambda) \nonumber \\
=&P(-\epsilon_{i,\tau}^{j}\geq \lambda|\theta_{\tau}\cdot(u_{B(\theta_{\tau})}-u_{\hat{B}(\theta_{\tau})})> 2Wm\lambda) \nonumber \\
\leq & 12\exp\left(\frac{-1.8 n\lambda^2}{16\sigma^2}\right). \nonumber 
\end{align}

Therefore, we know $P(|\epsilon_{i,\tau}^{j}|\geq \lambda|\theta_{\tau}\cdot(u_{B(\theta_{\tau})}-u_{\hat{B}(\theta_{\tau})})> 2Wm\lambda)\leq 24\exp\left(\frac{-1.8 n\lambda^2}{16\sigma^2}\right)$. Thus, we know
\begin{align}
&\sum_{i}\sum_{j} P(|\epsilon_{i,\tau}^{j}|\geq \lambda|\theta_{\tau}\cdot(u_{B(\theta_{\tau})}-u_{\hat{B}(\theta_{\tau})})> 2Wm\lambda)  \nonumber \\
\leq& 24Nm \exp\left(\frac{-1.8 n\lambda^2}{16\sigma^2}\right). \nonumber
\end{align}
\end{proof}

Before we start analyzing the cumulative regret, we first prove the following lemma which bounds the expected length of each round.
\begin{lemma}
Using our algorithm, we have $\mathbb{E}[t_{n+1}-t_{n}]\leq Nn$, $\forall n\geq 1$.
\label{round:length}
\end{lemma}
\begin{proof}
A round completes when each arm is pulled at least once in that round. Let $X_{i}$ be the number of agents who come to the system between the time after the $(i-1)^{th}$ unique arm was pulled, up to and including the time when the $i^{th}$ unique arm was pulled. Then we know 
\begin{equation*}
\mathbb{E}[t_{n+1}-t_{n}]=\sum_{i=1}^{N}E[X_{i}].
\end{equation*}


Fix $i$. In bounding $X_i$, we think of agents as ``trials'', where each trial can result in a new unique arm being pulled (which we call a ``successful'' trial), or not.  There are two ways a trial can be successful:
\begin{itemize}
\item If there is at least one arm that has not been pulled and the probability of an agent utility function that would pull this arm without incentives is less than $n^{-1}$, then the principal will offer an incentive that causes this arm to be pulled (or one of these arms if there is more than one). In this case, the probability that the trial is succesful is $1$.  
\item The probability of an agent utility function that would pull each un-pulled arm without incentives is at least $n^{-1}$. In this case, the probability that the trial is successful is at least $n^{-1}$.
\end{itemize}

Thus, $X_{i}$ is stochastically dominated below by a geometric random variable with success probability $n^{-1}$, the expected number of trials up to and including the first success, $E[X_i]$, is bounded above by $n$.  Thus,
\begin{align}
E[t_{n+1}-t_{n}]\leq Nn. \nonumber
\end{align}
\end{proof}
                  
Using the above lemmas, we are ready to prove Theorem~\ref{rst:budget}. We include the proof of Theorem~\ref{rst:budget} in the appendix. Below, we bound the expected number of payments for our algorithm.
\begin{lemma}
The expected number of payments for Algorithm~\ref{Alg1} is bounded above by $O(N^2)$.
\label{lemma:numP}
\end{lemma}
\begin{proof}
If $|u_{i}^{j}-u_{i,t}^{j}|\leq \lambda$ is true $\forall i$, $\forall j$, then we know for those $\theta\in \{\theta:\theta\cdot u_{B(\theta)}-\max_{j\neq B(\theta)}\{\theta \cdot u_{j}\}> 2mW\lambda\}$, they will correctly identify their best arm. Thus we know, in the $n^{th}$ round, if $|u_{i}^{j}-u_{i,t}^{j}|\leq \frac{p^{-1}(\frac{p}{2})}{2Wm}$ $\forall i$ and $\forall j$, and $n^{-1}\leq p/2$, we do not need to incentivize any arms. In order to have $n^{-1}\leq \frac{p}{2}$, we need $n\geq \frac{2}{p}$. Denote $n_1=\max\{n_{0}, \frac{2}{p}\}$. Denote $\delta_{0}=p^{-1}(\frac{p}{2})>0$ (because of Assumption~\ref{A1}).

Define $\tau_{n}^{i}$ to be the first time we pull arm $i$ in the $n^{th}$ round. Then
\begin{align}
\sum_{t=1}^{\infty}\mathbbm{1}\{c(t)>0\} =\sum_{n=1}^{\infty}\sum_{i=1}^{N}\mathbbm{1}\{c(\tau_{n}^{i})>0\}. \nonumber
\end{align}

The cumulative expected number of payments is bounded above by:
\begin{align}
&E\left[\sum_{t=1}^{\infty}\mathbbm{1}\{c(t)>0\}\right] \nonumber \\
=&\sum_{n=1}^{\infty}\sum_{i=1}^{N}P(c(\tau_{n}^{i})>0) \nonumber \\
\leq &\sum_{n=n_{1}}^{\infty}\sum_{i=1}^{N}P\left(\exists i,j:|u_{i}^{j}-u_{i,\tau_{n}^{i}}^{j}|>\frac{p^{-1}(\frac{p}{2})}{2Wm}\right)+\sum_{n=1}^{n_1}N \nonumber \\
\leq &\sum_{n=n_{1}}^{\infty}\sum_{i=1}^{N}24Nm \exp\left(\frac{-1.8 n\delta_{0}^2}{64W^2 m^2\sigma^2}\right) +\sum_{n=1}^{n_1}N \nonumber \\
\leq & \sum_{n=n_{1}}^{\infty}24Nm \exp\left(\frac{-1.8 n\delta_{0}^2}{64W^2 m^2\sigma^2}\right)\times N+\sum_{n=1}^{n_1}N \nonumber  \\
\leq&24N^2 m \frac{1}{\exp(\frac{1.8\delta_{0}}{64W^2 m^2\sigma^2})-1} + Nn_1, \nonumber
\end{align}

Thus, we know the expected number of payments is bounded above by $O(N^2)$.

\end{proof}

Now we are ready to prove our second main result, Theorem~\ref{rst:regret}.
\begin{proof}
For regret incurred in the first $n_0$ round, it is bounded above by $\sum_{n=1}^{n_{0}}NRn$.

For regret incurred after the first $n_0$ round, it has two different components: the regret incurred when we let the agents play myopically and the regret incurred when we incentivize the agents. Using Lemma~\ref{lemma:numP}, the expected regret incurred when we incentivize the agents is bounded above by: $\left[24N^2 m \frac{1}{\exp(\frac{1.8\delta_{0}}{64W^2 m^2\sigma^2})-1} + Nn_1\right]R$.

For the regret incurred when we let the agents play myopically at time $t\geq t_{n_0}$, it consists of the following two components:
\begin{itemize}
\item For those users whose utility difference between their best and the second best arm is greater than $f(t)$: we define a sequence of stopping time $\tau_{n}^{k}$ to be the $k^{th}$ time period in the $n^{th}$ round. For $k>t_{n+1}-t_{n}$, we define $\tau_{n}^{k}=\infty$. For $\tau_{n}^{k}=t$, the probability of these users making a mistake is bounded above by $24Nm\exp\left(-\frac{1.8n f(\tau_{n}^{k})^2}{64 W^2 m^2\sigma^2}\right)$ and the expected regret is bounded above by $24Nm\exp\left(-\frac{1.8n f(\tau_{n}^{k})^2}{64 W^2 m^2\sigma^2}\right)\times R$. We denote the regret incurred by these agents as $r_1(\tau_{n}^{k})$. For $k>t_{n+1}-t_{n}$, we define $r_1(\tau_{n}^{k})=0$.
\item For those user whose utility difference between their best and the second best arm is smaller than $f(t)$: this happens with probability $S(f(t))$ at each time and regret is bounded above by $S(f(t)) \times f(t)=Mf(t)^2$. We denote the regret incurred by these agents as $r_2(t)$.
\end{itemize}

Thus, the cumulative expected regret incurred up to time $T$ when we let the agent play myopically is bounded above by:
\begin{align}
&E\left[\sum_{t=1}^{T}r(t)\right] \nonumber \\
=&E\left[\sum_{t=1}^{t_{n_{0}}} r(t) + \sum_{t=t_{n_{0}}}^{T}(r_1(t)+r_2(t))\right]  \nonumber \\
\leq & \sum_{n=1}^{n_{0}}NRn + E\left[\sum_{n=n_{0}}^{T}\sum_{t=t_{n}}^{t_{n+1}-1}r_1(t)\right]+ E\left[\sum_{t=1}^{T}r_2(t)\right] \nonumber \\
=& \sum_{n=1}^{n_{0}}NRn + E\left[\sum_{n=n_{0}}^{T}\sum_{k=1}^{\infty}r_1(\tau_{n}^{k})\right]+ E\left[\sum_{t=1}^{T}r_2(t)\right]. \label{chap5:equ:r1}
\end{align}
Since
\begin{align}
& E\left[\sum_{n=n_{0}}^{T}\sum_{k=1}^{\infty}r_1(\tau_{n}^{k})\right] \nonumber\\
= &  \sum_{n=n_{0}}^{T}\sum_{k=1}^{\infty}E[r_1(\tau_{n}^{k})] \nonumber \\
= &  \sum_{n=n_{0}}^{T}\sum_{k=1}^{\infty}(E[r_1(\tau_{n}^{k})|\tau_{n}^{k}<\infty]\times P(\tau_{n}^{k}<\infty)+E[r_1(\tau_{n}^{k})|\tau_{n}^{k}=\infty]\times P(\tau_{n}^{k}=\infty)) \nonumber \\
= & \sum_{n=n_{0}}^{T}\sum_{k=1}^{\infty}E[r_1(\tau_{n}^{k})|\tau_{n}^{k}<\infty]\times P(\tau_{n}^{k}<\infty), \nonumber
\end{align}

we have
\begin{align}
\eqref{chap5:equ:r1}= & \sum_{n=1}^{n_{0}}NRn + \sum_{n=n_{0}}^{T}\sum_{k=1}^{\infty}E[r_1(\tau_{n}^{k})|\tau_{n}^{k}<\infty]\times P(\tau_{n}^{k}<\infty) + E\left[\sum_{t=1}^{T}r_2(t)\right] \nonumber \\
\leq & \sum_{n=1}^{n_{0}}NRn + \sum_{n=n_{0}}^{T}\left[\sum_{k=1}^{\infty}24Nm\exp\left(-\frac{1.8n f(\tau_{n}^{k})^2}{64 W^2 m^2\sigma^2}\right) R\times P(\tau_{n}^{k}<\infty) \right]+ \sum_{k=1}^{T}Mf(t)^2 \nonumber \\
\leq & \sum_{n=1}^{n_{0}}NRn + \sum_{n=1}^{T} 24Nm\exp\left(-\frac{1.8n f(n)^2}{64 W^2 m^2\sigma^2}\right) R \times Nn+ \sum_{t=1}^{T}Mf(t)^2. \label{chap5:equ:regret}
\end{align}

Thus the cumulative regret at time $T$ is bounded above by
\begin{align}
&\sum_{n=1}^{n_{0}}NRn + \sum_{n=1}^{T} 24Nm\exp\left(-\frac{1.8n f(n)^2}{64 W^2 m^2\sigma^2}\right)\times R \times Nn+ \sum_{t=1}^{T}Mf(t)^2 \nonumber \\
+ & 24N^2 m \frac{1}{e^{\frac{1.8\delta_{0}}{64W^2 m^2\sigma^2}}-1}R+N\left(\max\left\{n_{0},\frac{2}{p}\right\}\right)R. \nonumber
\end{align}

For a fixed $T$, we only need to minimize the following two terms since all others are constant:

\begin{align}
\sum_{n=1}^{T} 24Nm\exp\left(-\frac{1.8n f(n)^2}{64 W^2 m^2\sigma^2}\right)\times R \times Nn+ \sum_{t=1}^{T}Mf(t)^2. \label{equ:regret}
\end{align}

If we set $f^2(t)=\frac{2\log(T)\times 64W^2 m^2\sigma^2}{1.8t}$, then
\begin{align}
&\sum_{n=1}^{T} 24Nm\exp\left(-\frac{1.8n f(n)^2}{64 W^2 m^2\sigma^2}\right)\times R \times Nn+ \sum_{t=1}^{T}Mf(t)^2 \nonumber \\ 
\leq & \sum_{n=1}^{T} 24N^2 mnR \exp\left(-2\log(T)\right)  + \frac{128W^2 m^2\sigma^2 M\log(T)}{1.8}\sum_{t=1}^{T}\frac{1}{n} \nonumber \\
\leq &  24N^2 m R\frac{T(T-1)}{2T^2}  + 71.12 W^2 m^2\sigma^2 M\log(T)(\log(T)+1) \nonumber \\
\leq &  12 N^2 m R  + 71.12 W^2 m^2\sigma^2 M\log(T)(\log(T)+1). \nonumber
\end{align}

Thus, the cumulative expected regret is bounded by $O(N^2 m + M m^2(\log(T))^2)$.
\end{proof}

\begin{corollary}
If $\exists \delta>0$ such that $F_{i,j}(\delta)=0$ for all $i,j$, then the cumulative expected regret is bounded by $O(N^2)$.
\end{corollary}

\begin{proof}
The proof of this corollary is similar to the proof of Theorem~\ref{rst:regret}. If we set $f^2(t)=\frac{2\log(T)\times 64W^2 m^2\sigma^2}{1.8t}$, then there exists a $t_{0}$ such that for $t>t_{0}$, $S(f(t))=0$. Thus, similar to equation~\eqref{equ:regret}, we know the cumulative expected regret when we let the agents play myopically is bounded above by
\begin{align}
\sum_{n=1}^{n_{0}}NRn + \sum_{n=1}^{T} 24Nm\exp\left(-\frac{1.8n f(n)^2}{64 W^2 m^2\sigma^2}\right)\times R \times Nn+ \sum_{t=1}^{t_{0}}Mf(t)^2 \nonumber
\end{align}
Therefore, based on the same analysis of Theorem~\ref{rst:regret}, we know the cumulative regret is bounded by $O(N^2)$.
\end{proof}

\subsection{Practical Issues}
\label{sec:pi}
In Algorithm~\ref{Alg1}, we use ``pay whatever it takes'' strategy when we decide to incentivize the agent. However, ''pay whatever it takes'' only shows up in the proof of Lemma~\ref{round:length}. Without loss of generality, suppose we want to incentivize arm $i$ at time $t$ at the $n^{th}$ round. Based on the proof of Lemma~\ref{round:length}, as long as we offer a payment $c_{i,t}$ such that arm $i$ has at least $n^{-1}$ probability being pulled at time $t$, our results still hold true. We could compute this $c_{i,t}$ dynamically based on $F(\cdot)$ as well as our current estimate $u_{i,t}$. Here is the revised algorithm which would work well in practice:

\begin{algorithm}
\caption{Algorithm: Incentivizing Exploration}
\label{Alg2}
\begin{algorithmic}
\STATE Set n = 1 to denote the round number; Let $V=\emptyset$ be the set of arms that were pulled in the current round;
\FOR{ $t = 1, 2, 3, \cdots$}{
\STATE Let $S = \{ i : P( \theta \cdot u_{i,t} > \theta \cdot u_{j,t}\ \forall j\ne i | u_{j,t}\ \forall j) < n^{-1}\}$ be the set of arms with unincentivized probability of being pulled below $n^{-1}$.
\IF {$S\setminus V$ is non-empty}
\STATE{Choose an arm $i$ uniformly at random from $S\setminus V$}
\STATE{Offer payment $c_{i,t}=\inf\{c: P(\theta\sim F: \theta\cdot u_{i,t}+c>max_{j}\theta\cdot u_{j,t})>n^{-1}\}$}
\ELSE
\STATE {Let agents play myopically, i.e., offer payment $c_{j,t} = 0$ for all $j$}
\ENDIF
\STATE Denote $A_t$ as the pulled arm, update $V=V\cup\{A_t\}$, $u_{A_t,t}$ and $N(A_t,t)$
\IF {$n\neq \min_{i}N(i,t)$} 
\STATE $V=\emptyset$
\ENDIF
\STATE Update the round number, $n = \min_{i} N(i,t)$
}\ENDFOR

\end{algorithmic}
\end{algorithm}


The same proof would work and we can get the exact same results as Algorithm~\ref{Alg1}. 





\section{Lower Bound $\Omega(\log(T))$}
\label{sec:lb}

In this section, we assume $\theta$ follows a continuous distribution $F(\cdot)$. We provide an example to show the best possible lower bound is $\Omega(\log(T))$ regardless of the incentivizing strategy.

Suppose we have two arms. Arm $1$ has attribute vector $(0,0)$ and arm $2$ has attribute vector $(0,1)$. We assume the users' preference are uniformly distributed on the unit circle. If the user knows the exact attribute vectors for both arms, then the users with preference on the bottom half circle will choose arm $1$ and the users with preference on the top half circle will choose arm $2$.

Consider the following algorithm: at each step, let the agents play myopically; however, they are going to see the noisy rewards for both arms.

To lower bound the regret, we assume that the agents already know the true attribute vector for arm $1$. Without loss of generality, denote $u_{2,t} = (0,1)+(z_{t,1},z_{t,2}) = (0,1)+ (N(0,1/t),N(0,1/t))$ to be the estimate attribute vector for arm $2$ (Without loss of generality, we assume the variance for the noise is $1$). 

Since $(z_{t,1}, z_{t,2})$ is symmetric around $(0,0)$, we know 

\begin{align}
&E[r(t)] \nonumber \\
= &E[r(t) | z_{t,1}>0,z_{t,2}>0] \times P(z_{t,1}>0,z_{t,2}>0) + E[r(t) |z_{t,1}>0,z_{t,2}<0] \times P(z_{t,1}>0,z_{t,2}<0) \nonumber \\
&+E[r(t) | z_{t,1}<0,z_{t,2}>0] \times P(z_{t,1}<0,z_{n,2}>0) + E[r(t) |z_{t,1}<0,z_{t,2}<0] \times P(z_{t,1}<0,z_{t,2}<0) \nonumber \\
\geq & 0.25 \times E[r(t) | z_{t,1}>0, z_{t,2}>0]. \nonumber
\end{align}

Given $z_{t,1}>0$ and $z_{t,2}>0$, we know users whose preference vector between $(-1,0)$ and $\left(\frac{-1-z_{t,2}}{\sqrt{z_{t,1}^2+(1+z_{t,2})^2}}, \frac{z_{t,1}}{\sqrt{z_{t,1}^2+(1+z_{t,2})^2}}\right)$ as well as users whose preference vector between $(1,0)$ and $\left(\frac{1+z_{t,2}}{\sqrt{z_{t,1}^2+(1+z_{t,2})^2}}, \frac{-z_{t,1}}{\sqrt{z_{t,1}^2+(1+z_{t,2})^2}}\right)$ will make a mistake. The regret is the absolute value of the second coordinate of the user's preference vector. Thus, we know

\begin{align}
&E[r(t)| z_{t,1}>0, z_{t,2}>0] \nonumber \\
=& 4\times 2 \int_{0}^{\infty} \int_{0}^{\infty} \int_{0}^{\arctan\left(\frac{z_{t,1}}{1+z_{t,2}}\right)}\frac{\sin(\theta)}{2\pi}d(\theta)\frac{e^{-\frac{t \times z_{t,1}^2}{2}}\sqrt{t}}{\sqrt{2\pi}}d(z_{t,1})\frac{e^{-\frac{t \times z_{t,2}^2}{2}}\sqrt{t}}{\sqrt{2\pi}}d(z_{t,2}) \nonumber \\
=& \frac{2}{\pi^2}\int_{0}^{\infty} \int_{0}^{\infty}t\times \left[1-\frac{1+z_{t,2}}{\sqrt{z_{t,1}^2+(1+z_{t,2})^2}}\right]e^{-\frac{t \times z_{t,1}^2}{2}}e^{-\frac{t \times z_{t,2}^2}{2}}d(z_{t,1})d(z_{t,2}) \nonumber \\
=& \frac{2}{\pi^2}\int_{0}^{\infty} \int_{0}^{\infty} \left[1-\frac{\sqrt{t}+z_{t,2}}{\sqrt{z_{t,1}^2+(\sqrt{t}+z_{t,2})^2}}\right]e^{-\frac{z_{t,1}^2}{2}}e^{-\frac{z_{t,2}^2}{2}}d(z_{t,1})d(z_{t,2}) \nonumber 
\end{align}

Below, we want to show 
\begin{align}
\lim_{t\rightarrow\infty}\frac{E[r(t)| z_{t,1}>0, z_{t,2}>0]}{\frac{1}{t}} = O(1), \nonumber
\end{align}
and use the fact that $\sum_{n=1}^{T}\frac{1}{n}=O(\log(T))$ to show the regret is at least $\Omega(\log(T))$.

Denote $d(t)=t\left[1-\frac{\sqrt{t}+z_{t,2}}{\sqrt{z_{t,1}^2+(\sqrt{t}+z_{t,2})^2}}\right]$. Based on our calculation (see appendix for details), we know

\begin{align}
\lim_{t\rightarrow \infty} d(t)=\frac{z_{t,1}^2}{2}. \label{ex:limit}
\end{align}
Thus,
\begin{align}
&\lim_{t\rightarrow \infty}\frac{E[r(t)| z_{t,1}>0, z_{t,2}>0]}{\frac{1}{t}} \nonumber \\
=& \frac{2}{\pi^2}\int_{0}^{\infty} \int_{0}^{\infty}\lim_{t\rightarrow \infty}\left[ t\left[1-\frac{\sqrt{t}+z_{t,2}}{\sqrt{z_{t,1}^2+(\sqrt{t}+z_{t,2})^2}}\right]e^{-\frac{z_{t,1}^2}{2}}e^{-\frac{z_{t,2}^2}{2}}\right]d(z_{t,1})d(z_{t,2}) \nonumber  \\
=&\frac{2}{\pi^2}\int_{0}^{\infty} \int_{0}^{\infty}\left[ \frac{z_{t,1}^2}{2}e^{-\frac{z_{t,1}^2}{2}}e^{-\frac{z_{t,2}^2}{2}}\right]d(z_{t,1})d(z_{t,2}) = \frac{1}{2\pi}. \nonumber
\end{align}


Thus, the cumulative expected regret is at least $\Omega(\log(T))$.






\section{Conclusion}
In this paper, we study the incentivizing exploration problem with heterogeneous user preferences, which generalize the problem setting studied by \cite{frazier2014incentivizing} and \cite{han2015incentivizing}. We propose a simple policy that mostly exploits and occasionally incentivizing exploration, which can achieve $O(N^2+M(\log(T))^2)$ cumulative expected regret with $O(N^2)$ payment budget.




\newpage

\appendix

\section*{Proof of Lemma~\ref{lem:n0-inequality}}

\begin{proof}
First, we observe that
\begin{align}
&\frac{nx}{4\sigma}\geq \sqrt{0.6n\log(\log_{1.1}(n)+1)} \nonumber \\
\iff &\frac{n}{\log(\log_{1.1}(n)+1)}\geq \frac{9.6\sigma^2}{x^2}. \nonumber 
\end{align}
Since $\log(x)\leq x-1$ for $x>0$, we know 
\begin{align}
\log(\log_{1.1}(n)+1)=\log\left(\frac{\log(n)}{\log(1.1)}+1\right)\leq \log(11\log(n)+1)\leq \log(11n)\leq 3+\log(n). \nonumber
\end{align}

Thus, we know
\begin{align}
\frac{n}{\log(\log_{1.1}(n)+1)}\geq \frac{n}{3+\log(n)}. \nonumber 
\end{align}

To prove the lemma, we just need to show for $n\geq s_{0}$, we have
\begin{align}
\frac{n}{3+\log(n)}\geq \frac{9.6\sigma^2}{x^2}. \label{n0-equ}
\end{align}
Inequality~(\ref{n0-equ}) is true because of the following two observations:
\begin{itemize}
\item for $n\geq 50$, we have $\frac{n}{3+\log(n)}\geq n^{0.5}$;
\item for $n\geq \frac{92.16\sigma^4}{x^4}$, we have $n^{0.5}\geq {9.6\sigma^{2}}{x^{2}}$.
\end{itemize}

Thus, we know our lemma is true.

\end{proof}




\section*{Proof of Theorem~\ref{rst:budget}}

We need the following lemma in part of the proof of Theorem~\ref{rst:budget}.
\begin{lemma}
For all $n\geq 1$, we have
\begin{align}
5n^{3/5} \geq \sqrt{0.6n \log(\log_{1.1}(n)+1)}. \nonumber
\end{align}
\label{lemma:cal2}
\end{lemma}
\begin{proof}
\begin{align}
&5 n^{3/5} \geq \sqrt{0.6n \log(\log_{1.1}(n)+1)} \nonumber \\
\Longleftarrow &  25 n^{6/5} \geq n \log(\log_{1.1}(n)+1) \nonumber  \\
\Longleftarrow &  25 n^{1/5} \geq \log(\log_{1.1}(n)+1) \nonumber \\
\Longleftarrow &  25 n^{1/5} \geq \log_{1.1}(n)+1. \nonumber 
\end{align}

Denote $f(x) = 25x^{1/5} - \log_{1.1}(x)-1$. Set it's first derivate to $0$, we get its global mimumum at $x_0=\frac{161051}{3125}$ and $f(x_0)> 0$. Thus, our Lemma holds true.

\end{proof}


Now we are ready to prove our first main result, Theorem~\ref{rst:budget}.

\begin{proof}

\noindent\textbf{Step 1: Categorize measurement errors into different radius envelopes}


Denote $\epsilon_{i,t}=\ArmEV{t}{i}-\ArmV{i}$ to be the estimation error for the attribute vector $\ArmV{i}$ at time $t$. Denote $\epsilon_{i,t}^{j}$ to be the $j^{th}$ component of $\epsilon_{i,t}$. Denote $\omega$ to be a sample path and $s(t,\omega)$ to be the phase number for sample path $\omega$ at time $t$. For a fixed time $t$, define
\begin{align}
L^{'}[\ell](t) = \{\omega:|\epsilon_{i,t}^{j}(\omega)|\leq g(s(t,\omega),\ell), \forall i,j\}\nonumber
\end{align}
where $g(s,\ell)$ is a function which we will define later. Define $L[1](t) = L^{'}[1](t)$ and $L[i](t) = L^{'}[i](t)\setminus L^{'}[i-1](t)$ for $i\geq 2$. We call $L[\ell](t)$ the $\ell^{th}$ envelope at time $t$. We often simplify the notation and use $L[\ell]$ instead of $L[\ell](t)$ without confusion.

In the calculation below, we omit the dependency on $\omega$ when refering to variables $c(t)$, $\epsilon_{i,t}^{j}$ and $t_s$. Based on the definition of $L[\ell]$, we know if $\omega\in L[\ell]$, the maximum payment we need to offer at time $t$ is bounded above by 
\begin{align}
&\max_{i}\AgentV{t}\cdot \ArmEV{t}{i} - \min_{j}\AgentV{t}\cdot \ArmEV{t}{j} \nonumber \\
= &\max_{i}\AgentV{t}\cdot (\epsilon_{i,t}+\ArmV{i}) - \min_{j}\AgentV{t}\cdot (\epsilon_{j,t}+\ArmV{j}) \nonumber \\
\leq &\max_{i}\AgentV{t}\cdot \ArmV{i} - \min_{j}\AgentV{t}\cdot \ArmV{j} +\max_{i}\AgentV{t}\cdot \epsilon_{i,t} - \min_{j}\AgentV{t}\cdot \epsilon_{j,t}\nonumber \\
\leq & R + 2\Diam d\cdot g(s,\ell). \nonumber
\end{align}

We denote $\bar{c}(s,\ell)=R + 2\Diam d\cdot g(s,\ell)$ as the upper bound of payment for phase $s$ if our measurement error lie in the $\ell^{th}$ envelope.


\noindent\textbf{Step 2: Introduce a new stochastic process which bounds the total payment in a phase}

Denote $\mathcal{F}_t$ as the filtration up to time $t$. Denote $V_t$ as the set of arms that we need to pull, but haven't pulled yet in this phase. Based on our algorithm, we know

\[ \Prob{\text{play a new arm}|\mathcal{F}_t} =
\begin{cases}
1       & \quad \text{if we incentivize}\\
\sum_{i\in V} P(\AgentV{t} \cdot \ArmEV{t}{i}>\AgentV{t}\cdot \ArmEV{t}{j} \forall j\neq i)  & \quad \text{otherwise}  
\end{cases} \label{dom_stoc}
\] 
\hspace{1cm}$\geq (n-|V|)\times \frac{1}{n}=1-\frac{|V|}{n}$.
\bccomment{Wait for Peter to address this}

Let $(Z_{s,v,m}:m\geq 0)$ be a sequence of independent Bernoulli random variable with success probability $(1-\frac{v}{s})$. We will construct an alternative stochastic process for selecting which arm gets played that has the same distribution as the original process, but under which
\begin{align}
t_{s}-t_{s-1}\leq \bar{T}_{s}:=\sum_{v=0}^{\ARMNUM-1}\bar{T}_{s,v}, \nonumber 
\end{align}
where $\bar{T}_{s,v}:=\inf\{m\geq 0: Z_{s,v,m}=1\}+1$.

The new stochastic process will have the property that whenever $Z_{s,v(s,t),m(s,t)}=1$, we will play a new arm at time t for $t\in [t_{s-1}, t_{s}]$, where $v(s,t)$ is the number of unique arms played in phase $s$ strictly before time $t$, and $m(s,t)$ is the number of times we have pulled a previously pulled arm for the current value of $v(s,t)$. At time $t$, to determine what arm to pull, calculate \eqref{dom_stoc} and let $q_t$ be the probability computed. Note $q_t\geq 1-\frac{v(s,t)}{s}$.

If $Z_{s,v(s,t),m(s,t)}$ is 1, decide to play a new arm. Otherwise, draw a second Bernoulli random variable with probability $\frac{q_t-(1-\frac{v(s,t)}{s})}{1-(1-\frac{v(s,t)}{s})}$, and if it is 1, decide to play a new arm, and otherwise decide to play an old arm. Note that

\begin{align}
\Prob{\text{play a new arm}|\mathcal{F}_{t}}=\left[1-\frac{v(s,t)}{s}\right]+\left[1-(1-\frac{v(s,t)}{s})\right]\times \frac{q_t-(1-\frac{v(s,t)}{s})}{1-(1-\frac{v(s,t)}{s})}=q_t.
\end{align}

Since we pull a new arm only when $Z_{s,v(s,t),m(s,t)}=1$, the number of times we pull an arm when $v(s,t)=v$ is bounded above by $\bar{T}_{s,v}$. Thus,

\begin{align}
t_{s}-t_{s-1}\leq \sum_{v=0}^{\ARMNUM-1}\bar{T}_{s,v}=\bar{T}_{s}.
\end{align}
Finally, we want to bound $\sum_{t}E[c(t)]$. Let $C(s)$ be the cost incurred in phase $s$. Thus $\sum_{t}E[c(t)]=\sum_{s}E[C(s)]$. We know
\begin{align}
C(s)&=\sum_{t=t_{s-1}+1}^{t_{s}}c(t)=\sum_{t=t_{s-1}+1}^{t_{s}}\sum_{\ell}\mathbbm{1}\{\omega\in L(\ell)\}c(t) \nonumber \\
&\leq \sum_{t=t_{s-1}+1}^{t_{s}}\sum_{l}\mathbbm{1}\{\omega\in L(\ell)\}\bar{c}(s,\ell) \nonumber \\
&\leq (t_{s}-t_{s-1})\sum_{\ell} \mathbbm{1}\{\omega\in L(\ell)\}\bar{c}(s,\ell) \nonumber \\
&\leq \bar{T}_{s}\sum_{l}\mathbbm{1}\{\omega\in L(\ell)\}\bar{c}(s,\ell). \nonumber
\end{align}

Since $\bar{T}_{s}$ is independent of $\omega$, we know
\begin{align}
E[C(s)]\leq E[\bar{T}_{s}]\sum_{\ell}\Prob{\omega\in L(\ell)}\bar{c}(s,\omega). \nonumber
\end{align}
\noindent\textbf{Step 3: Rewrite the total payment expression}.

Based on the above notations, we can rewrite the cumulative payment as follows:
\begin{align}
\sum_{t=1}^{\infty}c(t) =\sum_{s=1}^{\infty}C(s)
\leq  \sum_{s=1}^{\infty}\sum_{\ell=1}^{\infty}\bar{T}_{s}\mathbbm{1}\{\omega\in L(\ell)\}\bar{c}(s,\ell). \nonumber
\end{align}

Set $g(s,\ell)$ to be $\frac{12\sigma \ell}{s^{2/5}}$. Since if $|\Arm{i}{j}-\ArmE{t}{i}{j}|\leq \lambda$ is true $\forall i$, $\forall j$, then we know for those $\AgV\in \{\AgV:\AgV\cdot \ArmV{\Best{\AgV}}-\AgV \cdot \ArmV{\Second{\AgV}}> 2\Diam d\lambda\}$, they will correctly identify their best arm. Thus, if $|\Arm{i}{j}-\ArmE{t}{i}{j}|\leq \frac{12\sigma l}{s^{2/5}} \leq \frac{p^{-1}(\frac{\MinProb}{2})}{2\Diam d}$ $\forall i$ and $\forall j$, then the probability that an unincentivized agent would pull arm $i$ is at least $\frac{p}{2}$. Further, if time $t$ is in a phase $s$ that satisfies $s^{-1}\leq \MinProb/2$, then our algorithm will not incentivize pulling any arms. Denote $a_0=\frac{24\Diam d\sigma}{p^{-1}(\frac{\MinProb}{2})}$. In order to have $\frac{12\sigma l}{s^{2/5}}\leq \frac{p^{-1}(\frac{\MinProb}{2})}{2\Diam d}$, it is sufficient to have $s\geq \lceil (a_{0} l)^\frac{5}{2} \rceil$. In order to have $s^{-1}\leq \frac{\MinProb}{2}$, we need $s\geq \frac{2}{\MinProb}$. Denote $s_2=\frac{2}{\MinProb}$. Thus, we know we can only incur regret for sample paths $\omega$ in the $\ell^{th}$ envelope in the first $\max\{s_2,\lceil (a_0 \ell)^\frac{5}{2} \rceil\}$ phases.

Thus,
\begin{align}
\sum_{t=1}^{\infty}c(t)\leq\sum_{\ell=1}^{\infty}\sum_{s=1}^{\max\{s_2,\lceil (a_0 \ell)^\frac{5}{2} \rceil\}}\bar{c}(s,\ell)\mathbbm{1}\{\omega\in L[\ell]\}\bar{T}_{s}. \nonumber
\end{align}

Therefore,

\begin{align}
&E\left[\sum_{t=1}^{\infty}c(t)\right] \nonumber\\
\leq &\sum_{\ell=1}^{\infty}\sum_{s=1}^{\max\{s_2,\lceil (a_0 \ell)^\frac{5}{2} \rceil\}}\bar{c}(s,\omega)\Prob{\omega \in L(\ell)}E[\bar{T}_{s}] \nonumber \\
=&\sum_{\ell=1}^{\infty}\sum_{s=1}^{\max\{s_2,\lceil (a_0 \ell)^\frac{5}{2} \rceil\}}\left[\MAXR+2\Diam d\frac{12\sigma \ell}{s^{2/5}}\right]\Prob(\omega\in L[\ell])E[\bar{T}_{s}] \nonumber \\
\leq &\sum_{\ell=1}^{\infty}\sum_{s=1}^{\max\{s_2,\lceil (a_0 \ell)^\frac{5}{2} \rceil\}}\left[\MAXR+24\Diam d\sigma l\right]\Prob(\omega\in L[\ell])E[\bar{T}_{s}]. \nonumber
\end{align}

\noindent\textbf{Step 4: Bound $\Prob{\omega\in L(\ell)}$}.

We now bound $\Prob{\omega\in L[\ell]}$ for $\ell\geq 2$. As a reminder, we omit the dependency between $\epsilon_{i,t}^{j}$, $s$ and $\omega$. We know 
\begin{align}
&\Prob{\omega\in L[\ell]} \nonumber \\
=&\Prob{\omega\in L^{'}[\ell]}- \Prob{\omega\in L^{'}[\ell-1]} \nonumber \\
\leq & 1-\Prob{|\epsilon_{i,t}^{j}|<\frac{12\sigma (\ell-1)}{s^{2/5}}, \forall i,j} \nonumber \\
=&\Prob{\exists i,j, s.t. |\epsilon_{i,t}^{j}|\geq \frac{12\sigma (\ell-1)}{s^{2/5}}} \nonumber  \\
\leq &\sum_{i,j}\Prob{|\epsilon_{i,t}^{j}|\geq \frac{12\sigma (\ell-1)}{s^{2/5}}} \nonumber
\end{align}

Define $S_{i,t}^{j}=\frac{\NumPull{t}{i}\epsilon_{i,t}^{j}}{2\sigma}$, then we know $S_{i,t}^{j}$ is a summation of $1/2$ gaussian random numbers. Therefore,
\begin{align}
&\sum_{i,j}\Prob{|\epsilon_{i,t}^{j}|\geq \frac{12\sigma(\ell-1)}{n^{2/5}}} \nonumber \\ 
=&\sum_{i,j}\Prob{|S_{i,t}^{j}|\geq \frac{6\NumPull{t}{i}(\ell-1)}{n^{2/5}}} \nonumber \\
\leq &\sum_{i,j}\Prob{|S_{i,t}^{j}|\geq 6\NumPull{t}{i}^{3/5}(\ell-1)}. \nonumber
\end{align}

Based on Lemma~\ref{lemma:cal2}, we know
\begin{align}
&6\NumPull{t}{i}^{3/5}(\ell-1) \nonumber \\
=& 5\NumPull{t}{i}^{3/5} + \NumPull{t}{i}^{3/5}(\ell-1) \nonumber \\
\geq & \sqrt{0.6\NumPull{t}{i}\log(\log_{1.1}(\NumPull{t}{i})+1)} + \sqrt{(\ell-1)^2 \NumPull{t}{i}} \nonumber \\
\geq & \sqrt{0.6\NumPull{t}{i}\log(\log_{1.1}(\NumPull{t}{i})+1)+(\ell-1)^2 \NumPull{t}{i}}. \nonumber
\end{align}

Based on Lemma~\ref{lem:ACI-inequality}, we know
\begin{align}
& \Prob{\omega\in L(\ell)} \nonumber \\
\leq &\sum_{i,j}\Prob{|S_{i,t}^{j}|\geq 6\NumPull{t}{i}^{3/5}(\ell-1)} \nonumber \\
\leq & \sum_{i,j}24e^{-1.8(\ell-1)^2} = 24\ARMNUM d\exp(-1.8(\ell-1)^2). \nonumber
\end{align}

\noindent\textbf{Step 5: Final Step}

Thus,
\begin{align}
&\sum_{t=1}^{\infty}c(t) \nonumber \\
\leq&\sum_{\ell=1}^{\infty}\sum_{s=1}^{\max\{s_2,\lceil (a_0 \ell)^{5/2} \rceil\}}\left[\MAXR+24\Diam d\sigma \ell\right]\Prob{\omega\in L[\ell])E[\bar{T}_{s}} \nonumber \\
\leq& \sum_{s=1}^{\max\{s_2,\lceil (a_0)^{5/2} \rceil\}}\left[\MAXR+24\Diam d\sigma \right]\Prob{\omega\in L[1]}E[\bar{T}_{s}] +\sum_{l=2}^{\infty}\sum_{s=1}^{\max\{s_2,\lceil (a_0 \ell)^{5/2} \rceil\}}\left[\MAXR+24\Diam d\sigma \ell\right]\Prob{\omega\in L[\ell]}E[\bar{T}_{s}] \nonumber \\
\leq& \sum_{s=1}^{\max\{s_2,\lceil (a_0)^{5/2} \rceil\}}\left[R+24\Diam d\sigma \right]E[\bar{T}_{s}] +\sum_{\ell=2}^{\infty}\sum_{s=1}^{\max\{s_2,\lceil (a_0 \ell)^{5/2} \rceil\}}\left[\MAXR+24\Diam d\sigma \ell\right]24\ARMNUM d e^{-1.8(\ell-1)^2}E[\bar{T}_{s}]\nonumber \\
\leq& \sum_{s=1}^{\max\{s_2,\lceil (a_0)^{5/2} \rceil\}}\left[\MAXR+24\Diam d\sigma \right]\ARMNUM s +\sum_{\ell=2}^{\infty}\sum_{s=1}^{\max\{s_2,\lceil (a_0 \ell)^{5/2} \rceil\}}\left[\MAXR+24\Diam d\sigma \ell\right]24 \ARMNUM d e^{-1.8(\ell-1)^2}\ARMNUM s \nonumber \\
\leq & \left[\MAXR+24\Diam d\sigma \right]\ARMNUM (\max\{s_2,\lceil (a_0)^{5/2} \rceil\})^2+\sum_{\ell=2}^{\infty}24\ARMNUM^2 d[\MAXR+24\Diam d\ell\sigma](\max\{s_2,\lceil (a_0 \ell)^{5/2} \rceil\})^2 e^{-1.8(\ell-1)^2} \nonumber \\
=& O(\frac{N}{\MinProb^5}+ \frac{\ARMNUM^2}{\MinProb^5}). \nonumber
\end{align}

\end{proof}

\section*{Calculation for Equation~\eqref{ex:limit}}

In this section, we prove $d(t)\leq \frac{z_{t,1}^{2}}{2}$ and $\lim_{t\rightarrow\infty}d(t)=\frac{z_{t,1}^2}{2}$. Denote $w(x)=\sqrt{x}$. Then we know $w(x+y)\leq \sqrt{x} + y\frac{1}{2\sqrt{x}}$ and $w(x+y)\geq \sqrt{x} + y\frac{1}{2\sqrt{x+y}}$ for $y\geq 0$. Thus, we know

\begin{align}
&\lim_{t\rightarrow\infty} t\left[1-\frac{\sqrt{t}+z_{t,2}}{\sqrt{z_{t,1}^2+(\sqrt{t}+z_{t,2})^2}}\right] \nonumber \\
=& \lim_{t\rightarrow\infty} t\left[\frac{\sqrt{z_{t,1}^2+(\sqrt{t}+z_{t,2})^2}-\sqrt{t}-z_{t,2}}{\sqrt{z_{t,1}^2+(\sqrt{t}+z_{t,2})^2}}\right]  \nonumber \\
\geq & \lim_{t\rightarrow\infty} t\left[\frac{\sqrt{t}+z_{t,2}+z_{t,1}^{2}\frac{1}{2\sqrt{z_{t,1}^2+(\sqrt{t}+z_{t,2})^2}}-\sqrt{t}-z_{t,2}}{\sqrt{z_{t,1}^2+(\sqrt{t}+z_{t,2})^2}}\right]  \nonumber \\
= & \lim_{t\rightarrow\infty} t \times \frac{z_{t,1}^2}{2(z_{t,1}^2+(\sqrt{t}+z_{t,2})^2)} \rightarrow \frac{z_{t,1}^2}{2}. \nonumber 
\end{align}
and
\begin{align}
&\lim_{t\rightarrow\infty} t\left[1-\frac{\sqrt{t}+z_{t,2}}{\sqrt{z_{t,1}^2+(\sqrt{t}+z_{t,2})^2}}\right] \nonumber \\
=& \lim_{t\rightarrow\infty} t\left[\frac{\sqrt{z_{t,1}^2+(\sqrt{t}+z_{t,2})^2}-\sqrt{t}-z_{t,2}}{\sqrt{z_{t,1}^2+(\sqrt{t}+z_{t,2})^2}}\right]  \nonumber \\
\leq & \lim_{t\rightarrow\infty} t\left[\frac{\sqrt{t}+z_{t,2}+z_{t,1}^{2}\frac{1}{2\sqrt{(\sqrt{t}+z_{t,2})^2}}-\sqrt{t}-z_{t,2}}{\sqrt{z_{t,1}^2+(\sqrt{t}+z_{t,2})^2}}\right]  \nonumber \\
= & \lim_{t\rightarrow\infty} t \times \frac{z_{t,1}^2}{2\sqrt{z_{t,1}^2+(\sqrt{t}+z_{t,2})^2}(\sqrt{t}+z_{t,2})} \rightarrow \frac{z_{t,1}^2}{2}. \label{inequ:dct}
\end{align}   
Based on inequality~\eqref{inequ:dct}, we know $d(t)\leq \frac{z_{t,1}^{2}}{2}$ and $\lim_{t\rightarrow\infty}d(t)=\frac{z_{t,1}^2}{2}$.

\section*{Proof of Lemma~\ref{lem:n1-inequality}}

\begin{proof}
We first prove $2\sqrt{\frac{2}{1.8}s\log(T)}\geq \sqrt{0.6s\log(\log_{1.1}(s)+1)}$ for $s\geq 2$. This is true because
\begin{align}
&2\sqrt{\frac{2}{1.8}s\log(T)}\geq \sqrt{0.6s\log(\log_{1.1}(s)+1)} \nonumber \\
\Longleftarrow & 2\sqrt{s\log(T)}\geq \sqrt{s\log(\log_{1.1}(s)+1)} \nonumber \\
\Longleftarrow & 4\log(T)\geq \log(\log_{1.1}(s)+1) \nonumber \\
\Longleftarrow & T^4 \geq \log_{1.1}(s)+1 \nonumber \\
\Longleftarrow & s^4 \geq \log_{1.1}(s)+1, \nonumber
\end{align}
the derivative of $s^4-\log_{1.1}(s)-1$ is positive for $s\geq 2$ and $2^4-\log_{1.1}(s)-1>0$. 

Therefore, we know
\begin{align}
&3\sqrt{\frac{2}{1.8}s\log(T)} \nonumber \\
= & 2\sqrt{\frac{2}{1.8}s\log(T)} + \sqrt{\frac{2}{1.8}s\log(T)} \nonumber \\
\geq & \sqrt{0.6s\log(\log_{1.1}(s)+1)}+\sqrt{\frac{2}{1.8}s\log(T)} \nonumber \\
\geq & \sqrt{0.6s\log(\log_{1.1}(s)+1)+\frac{2}{1.8}\log(T)s}. \nonumber 
\end{align}

\end{proof}


\bibliography{reference}
\end{document}
