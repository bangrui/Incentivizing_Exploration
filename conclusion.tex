\section{Conclusion}
We study the problem of incentivizing exploration with heterogeneous
user preferences.
We propose an algorithm that achieves expected cumulative regret of
$O(\ARMNUM \e^{2/\MinProb} + \ARMNUM \log^3(T))$,
using expected cumulative payments of $O(\ARMNUM^2 \e^{2/\MinProb})$.
It is possible to improve these bounds to polynomial (in \ARMNUM and
$1/\MinProb$) when \MinProb is known or the preference distribution is
discrete.
In fact, we conjecture that this should be possible even in the full
generality of our model.
As a first step towards such a polynomial bound, we believe that it
should be possible to obtain an exponential dependence on
$1/(\MinProb \ARMNUM)$, which gives polynomial dependence unless some
arm has a much smaller fraction of the population preferring.

Taking this goal one step further, we would like to 
develop algorithms that do not require all arms to be preferred by a
strictly positive fraction of agents.
An alternate algorithm might only incentivize an arm if its estimated
attribute vector is close enough to a Pareto frontier.
The regret will then be $\Omega(\log(T)$ when at least one arm falls
below the Pareto frontier, as we no longer have free exploration of
all arms. 
It is likely that a bound will deteriorate as the number of such
unpreferred arms increases.

Finally, it would be desirable to generalize the class of utility
functions that can be handled beyond inner products.
We believe that similar results hold for arbitrary
Lipschitz-continuous utility functions of the arm's attribute vector,
and that only minor modifications are necessary to the algorithm and
proofs.
