\documentclass{article}
\usepackage[utf8]{inputenc}

\usepackage[english]{babel}

\usepackage{blindtext}
\usepackage{amssymb}

\usepackage[]{algorithm}
\usepackage[]{algorithmic}
\usepackage[T1]{fontenc}    % use 8-bit T1 fonts
\usepackage{xr} 
\usepackage{amsfonts}       % blackboard math symbols
\usepackage{nicefrac}       % compact symbols for 1/2, etc.
\usepackage{microtype}      % microtypography
\usepackage{amsthm}
\usepackage{comment}
\usepackage{graphicx}
\usepackage{subcaption}
\usepackage{xcolor}
\usepackage{amsmath}
\usepackage{dblfloatfix}
\usepackage{bbm}

\newcommand{\IE}{IE} 
\newcommand{\argmax}{\mathop{\mathrm{argmax}}}



\newcommand\floor[1]{\lfloor#1\rfloor}
\newcommand\ceil[1]{\lceil#1\rceil}

\newcommand{\bccomment}[1]{{\color{blue}BC: #1}}
\newcommand{\pfcomment}[1]{{\color{blue}PF: #1}}


\newtheorem{definition}{Definition}[section]
\newtheorem{theorem}{Theorem}
\newtheorem{lemma}{Lemma}
\newtheorem{corollary}{Corollary}
\newtheorem{assumption}{Assumption}



\title{Incentivizing Exploration with Heterogeneous Utilities}

\begin{document}

\maketitle


\section{Introduction}
In this paper, we study a variant of incentivizing exploration with heterogeneous agents' preferences, that generalizes \cite{frazier2014incentivizing} and \cite{han2015incentivizing}. 

In our setting we have arms that have unknown multivariate attributes, and agents have heterogeneous utility functions that map these attribute vectors onto utilities. All agents see the attribute outcomes from previous agents. We study a simple policy that mostly exploits, and occasionally incentivizes exploration when an arm would be unpulled by all agent types given the current posterior.  

The intuition behind is that heterogeneity of a certain kind reduces but doesn't eliminate the need for incentivizing exploration --- basically that you get a certain amount of exploration for free via heterogeneity.

We structure our paper as follows: In Section~\ref{sec:prob}, we state our problem formulation; In Section~\ref{sec:ub}, we state our algorithm and prove a upper bound for both the payment budget as well as the regret; In Section~\ref{sec:lb}, we construct an example to show that the regret bound is at least $O(\log(T))$ in the worst case example, regardless the payment budget.



\subsection{Related Work}

With money transfer: \cite{frazier2014incentivizing} considers a problem setting where the principal pays the agents extra money in order to incentivize them to explore. In their paper, they assume all agents have equal value of money and provide a complete characterization of achievable reward with a fixed budget. \cite{han2015incentivizing} generalizes this framework to allow agents have heterogeneous value of money. Further, it assumes there is a external signal scheme which provides partial information about each agent. Under this setting, they proved a similar bound as in \cite{frazier2014incentivizing} which depends on the signal scheme.

Without money transfer but using information asymmetry:
\cite{kremer2014implementing} considers a simple model where agents arrive to the principal one by one and there are only two actions at each time. \cite{mansour2015bayesian} generalizes \cite{kremer2014implementing} by allowing finite number of actions at each time. \cite{mansour2016bayesian} considers a problem setting where there are multiple agents at each time and allow agents to interact with each other. At each time, the principal provides each agent an recommendation, which is Bayesian incentive-compatible. They prove principal can achieve constant regret when utilities are deterministic and logarithmic regret when utilities are stochastic.

\subsection{Applications}

The motivation of this project is to build a recommender system. Agent arrives to the system one at a time. Upon arrival, he/she receives a recommendation from the system and chooses a product based on the recommendation. At the end, the agent provides some feedback about the product when he/she leaves the system.
Here are a few applications where we can apply our framework:

\begin{itemize}
\item \textit{Recommender systems/Online Retailers}: For recommender systems such as Yelp, they want to find the best restaurants based on the user's review. As the system, it wants all restaurants to have as many reviews as possible so that it knows which restaurants are good. However, as a user, he/she may only want to try the restaurants with a large number of positive reviews. In this case, the system may incentivize the user to review the new restaurants by offering them extra money as a reward. As for users, they may have different utilities vectors about the restaurant's location, environment, service, queueing time etc, which could be the restaurant's attribute vector. This is also the case for online retailers such as Amazon. In this case, the attribute vector could be each item's quality, design and price etc.
\item \textit{Citizen Science}:  \cite{frazier2014incentivizing} mentioned for scientific organizations like Galaxy Zoo, they will guide the enthusiasts to explore the less-explored parts of domain. For each enthusiast, he/she may have different preference towards the location's sight-seeing, weather, distance and safety etc. Thus, as the organization, it needs to incentivize each person differently. 
\end{itemize}




\section{Problem Setting}
\label{sec:prob}

We have $N$ arms. Arm $i$ has a fixed but unknown attribute vector $u_i\in \mathbb{R}^{m}$. 

A stream of myopic selfish agents come to our system.  Agent $t$ has linear preferences over attributes described by a vector $\theta_t \in \mathbb{R}^m$ that is unknown to the principal and drawn iid from a known distribution $F$.  We refer synonomously to an agent and that agent's preference vector: when we say ``an agent $\theta$'', we mean ``an agent with preference vector $\theta$.''

Each agent $t$ chooses an arm to pull $A_t$, according to a process described below, and obtains utility $\theta_t \cdot u_{A_{t}}$.  The principal and all agents then see a noisy observation of the attribute vector of the pulled arm of the form $O_t=u_{A_{t}}+\epsilon_{t}$, where $\epsilon_t\sim N(0, \sigma^2 I_{m})$ is independent normally distributed noise, and $I_m$ denotes an $m$-dimensional identity matrix.  Although we assume a common variance across attributes for simplicity of presentation, our theoretical results hold if the variance differs.

At each time $t$, for each arm $i$, we (the principal) offer a non-negative payment $c_{i,t}\geq 0$ based on previous observations.
We assume that agent $t$ chooses to pull the arm that myopically maximizes the sum of this payment and an estimate of the utility obtained $\theta_t \cdot u_{i,t}$ where $u_{i,t}$ denotes the simple average of $O_s$ over all previous pulls of arm $i$ if it has been pulled at least once, or a default value of $0$ if not.  That is, $u_{i,t} = \sum_{s<t} O_s 1\{A_s = i\} / \sum_{s<t} 1\{A_s = i\}$ if the denominator is strictly positive and $0$ otherwise; and $A_t=\argmax_{i}\{\theta_t\cdot u_{i,t}\}$, breaking ties in favor of the arm with the highest incentive.  We use $c_t=c_{A_{t},t}$ to denote the actual incentive payment at time $t$.  \pfcomment{Bangrui, I assumed that $u(i,t)$ starts from $0$.  Do you want to do this, or do you want to assume that every arm is pulled at least once?  Or do you want to assume that they start from some general starting value?  We could perhaps do the analysis assuming they start from $0$, but then comment that the proofs all carry through if they start from some other value.}

This behavior may be recovered if agents are Bayesian and share a common non-informative prior distribution that is constant over $\mathbb{R}^m$ and know $\sigma^2$.  In this case, the posterior distribution on $u_{i}$ at time $t$ is multivariate normal with mean $u_{i,t}$, and the expected value of $\theta_t \cdot u_i$ under this posterior conditioned on $\theta_t$ is $\theta_t \cdot u_{i,t}$ \cite[equation 2.13 in section 2.5]{Ge04}.  Alternatively, one may simply take our assumption that agents use the average as their estimate of an attribute value directly without such a Bayesian justification.

We define the regret at time $t$ as $r(t)=\max_{i}\{\theta_{t}\cdot u_{i}\}-\theta_t\cdot u_{A_t}$, and the cumulative regret up to time T as $R(T)=\sum_{t=1}^{T}r(t)$. Define the cumulative payment up to time T similarly as $C(t)=\sum_{t=1}^{T}c(t)$. 
As the principal, we want to find a strategy $\mathcal{A}$ under which both the cumulative expected regret $\mathbb{E}_{\mathcal{A}}[R(T)]$ and the cumulative expected payment $\mathbb{E}_{\mathcal{A}}[C(T)]$ are small.

To support later development, we define three additional pieces of notation.
We let $B(\theta)$ refer to the index of the arm that is best for an agent with preference vector $\theta$, $B(\theta) \in \argmax_i \theta \cdot u_i$, breaking ties uniformly at random. 
We use $\tau(i,t)$ to denote the set of times at which arm i was pulled up to time $t$,
$\tau(i,t) = \{ s : 1 \le s \le t, A_s = i \}$.
We let $N(i,t)$ denote the cardinality of $A_t$, i.e., the number of pulls of arm $i$ at times up to and including $t$.
\pfcomment{Bangrui, if we don't use $\tau(i,t)$ or $N(i,t)$ later, we can drop them here.}

\section{Upper Bound}
\label{sec:ub}

In this section, we propose a simple policy that mostly exploits, and occasionally incentivizes exploration when an arm would be unpulled by all agent types given the current posterior. We prove that with the help of heterogeneous preferences, we can get a certain amount of exploration for free via heterogeneity. 



\subsection{Assumptions}
Throughout this section, we use $F(\cdot)$ to denote the agent's preference distribution, which could either be a continuous distribution or a discrete distribution. Denote

\begin{align}
\Omega_{i,j}=\{\theta:B(\theta)=i, \hat{B}(\theta)=j\}, \nonumber 
\end{align}
which is the users whose best arm is arm $i$ and the second best arm is arm $j$. We need the following assumptions in our analysis:

\begin{assumption} Let $f_{i,j}(y)$ be the marginal probability density function (or marginal probability mass function if $F(\cdot)$ is a discrete distribution) of $(u_i-u_j)\cdot\theta$ conditioned on $\theta \in \Omega_{i,j}$. We assume $M^{'}=\max_{i,j}\limsup_{y\rightarrow 0^{+}}f_{i,j}(y) <\infty$.
\label{A1}
\end{assumption}

Intuitively, assumption~\ref{A1} states that there are not many agents who are indifferent between their best arm and the second best arm. For simplicity in our analysis, we use a slightly stronger condition below: $M=\max_{i,j}sup_{y\in R^{+}}f_{i,j}(y) <\infty$.


\begin{assumption} We assume $F$ has a compact support set. Without loss of generality, we assume $\theta\in [0,W]^m$.
\label{A2}
\end{assumption}

Throughout this write up, we use $R$ to denote the maximum regret it can occur at each time. We know $R<\infty$ due to Assumption~\ref{A2}. 

\begin{assumption}
Denote $p=\min_{i}P(\{\theta: B(\theta)=i\})$. Throughout this paper, we assume $p>0$.
\label{A3}
\end{assumption}

Assumption~\ref{A3} means for each arm $i$, there exist a positive proportion of users whose best arm is $i$. 

\subsection{Our Algorithm}
Before we introduce our algorithm, we need to use the following definition:

\begin{definition}
We call time $t_{n}$ as the starting point of the $n_{th}$ round if $t_{n}=\min_{t}\{\forall i, N(i,t)\geq n\}$. We call the time period from $t_{n}$ to $t_{n+1}$ as the $n_{th}$ round.
\end{definition}

In this setting, we consider an algorithm that offers incentive to an arm if and only if 
\begin{itemize}
\item there is no agents will play it anymore;
\item this arm has not been played or incentivized before in this round.
\end{itemize}
We break ties randomly. If we decides to incentivize an arm, we use ``pay what it takes strategy''. This algorithm does not need to know $T$ in advance. The detailed algorithm is as follows:

\begin{algorithm}
\caption{Algorithm: Upper Bound}
\label{Alg1}
\begin{algorithmic}
\STATE Set n = 1 to denote the round number; g(n) to be a increasing function which we choose later;
\FOR{ $t = 1, 2, 3, \cdots$}{
	\IF{the myopic probability of pulling an arm is less than $g(n)^{-1}$ and we haven’t pulled that arm in this round}
	\STATE{ we pay whatever it takes to incentivize that arm}\ELSE 
		\STATE {let the agents play myopically}\ENDIF
		\STATE $n = min_{i} N(i,t)$
}\ENDFOR

\end{algorithmic}
\end{algorithm}


\subsection{General Results}

Now we analyze the regret occurred by Algorithm~\ref{Alg1} and its payment budget. Before we start our analysis, we need the following notations. Denote $\hat{B}(\theta)=\arg\max_{i\neq B(\theta)}\{\theta\cdot u_{i}\}$ to the $\theta$'s second best arm. We use $S(\delta)$ to denote the proportion of user whose utility difference between their best arm and the second best arm is less than $\delta$. Formally, $S(\delta)=P(\theta: \theta \cdot u_{B(\theta)}-\theta\cdot u_{\hat{B}(\theta)}\leq \delta)$. Denote $\lambda(\delta)=\frac{\delta}{2m}$.  Denote $p(\delta)=\min_{i}P(\{\theta:\theta\cdot u_{B(\theta)}-\theta\cdot u_{\hat{B}(\theta)}>\delta\})$ and we know $p(0)=p$. 

With the help of our assumptions and the above notations, we can prove the following main results:

\begin{theorem}
The payment budget for Algorithm~\ref{Alg1} is bounded above by $O(N^2)$.
\label{rst:budget}
\end{theorem}


\begin{theorem}
The cumulative regret for Algorithm~\ref{Alg1} is bounded above by $O(N^2) + O(N \sum_{n=1}^{T}n^{\alpha})$ for $-1<\alpha<0$.
\label{rst:regret}
\end{theorem}

Before we prove our main results, we need to prove several lemmas. First, based on Assumption~\ref{A1}, we can get the following bound for $S(\delta)$.

\begin{lemma}
$S(\delta)\leq M\delta$.
\label{lemma:sdelta}
\end{lemma}

\begin{proof}
\begin{align}
S(\delta)&=\sum_{B(\theta)=i}\sum_{\hat{B}(\theta)=j}P(\{\theta(u_{i}-u_{j})<\delta\}) \nonumber \\
&=\sum_{B(\theta)=i}\sum_{\hat{B}(\theta)=j}P(\{\theta(u_{i}-u_{j})<\delta|\theta\in \Omega_{i,j}\})P(\theta\in \Omega_{i,j}) \nonumber \\
&\leq \sum_{B(\theta)=i}\sum_{\hat{B}(\theta)=j}M\delta \times P(\theta\in \Omega_{i,j}) \nonumber \\
&=M\delta \nonumber 
\end{align}
\end{proof}

The following lemma bounds the probability of making a mistake if we let the agents play myopically in the $n_{th}$ round, given that the utility difference between his/her best and second best arm is bounded below by a constant.

\begin{lemma}
For those user whose utility difference between their best arm and the second best arm is at least $2m\lambda$, the probability that myopic action can incur regret at any time $t\in [t_{n-1},t_{n}-1]$ in the $n_{th}$ round is bounded above by $24Nm\exp\left(-\frac{1.8n\lambda^2}{8\sigma^2}\right)$, for $n\geq n_{0}=\max\{50, \frac{23.04\sigma^4}{\lambda^4}\}$.
\label{round:prob}
\end{lemma}


We need the following lemma in order to prove Lemma~\ref{round:prob}.

\begin{lemma}
For $n\geq n_{0}=\max\{50, \frac{23.04\sigma^4}{\lambda^4}\}$, we have
\begin{align}
\frac{n\lambda}{2\sqrt{2}\sigma}\geq \sqrt{0.6n\log(\log_{1.1}(n)+1)}. \nonumber
\end{align}
\label{n0-inequality}
\end{lemma}

\begin{proof}
First, we observe that
\begin{align}
&\frac{n\lambda}{2\sqrt{2}\sigma}\geq \sqrt{0.6n\log(\log_{1.1}(n)+1)} \nonumber \\
\iff &\frac{n}{\log(\log_{1.1}(n)+1)}\geq \frac{4.8\sigma^2}{\lambda^2}. \nonumber 
\end{align}
Since $\log(x)\leq x-1$ for $x>0$, we know 
\begin{align}
\log(\log_{1.1}(n)+1)=\log\left(\frac{\log(n)}{\log(1.1)}+1\right)\leq \log(11\log(n)+1)\leq \log(11n)\leq 3+\log(n). \nonumber
\end{align}

Thus, we know
\begin{align}
\frac{n}{\log(\log_{1.1}(n)+1)}\geq \frac{n}{3+\log(n)}. \nonumber 
\end{align}

In order to prove the original lemma, we just need to show for $n\geq n_{0}$, we have
\begin{align}
\frac{n}{3+\log(n)}\geq \frac{4.8\sigma^2}{\lambda^2}. \label{n0-equ}
\end{align}
Inequality~(\ref{n0-equ}) is true because the following two observations:
\begin{itemize}
\item for $n\geq 50$, we have $\frac{n}{3+\log(n)}\geq n^{0.5}$;
\item $n\geq \frac{23.04\sigma^4}{\lambda^4}$ we have $n^{0.5}\geq {4.8\sigma^{2}}{\lambda^{2}}$.
\end{itemize}

Thus, we know our lemma is true.

\end{proof}

We also need to use an adaptive concentration inequality, which is introduce in \cite{zhao2016adaptive}. For reference, we state it as a Lemma.

\begin{lemma}[Corollary 1 in \cite{zhao2016adaptive}]
Let $X_{i}$ be zero mean $1/2$-subgaussian random variables. $\{S_{n}=\sum_{i=1}^{n}X_{i},n\geq 1\}$ be a random walk. Let $J$ be any random variable taking value in $\mathbb{N}$. If
\begin{align}
f(n)=\sqrt{0.6n\log(\log_{1.1}(n)+1)+bn}, \nonumber
\end{align}
then
\begin{align}
Pr[S_{J}\geq f(J)]\leq 12e^{-1.8b}. \nonumber
\end{align}
\label{ACI-inequality}
\end{lemma}


Now we can get back to prove Lemma~\ref{round:prob}.

\begin{proof}
In the $n_{th}$ round, we know all arms have been pulled at least $n$ times. Thus, the probability that myopic action can occur regret is bounded by
\begin{align}
&P(\text{myopic action incurs regret})\nonumber \\
\leq &P(\exists i, \exists j, |u_{i,t}^{j}-u_{i}^{j}|>\lambda) \nonumber \\
\leq & \sum_{i}\sum_{j} P(|u_{i,t}^{j}-u_{i}^{j}|>\lambda) \nonumber \\
= &  \sum_{i}\sum_{j} P\left(\bigg|\sum_{k\in \tau_{i,t}}\epsilon_{k}\bigg|>N(i,t)\lambda\right). \label{ACI}
\end{align}

To bound equation~(\ref{ACI}), we need to use Lemma~\ref{ACI-inequality}. Similar to Lemma~\ref{ACI-inequality}, we define
\begin{align}
S_{N(i,t)}=\sum_{k\in\tau_{i,t}}\frac{\epsilon_{k}}{\sqrt{2}\sigma}=\frac{1}{\sqrt{2}\sigma}\sum_{k\in\tau_{i,t}}\epsilon_{k}. \nonumber
\end{align}

Based on Lemma~\ref{n0-inequality}, we know there exists a $n_{0}=\max\{50, \frac{23.04\sigma^2}{\lambda^2}\}$ such that for $n\geq n_{0}$, we have
\begin{align}
\frac{n\lambda}{2\sqrt{2}\sigma}\geq \sqrt{0.6n\log(\log_{1.1}(n)+1)}. \nonumber
\end{align}

Thus, if we set $b=\frac{n\lambda^2}{8\sigma^2}$ in Lemma~\ref{ACI-inequality}, for any $N(i,t)\geq n\geq n_{0}$, we have
\begin{align}
\frac{N(i,t)\lambda}{\sqrt{2}\sigma}\geq & \sqrt{0.6N(i,t)\log(\log_{1.1}(N(i,t))+1)}+\frac{\lambda}{2\sqrt{2}\sigma}\sqrt{n N(i,t)} \nonumber \\
\geq & \sqrt{0.6N(i,t)\log(\log_{1.1}(N(i,t))+1)+bN(i,t)}, \nonumber 
\end{align}
where the last inequality is because $\sqrt{x}+\sqrt{y}\geq \sqrt{x+y}$. Thus, we have
\begin{align}
&P\left(\bigg|\sum_{k\in \tau_{i,t}}\epsilon_{k}\bigg|>N(i,t)\lambda\right) \nonumber \\
=&2P\left(S_{N(i,t)}>\frac{N(i,t)\lambda}{\sqrt{2}\sigma}\right) \nonumber \\
\leq & 2P\left(S_{N(i,t)}> \sqrt{0.6 N_{i,t}\log(\log_{1.1}(N(i,t))+1)+b N(i,t)}\right) \nonumber \\
\leq & 24\exp( -1.8b) \nonumber \\
= & 24\exp\left(\frac{-1.8 n\lambda^2}{8\sigma^2}\right). \nonumber
\end{align}


Thus, we know
\begin{align}
&\sum_{i}\sum_{j} P\left(\bigg|\sum_{k\in \tau_{i,t}}\epsilon_{k}\bigg|>N(i,t)\lambda\right)  \nonumber \\
=& 24Nm \exp\left(\frac{-1.8 n\lambda^2}{8\sigma^2}\right). \nonumber
\end{align}

\end{proof}





Before we start analyzing the cumulative regret, we first prove the following lemma which bounds the expected length of each round.

\begin{lemma}
Using our algorithm, we have $\mathbb{E}[t_{n}-t_{n-1}]\leq Ng(n)$, $\forall n\geq 1$.
\label{round:length}
\end{lemma}


\begin{proof}
In each round, each arm needs to be pulled at least once. Denote $X_{i}$ to be the number of agents come to the system between $i-1$ different arms were pulled and the $i_{th}$ different arm was pulled. Then we know 
\begin{align}
\mathbb{E}[t_{n}-t_{n-1}]=\sum_{i=1}^{N}E[X_{i}]. \nonumber
\end{align}


Denote $t^{0}=t_{n-1}+\sum_{j=1}^{i-1}X_{j}$, then $X_{i}$ follows a ``geometric distribution'', which counts the total number of trials before the $i_{th}$ different arm were pulled at time $t=t^{0},t^{0}+1,\cdots$. At each time $t=t^{0}, t^{0}+1,\cdots$, there are two possibilities for a success trial:
\begin{itemize}
\item If there exists an arm has not been pulled and the proportion of agents who would pull this arm is less than $\frac{1}{g(n)}$, then based on our algorithm, the principal will offer an payment for that arm and it will be pulled. In this case, the probability that the $i_{th}$ different arm will be pulled is 1.  
\item There exists at least $\frac{1}{g(n)}$ proportion of agents who will choose an un-pulled arm as his/her myopic action. In this case, we know each type of agent shows up with probability at least $p$. Thus, the probability that the $i_{th}$ different arm will be pulled is at least $\frac{1}{g(n)}$.
\end{itemize}

Since $X_{i}$ follows a ``geometric distribution'' with success probability at least $\frac{1}{g(n)}$ at each time, we know 
\begin{align}
E[t_{n}-t_{n-1}]\leq Ng(n). \nonumber
\end{align}
\end{proof}

\begin{lemma}
The expected incentive payment at round $n$ is bounded above by \bccomment{How to bound this conditional expectation?}
\end{lemma}

\begin{proof}
Denote $\Delta=\sup_{\theta}\max_{i}\{\theta\cdot u_{B(\theta)}-\theta\cdot u_{i}\}$ to be the theoretical maximum payment with no noisy. We first consider the expected payment for arm $1$ in the $n_{th}$ round. Denote $\epsilon_{i,t}=u_{i,t}-u_{i}$.

\begin{align}
&E[\max_{i}\theta_t\cdot u_{i,t}-\theta_t\cdot u_{1,t}|\text{agents pull arm $1$ with prob less than $g(n)^{-1}$}] \nonumber \\
\leq & \Delta + E[\max_{i}\theta_t\cdot \epsilon_{i,t}-\theta_t\cdot \epsilon_{B(\theta_t),t}|\text{agents pull arm $1$ with prob less than $g(n)^{-1}$}] \nonumber 
\end{align}

Denote $p_1=P_{\theta\sim F}(B(\theta)=1)$. 


If in the $n_{th}$ round we have $g(n)^{-1}>p_1$:


If in the $n_{th}$ round we have $g(n)^{-1}<\frac{p}{2}$:
\begin{align}
E[\max_{i}\theta_t\cdot \epsilon_{i,t}-\theta_t\cdot \epsilon_{B(\theta_t),t}|\text{agents pull arm $1$ with prob less than $g(n)^{-1}$}]  \nonumber 
\end{align}

\end{proof}



Now we are ready to prove our first main result, Theorem~\ref{rst:budget}.

\begin{proof}
If $|u_{i}^{j}-u_{i,t}^{j}|\leq \lambda$ is true $\forall i$, $\forall j$, then we know for those $\theta\in \{\theta:\theta\cdot u_{B(\theta)}-\max_{j\neq B(\theta)}\{\theta \cdot u_{j}\}\geq 2m\lambda\}$, they will correctly identify their best arm. Thus we know, in the $n^{th}$ round, if $|u_{i}^{j}-u_{i,t}^{j}|\leq \frac{p^{-1}(\frac{p}{2})}{2m}$ $\forall i$ and $\forall j$, and $1/g(n)<p/2$, we do not need to incentivize any arms.

Denote 
\begin{align}
n_{1}=\max\bigg\{50, \frac{23.04\sigma^4}{\lambda\left(p^{-1}(\frac{p}{2})\right)^4},\min_{n}\{n:1/g(n)<\frac{p}{2}\}\bigg\}. \nonumber
\end{align}

The cumulative payment is bounded above by:

\begin{align}
&E[C(T)] \nonumber \\
= & \sum_{t=1}^{\infty}E[\mathbbm{1}\{\text{need to offer payment to an arm}\}] \nonumber \\
\leq &\sum_{n=1}^{\infty}\sum_{t=1}^{\infty}E[\mathbbm{1}\{t_{n-1}\leq t\leq t_{n}: \text{myopic action occurs regret for agents with } \nonumber \\
&\theta\in \{\theta:\theta\cdot u_{B(\theta)}-\max_{j\neq B(\theta)}\{\theta \cdot u_{j}\}\geq p^{-1}(1/g(n))\}\}] \nonumber \\
\leq &\sum_{n=n_{1}}^{\infty}\sum_{t=1}^{\infty}P(\text{myopic action occurs regret for agents with }\nonumber \\
&\theta\in \{\theta:\theta\cdot u_{B(\theta)}-\max_{j\neq B(\theta)}\{\theta \cdot u_{j}\}\geq p^{-1}\left(\frac{p}{2}\right)\}|t_{n-1}\leq t\leq t_{n})P(t_{n-1}\leq t\leq t_{n})+Nn_{1} \nonumber \\
\leq & \sum_{n=n_{1}}^{\infty}24Nm \exp\left(\frac{-1.8 n\left(p^{-1}(\frac{p}{2})\right)^2}{32m^2\sigma^2}\right)\times Ng(n)+Nn_{1} \nonumber 
\end{align}

If we choose $g(n)=n$ and $\delta_{0}=p^{-1}(\frac{p}{2})$, then we have
\begin{align}
\frac{1}{n}<\frac{p}{2} \implies n\geq (\frac{2}{p})+1. \nonumber
\end{align}
We know the expected payment is bounded above by:
\begin{align}
E[C(T)] \leq 24N^2 m \frac{e^{\frac{1.8\delta_{0}}{32m^2\sigma^2}}}{(e^{\frac{1.8\delta_{0}}{32m^2\sigma^2}}-1)^2}+N(\max\{n_{0},(\frac{2}{p})+1\}). \nonumber
\end{align}

Since we have $N$ arms in total, we know the $\mathbb{E}[C(T)]=O(N^2)$.\end{proof}


Now we are ready to prove our second main result, Theorem~\ref{rst:regret}.

\begin{proof}
For regret occurred in the first $n_0$ round, it is bounded above by $\sum_{n=1}^{n_{0}}NRg(n)$.

For regret occurred after the first $n_0$ round, it has two different components: the regret occurred when we let the agents play myopically and the regret occured when we incentivize the agents. For the regret occurred when we let the agents play myopically at time t, it consists of the following two components:
\begin{itemize}
\item For those user whose utility difference between their best and the second best arm is greater than $f(t)$: suppose it is at the $n(t)_{th}$ round, the probability of these user making a mistake is bounded above by $24Nm\exp\left(-\frac{1.8n(t) f(t)^2}{32 m^2\sigma^2}\right)$ at time $t$ and the expected regret is bounded above by $24Nm\exp\left(-\frac{1.8n(t) f(t)^2}{32 m^2\sigma^2}\right)\times R$. We denote the regret occured by these agents as $r_1(t)$.
\item For those user whose utility difference between their best and the second best arm is smaller than $f(t)$: this happens with probability $S(f(t))$ at each time and regret is bounded above by $S(f(t)) \times f(t)=Mf(t)^2$. We denote the regret occured by these agents as $r_2(t)$.
\end{itemize}

Thus, the cumulative expected regret occured up to time $T$ when we let the agent play myopically is bounded above by:
\begin{align}
&E\left[\sum_{t=1}^{T}r(t)\right] \nonumber \\
=&E\left[\sum_{t=1}^{t_{n_{0}}} r(t) + \sum_{t=t_{n_{0}}}^{T}(r_1(t)+r_2(t))\right]  \nonumber \\
\leq & \sum_{n=1}^{n_{0}}NRg(n) + E\left[\sum_{t=t_{n_{0}}}^{T}r_1(t)\right]+ E\left[\sum_{t=t_{n_{0}}}^{T}r_2(t)\right] \nonumber \\
\leq & \sum_{n=1}^{n_{0}}NRg(n) + E\left[\sum_{n=n_{0}}^{T}\sum_{t=t_{n}}^{t_{n+1}}r_1(t)\right]+ E\left[\sum_{t=1}^{T}r_2(t)\right] \nonumber \\
\leq & \sum_{n=1}^{n_{0}}NRg(n) + \sum_{n=n_{0}}^{T}E\left[\sum_{t=t_{n}}^{t_{n+1}}24Nm\exp\left(-\frac{1.8n f(t)^2}{32 m^2\sigma^2}\right) R\right]+ \sum_{t=1}^{T}Mf(t)^2 \nonumber \\
\leq & \sum_{n=1}^{n_{0}}NRg(n) + \sum_{n=n_{0}}^{T}E\left[24Nm\exp\left(-\frac{1.8n f(t_n)^2}{32 m^2\sigma^2}\right)R \times (t_{n+1}-t_{n})\right]+ \sum_{t=1}^{T}Mf(t)^2 \nonumber \\
\leq & \sum_{n=1}^{n_{0}}NRg(n) + \sum_{n=1}^{T}E\left[24Nm\exp\left(-\frac{1.8n f(n)^2}{32 m^2\sigma^2}\right)R\times (t_{n+1}-t_{n})\right]+ \sum_{t=1}^{T}Mf(t)^2 \nonumber \\
\leq & \sum_{n=1}^{n_{0}}NRg(n) + \sum_{n=1}^{T} 24Nm\exp\left(-\frac{1.8n f(n)^2}{32 m^2\sigma^2}\right)\times R \times Ng(n)+ \sum_{t=1}^{T}Mf(t)^2 \nonumber
\end{align}

The expected regret occurred when we incentivize the agents is calculated in Theorem~\ref{rst:budget}. Thus if we set $g(n)=n$, the cumulative regret at time $T$ is bounded above by
\begin{align}
&\sum_{n=1}^{n_{0}}NRn + \sum_{n=1}^{T} 24Nm\exp\left(-\frac{1.8n f(n)^2}{32 m^2\sigma^2}\right)\times R \times Nn+ \sum_{t=1}^{T}Mf(t)^2 \nonumber \\
+ & 24N^2 m \frac{e^{\frac{1.8\delta_{0}}{32m^2\sigma^2}}}{(e^{\frac{1.8\delta_{0}}{32m^2\sigma^2}}-1)^2}+N(\max\{n_{0},(\frac{2}{p})+1\}). \nonumber
\end{align}

For a fixed $T$, we only need to minimizing the following two terms since all others are constant:

\begin{align}
\sum_{n=1}^{T} 24Nm\exp\left(-\frac{1.8n f(n)^2}{32 m^2\sigma^2}\right)\times R \times Nn+ \sum_{t=1}^{T}Mf(t)^2.
\end{align}


Basically, we need to choose $a$ to bound
\begin{align}
\sum_{t=1}^{T}\left[n\times exp(-n^a) + n^{-1+a}\right] \nonumber
\end{align}

\begin{align}
\int_{0}^{T}n^{-1+a}dn = \frac{T^{a}}{a} \nonumber 
\end{align}

If $a=\frac{1}{\log(T)}$, then

\begin{align}
&\sum_{t=1}^{T}\left[n\times exp(-n^a) + n^{-1+a}\right] \nonumber \\
= & \sum_{t=1}^{T} n\times exp(-n^a) + \frac{T^{a}}{a} \nonumber \\
\geq & \sum_{t=1}^{T} n\times exp(-T^a) + e\log(T) \nonumber \\
\geq & \sum_{t=1}^{T} n\times e^{-e} + e\log(T) \nonumber
\end{align}

\end{proof}


\subsection{Special Case}

In this section, we consider a special case of our problem, in which the agent's preference distribution $F(\cdot)$ satisfies $S(\delta)=0$ for $\delta<\delta^{'}$. Under this assumption, we still have $O(N^2)$ budget bound, however we get a better regret bound. The proof is similar to the proof of Theorem~\ref{rst:regret}, but for $t$ large enough, $S(f(t))=0$. Thus,

\subsection{Practical Issues}

In our algorithm, we take ``pay what it takes'' strategy when we decided to incentivize the agent. However, as long as 

\section{Lower Bound $O(\log(T))$}
\label{sec:lb}

In this section, we assume $\theta$ follows a continuous distribution $F(\cdot)$. We provide a example to show the best possible lower bound is $O(N\log(T))$.

Suppose you have two arms. Arm $1$ has utility vector $(0,0)$ and arm $2$ has utility vector $(0,1)$. We assume the users' preference are uniformly distributed on the unit circle. If the user knows the true reward for both arms, then the users with preference on the bottom half circle will choose arm $1$ and the users with preference on the top half circle will choose arm $2$.


Consider the following algorithm: at each step, let the agents play myopically; however, they are going to see the noisy rewards for both arms.

For simplicity, we assume that the agents already know the true utility vector for arm $1$. Without loss of generality, denote $u_{t} = (0,1)+(z_{t1},z_{t,2}) = (0,1)+ (N(0,1/n),N(0,1/n))$ to be the estimate reward for arm $2$ (I assume the variance for the noise is $1$). 


At time $n$, suppose we know arm $1$ has utility vector $(0,0)$ and arm $2$ has utility vector $u_n\sim (N(0,\frac{1}{n}), N(1,\frac{1}{n}))$. Since we know $(z_{t,1}, z_{t,2})$ is symmetric around $(0,0)$, we know 

\begin{align}
&E[r_n] \nonumber \\
= &E[r_n | z_{n,1}>0,z_{n,2}>0] \times P(z_{n,1}>0,z_{n,2}>0) + E[r_n |z_{n,1}>0,z_{n,2}<0] \times P(z_{n,1}>0,z_{n,2}<0) \nonumber \\
= &E[r_n | z_{n,1}<0,z_{n,2}>0] \times P(z_{n,1}<0,z_{n,2}>0) + E[r_n |z_{n,1}<0,z_{n,2}<0] \times P(z_{n,1}<0,z_{n,2}<0) \nonumber \\
\geq & 0.25 \times E[r_n | z_{n,1}>0, z_{n,2}>0]. \nonumber
\end{align}

Given $z_{n,1}>0$ and $z_{n,2}>0$, we know the user between $(-1,0)$ and $(\frac{-1-z_{n,2}}{\sqrt{z_{n,1}^2+(1+z_{n,2})^2}}, \frac{z_{n,1}}{\sqrt{z_{n,1}^2+(1+z_{n,2})^2}})$ and the user between $(1,0)$ to $(\frac{1+z_{n,2}}{\sqrt{z_{n,1}^2+(1+z_{n,2})^2}}, \frac{-z_{n,1}}{\sqrt{z_{n,1}^2+(1+z_{n,2})^2}})$ will make a mistake. And the regret is the absolute value of the second coordinate of the user's preference vector. Thus, we know

\begin{align}
&E[r_n| z_{n,1}>0, z_{n,2}>0] \nonumber \\
=& 4\times 2 \int_{0}^{\infty} \int_{0}^{\infty} \int_{0}^{\arctan(\frac{z_{n,1}}{1+z_{n,2}})}\frac{\sin(\theta)}{2\pi}d(\theta)\frac{e^{-\frac{n \times z_{n,1}^2}{2}}\sqrt{n}}{\sqrt{2\pi}}d(z_{n,1})\frac{e^{-\frac{n \times z_{n,2}^2}{2}}\sqrt{n}}{\sqrt{2\pi}}d(z_{n,2}) \nonumber \\
=& \frac{2}{\pi^2}\int_{0}^{\infty} \int_{0}^{\infty}n\times [1-\frac{1+z_{n,2}}{\sqrt{z_{n,1}^2+(1+z_{n,2})^2}}]e^{-\frac{n \times z_{n,1}^2}{2}}e^{-\frac{n \times z_{n,2}^2}{2}}d(z_{n,1})d(z_{n,2}) \nonumber \\
=& \frac{2}{\pi^2}\int_{0}^{\infty} \int_{0}^{\infty} \left[1-\frac{\sqrt{n}+z_{n,2}}{\sqrt{z_{n,1}^2+(\sqrt{n}+z_{n,2})^2}}\right]e^{-\frac{z_{n,1}^2}{2}}e^{-\frac{z_{n,2}^2}{2}}d(z_{n,1})d(z_{n,2}) \nonumber 
\end{align}

Below, we want to show 
\begin{align}
\lim_{n\rightarrow}\frac{E[r_n| z_{n,1}>0, z_{n,2}>0]}{n} = O(1). \nonumber
\end{align}

Denote $d(n)=n\left[1-\frac{\sqrt{n}+z_{n,2}}{\sqrt{z_{n,1}^2+(\sqrt{n}+z_{n,2})^2}}\right]$.
Since
\begin{align}
d^{'}(n)=\frac{-z_{n,1}^2(2z_{n,2}+3\sqrt{n})-2(z_{n,2}+\sqrt{n})^3}{2(z_{n,1}^2+(z_{n,2}+\sqrt{n})^2)^{3/2}}+1. \nonumber
\end{align}

and $\lim_{n\rightarrow \infty}d^{'}(n)=2$, we know for $n$ large enough, $d(n)$ is a increasing function in terms of n. Thus, based on the Monotone Convergence Theorem, we have

\begin{align}
&\lim_{n\rightarrow \infty}\frac{E[r_n| z_{n,1}>0, z_{n,2}>0]}{n} \nonumber \\
=& \frac{2}{\pi^2}\int_{0}^{\infty} \int_{0}^{\infty}\lim_{n\rightarrow \infty}\left[ \left[1-\frac{\sqrt{n}+z_{n,2}}{\sqrt{z_{n,1}^2+(\sqrt{n}+z_{n,2})^2}}\right]e^{-\frac{z_{n,1}^2}{2}}e^{-\frac{z_{n,2}^2}{2}}\right]d(z_{n,1})d(z_{n,2}) \nonumber  
\end{align}


Based on Wolframalpha, we know

\begin{align}
\lim_{n\rightarrow \infty} d(n)=\frac{z_{n,1}^2}{2}. \nonumber
\end{align}
Thus,
\begin{align}
&\lim_{n\rightarrow \infty}\frac{E[r_n| z_{n,1}>0, z_{n,2}>0]}{n} \nonumber \\
=&\frac{2}{\pi^2}\int_{0}^{\infty} \int_{0}^{\infty}\lim_{n\rightarrow \infty}\left[ \frac{z_{n,1}^2}{2}e^{-\frac{z_{n,1}^2}{2}}e^{-\frac{z_{n,2}^2}{2}}\right]d(z_{n,1})d(z_{n,2}) = \frac{1}{2\pi}. \nonumber
\end{align}


Thus, the cumulative expected regret is at least $O(\log(T))$.


\section{Conclusion}


\bibliography{reference}{}
\bibliographystyle{plain}


\end{document}

