\section*{Proof of Theorem~\ref{rst:budget}}

We also need the following lemma in part of the proof of Theorem~\ref{rst:budget}.
\begin{lemma}
For all $n\geq 1$, we have
\begin{align}
0.9n^{5/6} \geq \sqrt{0.6n \log(\log_{1.1}(n)+1)}. \nonumber
\end{align}
\label{lemma:cal2}
\end{lemma}
\begin{proof}
\begin{align}
&0.9n^{5/6} \geq \sqrt{0.6n \log(\log_{1.1}(n)+1)} \nonumber \\
\iff & 0.81 n^{5/3} \geq 0.6n \log(\log_{1.1}(n)+1) \nonumber \\
\iff &  \frac{81}{60} n^{2/3} \geq  \log(\log_{1.1}(n)+1) \nonumber
\end{align}

Denote $f(x) = \frac{81}{60}x^{2/3} - \log(\log_{1.1}(x)+1)$. It's easy to compute $f^{'}(x)=0$ has a unique solution $x_{0}=e^{2/3w(\frac{20e^{20000/314763}}{27})-\frac{10000}{104921}}$ (here $w(\cdot)$ is the Lambert W-Function) and it is the global minimum. Since $f(x_{0})\approx 0.0252 >0$, we know $f(x)>0$ for all $x\geq 1$. Thus, our lemma holds true.
\end{proof}


Now we are ready to prove our first main result, Theorem~\ref{rst:budget}.

\begin{proof}

\noindent\textbf{Step 1: Categorize measurement errors into different radius envelopes}


Denote $\epsilon_{i,t}=u_{i,t}-u_{i}$ to be the estimation error for the attribute vector $u_i$ at time $t$. Denote $\epsilon_{i,t}^{j}$ to be the $j^{th}$ component of $\epsilon_{i,t}$. Denote $\omega$ to be a sample path and $n(t,\omega)$ to be the round number for sample path $\omega$ at time $t$. For a fixed time $t$, define
\begin{align}
L^{'}[\ell](t) = \{\omega:|\epsilon_{i,t}^{j}(\omega)|\leq g(n(t,\omega),\ell), \forall i,j\}\nonumber
\end{align}
where $g(n,\ell)$ is a function which we will define later. Define $L[1](t) = L^{'}[1](t)$ and $L[i](t) = L^{'}[i](t)\setminus L^{'}[i-1](t)$ for $i\geq 2$. We call $L[\ell](t)$ the $\ell^{th}$ envelope at time $t$. We often simplify the notation and use $L[\ell]$ instead of $L[\ell](t)$ without confusion.

In the calculation below, we omit the dependency on $\omega$ when refering to variables $c(t)$, $\epsilon_{i,t}^{j}$ and $t_n$. Based on the definition of $L[\ell]$, we know if $\omega\in L[\ell]$, the maximum payment we need to offer at time $t$ is bounded above by 
\begin{align}
&\max_{i}\theta_t\cdot u_{i,t} - \min_{j}\theta_t\cdot u_{j,t} \nonumber \\
= &\max_{i}\theta_t\cdot (\epsilon_{i,t}+u_i) - \min_{j}\theta_t\cdot (\epsilon_{j,t}+u_j) \nonumber \\
\leq &\max_{i}\theta_t\cdot u_i - \min_{j}\theta_t\cdot u_j +\max_{i}\theta_t\cdot \epsilon_{i,t} - \min_{j}\theta_t\cdot \epsilon_{j,t}\nonumber \\
\leq & R + 2Wmg(n,\ell). \nonumber
\end{align}

We denote $\bar{c}(n,\ell)=R + 2Wmg(n,\ell)$ as the upper bound of payment for round $n$ if our measurement error lie in the $\ell^{th}$ envelope.


\noindent\textbf{Step 2: Introduce a new stochastic process which bounds the total payment in a round}

Denote $\mathcal{F}_t$ as the filtration up to time $t$. Based on our algorithm, we know

\[ P(\text{play a new arm}|\mathcal{F}_t) =
  \begin{cases}
      1       & \quad \text{if we incentivize}\\
          \sum_{i\notin (V\cup S)} P(\theta \cdot u_{i,t}>\theta\cdot u_{j,t} \forall j\neq i)  & \quad \text{if }  S\setminus V\neq \emptyset 
            \end{cases} \label{dom_stoc}
            \] 
            \hspace{1cm}$\geq (n-|V|)\times \frac{1}{n}=1-\frac{|V|}{n}$.

            Let $(Z_{n,v,m}:m\geq 0)$ be a sequence of independent Bernoulli random variable with success probability $(1-\frac{v}{n})$. We will construct an alternative stochastic process for selecting which arm gets played that has the same distribution as the original process, but under which
            \begin{align}
            t_{n+1}-t_{n}\leq \bar{T}_{n}:=\sum_{v=0}^{N-1}\bar{T}_{n,v}, \nonumber 
            \end{align}
            where $\bar{T}_{n,v}:=\inf\{m\geq 0: Z_{n,v,m}=1\}+1$.

            The new stochastic process will have the property that whenever $Z_{n,v(n,t),m(n,t)}=1$, we will play a new arm at time t for $t\in [t_{n}, t_{n+1}]$, where $v(n,t)$ is the number of unique arms played in round $n$ strictly before time $t$, and $m(n,t)$ is the number of times we have pulled a previously pulled arm for the current value of $v(n,t)$. At time $t$, to determine what arm to pull, calculate \eqref{dom_stoc} and let $q_t$ be the probability computed. Note $q_t\geq 1-\frac{v(n,t)}{n}$.

            If $Z_{n,v(n,t),m(n,t)}$ is 1, decide to play a new arm. Otherwise, draw a second Bernoulli random variable with probability $\frac{q_t-(1-\frac{v(n,t)}{n})}{1-(1-\frac{v(n,t)}{n})}$, and if it is 1, decide to play a new arm, and otherwise decide to play an old arm. Note that

            \begin{align}
            P(\text{play a new arm}|\mathcal{F}_{t})=\left[1-\frac{v(n,t)}{n}\right]+\left[1-(1-\frac{v(n,t)}{n})\right]\times \frac{q_t-(1-\frac{v(n,t)}{n})}{1-(1-\frac{v(n,t)}{n})}=q_t.
            \end{align}

            Since we pull a new arm only when $Z_{n,v(n,t),m(n,t)}=1$, the number of times we pull an arm when $v(n,t)=v$ is bounded above by $\bar{T}_{n,v}$. Thus,

            \begin{align}
            t_{n+1}-t_{n}\leq \sum_{v=0}^{n-1}\bar{T}_{n,v}=\bar{T}_{n}.
            \end{align}
            Finally, we want to bound $\sum_{t}E[c(t)]$. Let $C(n)$ be the cost incurred in round $n$. Thus $\sum_{t}E[c(t)]=\sum_{n}E[C(n)]$. We know
            \begin{align}
            C(n)&=\sum_{t=t_n}^{t_{n+1}-1}c(t)=\sum_{t=t_{n}}^{t_{n+1}-1}\sum_{\ell}\mathbbm{1}\{\omega\in L(\ell)\}c(t) \nonumber \\
            &\leq \sum_{t=t_{n}}^{t_{n+1}-1}\sum_{l}\mathbbm{1}\{\omega\in L(\ell)\}\bar{c}(n,\ell) \nonumber \\
            &\leq (t_{n+1}-t_{n})\sum_{\ell} \mathbbm{1}\{\omega\in L(\ell)\}\bar{c}(n,\ell) \nonumber \\
            &\leq \bar{T}_{n}\sum_{l}\mathbbm{1}\{\omega\in L(\ell)\}\bar{c}(n,\ell). \nonumber
            \end{align}

            Since $\bar{T}_{n}$ is independent of $\omega$, we know

            \begin{align}
            E[C(n)]\leq E[\bar{T}_{n}]\sum_{\ell}P(\omega\in L(\ell))\bar{c}(n,\omega). \nonumber
            \end{align}

            \noindent\textbf{Step 3: Rewrite the total payment expression}.

            Based on the above notations, we can rewrite the cumulative payment as follows:
            \begin{align}
            \sum_{t=1}^{\infty}c(t) =\sum_{n=1}^{\infty}C(n)
            \leq  \sum_{n=1}^{\infty}\sum_{\ell=1}^{\infty}\bar{T}_{n}\mathbbm{1}\{\omega\in L(\ell)\}\bar{c}(n,\ell). \nonumber
            \end{align}

            Set $g(n,\ell)$ to be $\frac{2\sigma \ell}{n^{1/6}}$. Since if $|u_{i}^{j}-u_{i,t}^{j}|\leq \lambda$ is true $\forall i$, $\forall j$, then we know for those $\theta\in \{\theta:\theta\cdot u_{B(\theta)}-\max_{j\neq B(\theta)}\{\theta \cdot u_{j}\}> 2Wm\lambda\}$, they will correctly identify their best arm. Thus, if $|u_{i}^{j}-u_{i,t}^{j}|\leq \frac{2\sigma l}{n^{1/6}} \leq \frac{p^{-1}(\frac{p}{2})}{2Wm}$ $\forall i$ and $\forall j$, then the probability that an unincentivized agent would pull arm $i$ is at least $\frac{p}{2}$. Further, if time $t$ is in a round $n$ that satisfies $n^{-1}\leq p/2$, then our algorithm will not incentivize pulling any arms. Denote $a_0=\frac{4Wm\sigma}{p^{-1}(\frac{p}{2})}$. In order to have $\frac{2\sigma l}{n^{1/6}}\leq \frac{p^{-1}(\frac{p}{2})}{2Wm}$, it is sufficient to have $n\geq \lceil (a_{0} l)^6 \rceil$. In order to have $n^{-1}\leq \frac{p}{2}$, we need $n\geq \frac{2}{p}$. Denote $n_2=\frac{2}{p}$. Thus, we know we can only incur regret for sample paths $\omega$ in the $\ell^{th}$ envelope in the first $\max\{n_2,\lceil (a_0 \ell)^6 \rceil\}$ rounds.

            Thus,
            \begin{align}
            \sum_{t=1}^{\infty}c(t)\leq\sum_{\ell=1}^{\infty}\sum_{n=1}^{\max\{n_2,\lceil (a_0 \ell)^6 \rceil\}}\bar{c}(n,\ell)\mathbbm{1}\{\omega\in L[\ell]\}\bar{T}_{n}. \nonumber
            \end{align}

            Therefore,

            \begin{align}
            &E\left[\sum_{t=1}^{\infty}c(t)\right] \nonumber\\
            \leq &\sum_{\ell=1}^{\infty}\sum_{n=1}^{\max\{n_2,\lceil (a_0 \ell)^6 \rceil\}}\bar{c}(n,\omega)P(\omega \in L(\ell))E[\bar{T}_{n}] \nonumber \\
            =&\sum_{\ell=1}^{\infty}\sum_{n=1}^{\max\{n_2,\lceil (a_0 \ell)^6 \rceil\}}\left[R+2Wm\frac{2\sigma \ell}{n^{1/6}}\right]P(\omega\in L[\ell])E[\bar{T}_{n}] \nonumber \\
            \leq &\sum_{\ell=1}^{\infty}\sum_{n=1}^{\max\{n_2,\lceil (a_0 \ell)^6 \rceil\}}\left[R+4Wm\sigma l\right]P(\omega\in L[\ell])E[\bar{T}_{n}]. \nonumber
            \end{align}



            \noindent\textbf{Step 4: Bound $P(\omega\in L(\ell))$}.

            We now bound $P(\omega\in L[\ell])$ for $n\geq n_0$ and $\ell\geq 2$. As a reminder, we omit the dependency between $\epsilon_{i,t}^{j}$, $n$ and $\omega$. We know 
            \begin{align}
            &P(\omega\in L[\ell]) \nonumber \\
            =&P(\omega\in L^{'}[\ell])- P(\omega\in L^{'}[\ell-1]) \nonumber \\
            \leq & 1-P\left(|\epsilon_{i,t}^{j}|<\frac{2\sigma (\ell-1)}{n^{1/6}}, \forall i,j\right) \nonumber \\
            =&P\left(\exists i,j, s.t. |\epsilon_{i,t}^{j}|\geq \frac{2\sigma (\ell-1)}{n^{1/6}}\right) \nonumber  \\
            \leq &\sum_{i,j}P\left(|\epsilon_{i,t}^{j}|\geq \frac{2\sigma (\ell-1)}{n^{1/6}}\right) \nonumber
            \end{align}

            Define $S_{i,t}^{j}=\frac{N(i,t)\epsilon_{i,t}^{j}}{2\sigma}$, then we know $S_{i,t}^{j}$ is a summation of $1/2$ gaussian random numbers. Therefore,

            \begin{align}
            &\sum_{i,j}P\left(|\epsilon_{i,t}^{j}|\geq \frac{2\sigma(\ell-1)}{n^{1/6}}\right) \nonumber \\ 
            =&\sum_{i,j}P\left(|S_{i,t}^{j}|\geq \frac{N(i,t)(\ell-1)}{n^{1/6}}\right) \nonumber \\
            \leq &\sum_{i,j}P(|S_{i,t}^{j}|\geq N(i,t)^{5/6}(\ell-1)). \nonumber
            \end{align}

            Based on Lemma~\ref{lemma:cal2}, we know
            \begin{align}
            &N(i,t)^{5/6}(\ell-1) \nonumber \\
            =& 0.9N(i,t)^{5/6} + N(i,t)^{5/6}(\ell-1.9) \nonumber \\
            \geq & \sqrt{0.6N(i,t)\log(\log_{1.1}(N(i,t))+1)} + \sqrt{(\ell-1.9)^2 N(i,t)} \nonumber \\
            \geq & \sqrt{0.6N(i,t)\log(\log_{1.1}(N(i,t))+1)+(\ell-1.9)^2 N(i,t)}. \nonumber
            \end{align}

            Based on Lemma~\ref{ACI-inequality}, we know
            \begin{align}
            & P(\omega\in L(\ell) \nonumber \\
            \leq &\sum_{i,j}P(|S_{i,t}^{j}|\geq N(i,t)^{5/6}(\ell-1)) \nonumber \\
            \leq & \sum_{i,j}24e^{-1.8(\ell-1.9)^2} = 24Nme^{-1.8(\ell-1.9)^2}. \nonumber
            \end{align}

            \noindent\textbf{Step 5: Final Step}

            Thus,
            \begin{align}
            &\sum_{t=1}^{\infty}c(t) \nonumber \\
            \leq&\sum_{\ell=1}^{\infty}\sum_{n=1}^{\max\{n_2,\lceil (a_0 \ell)^6 \rceil\}}\left[R+4Wm\sigma \ell\right]P(\omega\in L[\ell])E[\bar{T}_{n}] \nonumber \\
            \leq& \sum_{n=1}^{\max\{n_2,\lceil (a_0)^6 \rceil\}}\left[R+4Wm\sigma \right]P(\omega\in L[1])E[\bar{T}_{n}] +\sum_{l=2}^{\infty}\sum_{n=1}^{\max\{n_2,\lceil (a_0 \ell)^6 \rceil\}}\left[R+4Wm\sigma \ell\right]P(\omega\in L[\ell])E[\bar{T}_{n}] \nonumber \\
            \leq& \sum_{n=1}^{\max\{n_2,\lceil (a_0)^6 \rceil\}}\left[R+4Wm\sigma \right]E[\bar{T}_{n}] +\sum_{\ell=2}^{\infty}\sum_{n=1}^{\max\{n_2,\lceil (a_0 \ell)^6 \rceil\}}\left[R+4Wm\sigma \ell\right]24Nme^{-1.8(\ell-1.9)^2}E[\bar{T}_{n}]\nonumber \\
            \leq& \sum_{n=1}^{\max\{n_2,\lceil (a_0)^6 \rceil\}}\left[R+4Wm\sigma \right]Nn +\sum_{\ell=2}^{\infty}\sum_{n=1}^{\max\{n_2,\lceil (a_0 \ell)^6 \rceil\}}\left[R+4Wm\sigma \ell\right]24Nme^{-1.8(\ell-1.9)^2}Nn \nonumber \\
            \leq & \left[R+4Wm\sigma \right]N (\max\{n_2,\lceil (a_0)^6 \rceil\})^2+\sum_{\ell=2}^{\infty}24N^2 m[R+4Wm\ell\sigma](\max\{n_2,\lceil (a_0 \ell)^6 \rceil\})^2 e^{-1.8(\ell-1.9)^2} \nonumber \\
            =& O(N^2). \nonumber
            \end{align}

            \end{proof}

            \section*{Calculation for Equation~\eqref{ex:limit}}

            In this section, we prove $d(t)\leq \frac{z_{t,1}^{2}}{2}$ and $\lim_{t\rightarrow\infty}d(t)=\frac{z_{t,1}^2}{2}$. Denote $w(x)=\sqrt{x}$. Then we know $w(x+y)\leq \sqrt{x} + y\frac{1}{2\sqrt{x}}$ and $w(x+y)\geq \sqrt{x} + y\frac{1}{2\sqrt{x+y}}$ for $y\geq 0$. Thus, we know

            \begin{align}
            &\lim_{t\rightarrow\infty} t\left[1-\frac{\sqrt{t}+z_{t,2}}{\sqrt{z_{t,1}^2+(\sqrt{t}+z_{t,2})^2}}\right] \nonumber \\
            =& \lim_{t\rightarrow\infty} t\left[\frac{\sqrt{z_{t,1}^2+(\sqrt{t}+z_{t,2})^2}-\sqrt{t}-z_{t,2}}{\sqrt{z_{t,1}^2+(\sqrt{t}+z_{t,2})^2}}\right]  \nonumber \\
            \geq & \lim_{t\rightarrow\infty} t\left[\frac{\sqrt{t}+z_{t,2}+z_{t,1}^{2}\frac{1}{2\sqrt{z_{t,1}^2+(\sqrt{t}+z_{t,2})^2}}-\sqrt{t}-z_{t,2}}{\sqrt{z_{t,1}^2+(\sqrt{t}+z_{t,2})^2}}\right]  \nonumber \\
            = & \lim_{t\rightarrow\infty} t \times \frac{z_{t,1}^2}{2(z_{t,1}^2+(\sqrt{t}+z_{t,2})^2)} \rightarrow \frac{z_{t,1}^2}{2}. \nonumber 
            \end{align}
and
            \begin{align}
            &\lim_{t\rightarrow\infty} t\left[1-\frac{\sqrt{t}+z_{t,2}}{\sqrt{z_{t,1}^2+(\sqrt{t}+z_{t,2})^2}}\right] \nonumber \\
            =& \lim_{t\rightarrow\infty} t\left[\frac{\sqrt{z_{t,1}^2+(\sqrt{t}+z_{t,2})^2}-\sqrt{t}-z_{t,2}}{\sqrt{z_{t,1}^2+(\sqrt{t}+z_{t,2})^2}}\right]  \nonumber \\
            \leq & \lim_{t\rightarrow\infty} t\left[\frac{\sqrt{t}+z_{t,2}+z_{t,1}^{2}\frac{1}{2\sqrt{(\sqrt{t}+z_{t,2})^2}}-\sqrt{t}-z_{t,2}}{\sqrt{z_{t,1}^2+(\sqrt{t}+z_{t,2})^2}}\right]  \nonumber \\
            = & \lim_{t\rightarrow\infty} t \times \frac{z_{t,1}^2}{2\sqrt{z_{t,1}^2+(\sqrt{t}+z_{t,2})^2}(\sqrt{t}+z_{t,2})} \rightarrow \frac{z_{t,1}^2}{2}. \label{inequ:dct}
            \end{align}   
Based on inequality~\eqref{inequ:dct}, we know $d(t)\leq \frac{z_{t,1}^{2}}{2}$ and $\lim_{t\rightarrow\infty}d(t)=\frac{z_{t,1}^2}{2}$.


