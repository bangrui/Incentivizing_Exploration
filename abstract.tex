We consider the problem of incentivized exploration with heterogeneous agents.
In this problem, bandit arms provide vector-valued outcomes equal to an unknown
arm-specific attribute vector perturbed by independent normally distributed
noise.  Agents arrive sequentially and choose arms to pull based on their own
private and heterogeneous linear utility function over attributes and estimates
of the arms' attribute vectors derived from observations of other agents' past
pulls.  Agents are myopic and selfish and thus would choose the arm with
maximal estimated utility, except for an incentive payment that a principal may
offer to encourage agents to explore underplayed arms.  The principle seeks to
minimize the expected cumulative regret incurred by agents relative to their
best arms, while also making a small expected cumulative payment.  We propose
an algorithm that incentivizes arms played infrequently in the
past whose probability of being played in the next round would be small
without incentives.  Under the assumption that each arm is preferred by at
least a fraction $p>0$ of agents, we show this algorithm achieves expected
cumulative regret of $O(N\exp(2/p) + N \log(T)^3)$ using expected cumulative
payments of $O(N^2 \exp(2/p))$.  If $p$ is known or the distribution over agent
preferences is discrete, the term $\exp(2/p)$ can be eliminated.  In these
cases, the bound on payment improves to $O(N^2 / p^{5/2})$.  The bound on
regret improves to $O(N/p^4 + N^2/p^2 + N \log(T)^3 / p)$ for known $p$ and to
$O(N/p^4 + N^2/p^2)$ for discrete preference distributions.  Regret and
payments that are both constant in $T$ for discrete preferences stands
in contrast to the $\log(T)$ dependence on regret in standard multi-armed bandit
problems, and arises because heterogeneity in agent preferences allows
exploitation of arms to also explore fully.
Succinctly, heterogeneity provides free exploration.
